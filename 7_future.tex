\chapter{Future work}
\label{chapter:future}
This thesis has contributed a new ULP system stack that opens up many future research directions.
% 
In particular, the success of \snafu and \riptide make it possible to consider research questions besides those having to do with energy-efficiency.
% 
This chapter will discuss future work on improving area efficiency, performance, and programmability of CGRAs.

\section{Quantifying the progress made}

\figFutureModel
\subsection{Is compute energy-efficiency still a bottleneck?}
To answer the question of whether the energy-efficiency of compute is still the bottleneck in \riptide-based ULP sensor systems, we modelled device lifetime of such systems as function of problem size.
%
Device lifetime is directly related to energy-efficiency (in battery-powered systems) and problem size is a proxy for the amount of compute required and the amount of data that needs to be transmitted.
% 
\autoref{fig:future:model} shows two \riptide-based systems (red) as well as a system that transmits all data collected (blue), a system similar to \sonic that relies on a scalar MCU (green), and a theoretical system that achieves 10TOPS/W (black).
% 
Each system is composed of 1) a single AA battery, 2) a theoretical sensor (based on HM01B0 ULP camera), 3) an ULP MCU (e.g. scalar, \riptide, or theoretical), and 4) a bluetooth (BLE) radio.
% 
Besides the transmit-all configuration, each system is modelled to run a neural network similar to those in \sonic to smartly discard uninteresting data.
% 
Further, \riptide (Summary) models an application that only sends a short summary of captured data; this could be the result of classification (i.e. class label) or it could be a fragment of data deemed interesting.

If the goal is to achieve a device lifetime of five years, while processing a QQVGA frame once every five seconds, 


\section{Future research directions}
	\subsection{Area}
		\paragraph{Time-multiplexing}
		\paragraph{Alternative control-flow models}
		\paragraph{Optimal PE mix}

	\subsection{Performance}
		\paragraph{Parallelization}
			- top-level control/scheduling
		\subsubsection{Programming the memory}
			\paragraph{Memory hierarchy}
				- coherence
			\paragraph{Scratchpad management}
			\paragraph{Memory partitioning}
		\subsubsection{Speculation}
			\paragraph{Loop speculation}
			\paragraph{Memory speculation}
		\subsubsection{Compiler optimizations}
			\paragraph{Code scheduling}
			\paragraph{Loop unrolling}
			\paragraph{Loop auto-parallelization}

	\subsection{Programmability}
		\paragraph{Choosing the right IR}
		\paragraph{Scheduling}
		\paragraph{Annotations}
		\paragraph{DSLs}

\section{Long-term implications}
\graham{should this be the conclusion?}
\paragraph{Energy-efficiency matters at all scales}
\paragraph{Extreme heterogeneity is not the answer}
\paragraph{At last mass adoption of pure dataflow}
