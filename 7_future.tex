\chapter{Future work}
\label{chapter:future}
This thesis has contributed a new ULP system stack that opens up many future research directions.
% 
In particular, the success of \snafu and \riptide makes it possible to consider research questions besides those to do with energy efficiency.
% 
This chapter will discuss future work on improving area efficiency, performance, and compilation in the context of CGRAs.

\section{Quantifying the progress made}
\figProgress
\autoref{fig:progress} quantifies the significant progress that this thesis has made towards a new ULP, energy-minimal sensor system stack.
%
We compare an MSP430, Cortex-M3 (STM32L15RE~\cite{stm32l1}), and our scalar design to \riptide running neural network inference (the network is a derivative of LeNet~\cite{lecun:ieee89:lenet}).
% 
For the MSP430 and the Cortex-M3, we run the network on-device to measure latency and estimate power using a digital mulitimeter. 
% 
For our scalar design and \riptide, we implement each using a high-threshold voltage, industrial-grade process and use a post-synthesis annotated switching model to estimate power (same methodology as in~\autoref{riptide:method}).
% 
\autoref{fig:progress:energy} shows energy savings: our scalar design is already an aggressive baseline, saving $292\times$ energy v. MSP430 and Cortex-M3; \riptide improves even further, saving a massive $1913\times$.
% 
Even adjusting for process technology, \riptide saves $\approx 323 \times$ energy v. MSP430, suggesting that a significant portion of energy savings comes from the architecture and not from process technology scaling.
% 
\autoref{fig:progress:eff} and \autoref{fig:progress:perf} show similar trends in efficiency and speedup.
% 
\riptide achieves $64MOPS/mW$, $6.7\times$ greater than our scalar design, which is already $73\times$ better than the Cortex-M3.
% 
It is also $146\times$, $12.5\times$, and $7.6\times$ faster than the MSP430, Cortex-M3, and our scalar design, respectively.
% 
Together the energy savings, efficiency, and speedup of \riptide v. MSP430, Cortex-M3, and our scalar design represent significant progress and make \riptide an ideal platform for new applications in the ULP domain.

\figFutureModel
\subsection{Is compute energy efficiency still a bottleneck?}
With the progress that has been made, the natural question is whether the energy efficiency of compute is still the bottleneck in \riptide-based ULP sensor systems.
% 
To answer this question, we modeled device lifetime of such systems as a function of problem size with goal of achieving a five year lifetime processing a QQVGA ($160\times120$) frame once every five seconds.
%
Device lifetime is directly related to energy efficiency (in battery-powered systems), and problem size is a proxy for the amount of compute required and the amount of data that needs to be transmitted.
% 
\autoref{fig:future:model} shows two \riptide-based systems (red) as well as a system that transmits all data collected (blue), a system similar to \sonic that relies on a scalar MCU (green), and a hypothetical system that achieves 10TOPS/W (black).
% 
Each system is composed of 1) a single AA battery, 2) a sensor (similar to the HM01B0 ULP camera), 3) an ULP MCU (e.g., scalar, \riptide, or hypothetical), and 4) a bluetooth (BLE) radio.
% 
Besides the transmit-all configuration, each system is modeled to run a neural network similar to those in \sonic to smartly discard uninteresting data.
% 
Assuming interesting events are somewhat rare, this means these systems transmit infrequently, approximately every $20$ minutes (v. 5s interval of sensor readings).
% 
Finally, \riptide (Summary) models an application that only sends a short summary of captured data; this could be the result of classification (i.e., class label) or it could be a fragment of data deemed interesting.

% If the goal is to achieve a device lifetime of five years, while processing a QQVGA ($160\times120$) frame once every five seconds, 
\riptide-based systems achieve the five-year lifetime target while processing QQVGA frames, while the transmit-all system and the \sonic-like, scalar-MCU system fall well short.
% 
In fact, the \riptide-based system (lighter red) actually gets quite close to the hypothetical system that achieves a much higher efficiency of 10 TOPS/W.
% 
This suggests that the energy efficiency of compute is no longer the bottleneck --- rather, the energy efficiency of the radio is now more important.
% 
\riptide (Summary) (dark red) supports this conclusion; efficient onboard compute that summarizes and/or compresses sensor data, minimizes communication energy and extends lifetime even more. 
% 
Thus, \riptide has solved the problem of compute effiency for a variety of applications, opening the door to exploring to new problems.
% future ULP sensor systems need only maintain \riptide's level of energy-efficiency, while addressing other open problems.

\section{Future research directions}
Many open problems remain in ULP sensor systems and specifically power-constrained CGRAs.
% 
They can be classified into three categories: increasing area efficiency, scaling performance, and expanding programmability.
% 
Progress on each of these fronts will be important to drive adoption.

\subsection{Area}
\riptide's programmability and energy efficiency come at the cost of area.
% 
\riptide is larger than several ASICs (e.g., {\tt dmm}, {\tt fft}, and {\tt sort}) combined.
% 
Thus, improving area efficiency is an important goal for future designs, especially to lower fabrication costs and make even larger, more-capable fabrics competitive.

\paragraph{Time-multiplexing}
Area efficiency can be improved by increasing resource utilization through time-multiplexing.
% 
\riptide only supports a single operation per PE, so resources are underutilized when operations fire rarely.
% 
This is especially true for operations in outer loops.
% 
Mapping and time-multiplexing multiple operations on the same PE, will significantly improve utilization.
% 
This increases area efficiency, makes it possible to map larger programs onto smaller fabrics, and potentially decreases compilation times by simplifying mapping constraints.

There are two possible approaches: fine-grain~\cite{weng2020hybrid} and coarse-grain~\cite{nguyen2021fifer}.
% 
In the fine-grain approach, multiple operations share a single PE and incoming operands trigger reconfiguration of the PE.
% 
In the coarse-grain approach, a kernel is split into multiple subkernels with the CGRA switching between them when progress is stalled on the active subkernel.
% 
The key to amortizing the cost of coarse-grain reconfiguration
% , especially for programs split between outer and inner loops, 
is for the CGRA to buffer several inputs so when reconfiguration occurs, multiple instances of the subkernel can be initiated.

There is a role for the compiler in both the fine- and coarse- grained approaches.
% 
For fine-grain multiplexing, the compiler needs to determine which operations should multiplex.
% 
This may mean reasoning about the program's critical path, only time-multiplexing operations off this path.
% 
For coarse-grain multiplexing, the compiler needs to determine where to split a kernel, while considering live-in and live-out values and the cost of reconfiguration.

\paragraph{Alternative control-flow models}
Area efficiency can also be improved by reducing the resources needed by a program.
% 
In particular, supporting arbitrary control flow requires allocating a significant amount of resources (especially routing) to control-flow operations.
% 
Alternative control-flow models, therefore, may improve area efficiency if they require fewer operations.
% 
\riptide's steering-based ($\phi^{-1}$) control-flow model requires steering gates for every incoming value in untaken branches.
% 
Instead, there may be situations where selection ($\phi$) or even predication use less resources and/or achieve better performance.
% 
It is also possible to mix these control-flow models to minimize area, maximize resource utilization, or even increase performance.

\paragraph{Optimizing the topology}
Finally, an additional way to improve area efficiency is to optimize the CGRA topology and resource mix so that they better match the requirements of applications~\cite{revamp,melchert2021automated}.
% 
For example, a CGRA fabric could be specialized~\cite{weng2020dsagen} for a set of applications, like linear algebra kernels, to significantly reduce area ($>2\times$).
% 
This entails merging the computation graphs of a set of applications to form a minimally sized common graph that can be used to generate CGRA hardware.
% 
The common graph, however, is only as representative as the set of applications are.
% 
Thus, there is a trade-off between area and programmability that will require collaboration between hardware designers and application programmers to strike the right balance.

\subsection{Performance}
Performance is another dimension future designs could seek to improve.
% 
CGRAs like \riptide already outperform (in-order) scalar and vector designs by unlocking a large amount instruction-level parallelism.
% 
But there are other ways to improve performance, including thread-level parallelism, speculation, caching (and other memory hierarchy optimizations) and compiler optimizations (e.g., loop unrolling, etc.).
% 
Scaling up the performance of \riptide-like designs could make them competitive in different computing domains like wearables or mobile phones.

\subsubsection{Parallelization}
Exploiting thread-level parallelism is a natural next step to boost performance for power-constrained CGRAs.
% 
Multiple threads could be mapped to the same fabric, running in parallel on different PEs or time-multiplexing onto the same PEs.
%
Multiple threads could also be mapped to separate fabrics. 
% 
\riptide is tiny ($<0.5mm^2$) compared to even a wearable CPU and could be replicated hundreds of times to support many threads in parallel.
% 
This design offers two significant benefits over vector processors/GPUs.
% 
First, threads would not run in lockstep.
% % 
This means threads would better exploit instruction-level parallelism and would be able to diverge without under-utilizing resources.
% It also means that threads could better exploit instruction level parallelism.
Second, the design could scale better v. vector processors/GPUs, since it does not rely on a monolithic register file.
% % 
% Second, thread divergence is better supported since threads would not run in lockstep.
% % 
% And third, the design would be able to exploit the substantial amount of instruction-level parallelism.
% 
With that said, there are many interesting challenges with this design.
% 
Questions remain regarding thread scheduling and synchronization, memory-hierarchy design, and the programming interface, to name a few.

\subsubsection{Programming the memory}
As performance is scaled with more threads, memory becomes a bottleneck.
% 
New memory hierarchies and programming models need to be developed to improve data placement, which will minimize data movement and maximize utilization of available memory bandwidth.

\paragraph{Memory hierarchy}
Future CGRA memory hierarchies will be composed of both caches and scratchpads.
% 
Caches simplify the programmer's job, but raise the question of coherence.
% 
Coherence is especially challenging since CGRAs (potentially running many different threads) may have tens or even hundreds of memory accessors.
% 
Scratchpads offer an alternative, but complicate the job of the programmer.
% 
The CGRA needs to a provide a rich set of synchronization primitives so that code is correct and performant.

\paragraph{Programming models}
Memory management can be simplified with novel programming models.
% 
Using higher-level abstractions (e.g., functional languages or DSLs) might allow the compiler to generate code to manage a hierarchy of scratchpads.
%
These interfaces and/or programmer annotations might also allow the compiler to co-locate data with computation by suggesting efficient ways of partitioning data.
% 
This is especially important for future CGRA fabrics that might contains hundreds of PEs.
% 
Partitioning memory to physically locate computation close to data improves performance and energy-efficiency by minimizing data movement.

\subsubsection{Speculation}
Speculation is another lever future CGRA designs could pull to improve performance.
%
The key is to dynamically eliminate sequentializing dependencies without costing significant resources or energy on misspeculations.

\paragraph{Loop speculation}
% One form of speculation of interest is loop speculation --- speculating on the next iteration(s) of a loop.
Speculating on loop-carried dependencies would improve performance by minimizing initiation interval.
% 
It would also increase resource utilization as there would be more work in flight.
% This would increase resource utilization as there would be more work in flight.
% 
Further, \riptide already maintains the order of loop iterations so misspeculations could be handled by restoring fabric state from a checkpoint of an earlier loop iteration.
% 
However, memory accesses become more expensive: 
% values need to be tagged with the loop iteration and 
speculative stores would need to be buffered and subsequent loads checked against these buffered stores.

\paragraph{Memory speculation}
Another form of speculation that would boost performance is memory speculation.
% 
 % and specifically speculating that memory operations will not alias.
% 
It is difficult for the compiler to prove memory operations will not alias, resulting in extra dependencies between operations.
% 
These dependencies, like loop-carried dependencies in loop speculation, sequentialize execution and may lengthen critical paths.
% 
However, many memory operations, at runtime, will not access the same locations,  so enforcing the dependencies between them wastes time and resources.
% 
Instead, a transactional memory system that speculates on whether memory operations will alias could make sense.
% 
Detecting true aliases and recovering after an aborted transaction, though, are challenging especially in CGRAs where, unlike e.g., OoO cores, there may be no centralized control to initiate a flush.

\subsubsection{Code refactoring}
The programmer and the compiler can also play a role in increasing performance, by applying known code transformations in the new context of CGRAs, to refactor code.
% 
These transformations include loop-unrolling to increase instruction-level parallelism (which has been explored for CGRAs~\cite{karunaratne2017hycube}), automatic loop parallelization and vectorization, and code-motion to flatten dependence chains and reduce loop initiation intervals.
% 
\riptide provides evidence that these transformations can be effective; \riptide with hand-coded loop-unrolling on {\tt dmm} is $1.86\times$ faster and is 29\% more efficient than without.
% 
% To reduce the lengths of critical paths and/or loop initiation intervals, the compiler could flatten dependence chains and perform code-motion to reduce inner-loop work.

The CGRA context does change the way these transformations are applied v. traditional compilers targetting von Neumann machines, as there is a strict limit to the number of operations that can be mapped onto the fabric.
% 
There will be a need for an accurate cost model with a feedback loop between the middle-end optimizer of a compiler and its backend-mapper (that solves for the mapping to CGRA hardware) to inform which of the transformations are undertaken.
% 
% The compiler could also unroll loops to expose more instruction-level parallelism or even auto-parallelize loops by duplicating loop bodies and splitting iteration spaces.
% 
% These transformations trade more resources for additional performance so future compilers will need accurate cost models to make optimal decisions.

\subsection{Compilation}
Improving compilation for CGRAs is another important topic for future work.
% 
While \riptide made significant progress on this, it has a few limitations, including no support for methods and single-entry-multi-exit loops, that could be addressed in future iterations of its compiler.
% 
There are also some more fundamental questions regarding how programs are represented, how to accelerate the mapping of a program to hardware, and the correct programming interface.

\subsubsection{Choosing the right IR}
CGRA compilation, like VLIW compilation, is dependent on the available hardware resources (i.e. composition of the CGRA fabric).
% 
% composition of the CGRA fabric that is being targeted.
% 
This means that if there are several devices with different CGRA fabrics, a program needs to be recompiled for each.
% 
Further low-level hardware details need to be exposed to the compiler so that it can correctly optimize and map a program.
% 
This complicates not only compiler development, but also increases compilation complexity, reduces portability, and makes virtualization challenging since the programmer must compile for a specific fabric.

Instead, compilation could be split into two phases, one phase that is device-agnostic and another that optimizes for a specific device.
% 
This is similar to the approach taken for Nvidia GPUs; first CUDA programs are compiled to PTX~\cite{ptx} (device-agnostic byte-code) and then the PTX is optimized on the host machine for the target device.
% 
For CGRAs, compilation would be split between dataflow compilation and mapping.
% 
% Dataflow compilation is device agnostic, while mapping optimizes and maps a program to a specific device.
Dataflow compilation converts a program to an intermediate representation (IR) that can be targeted to many different CGRA fabrics.
% 
Then, mapping maps and optimizes the IR to a specific fabric.
% 
The key is developing an IR for a program that captures the right amount of information.
% 
Too simple, and the mapper may not have enough information to perform low-level, device-specific optimizations.
% 
Too complex, and the mapper will be complicated, making mapping slow and development costly.
%
And at the same time, the IR should maintain backwards compatibility, while being extensible as hardware matures and gains new capabilities.
% 
% Thus careful design of the IR is fundamental for future compilation pipelines.

\subsubsection{Scalability}
Mapping a program to a specific CGRA fabric is challenging, since it reduces to finding a colored subgraph isomorphism between a program's computation graph and CGRA hardware resources.
% 
For relatively small programs (small number of operations), \riptide's ILP- and SAT- based mappers do well to find valid mappings, but as program size increases mapping times increase superlinearly.
% 
For example, while DMV (12 operations) takes $\approx10s$ to map onto a $6\times6$ fabric, DFS (50 operations) takes just under three minutes ($\approx 18\times$ more time for $4.1\times$ more operations) with SAT and more than an hour using ILP.
% 
New encoding methods and ways of solving these constraint-based problem are required.
% 
% One possibility is to reformulate the problem in terms of boolean satisifiability and use SAT-solvers to find a valid mapping.
% 
% SAT may be a better approach since all variables are already binary in the ILP formulation.
% 
% Early results support this conclusion --- DFS takes just $\approx1.5m$ v. $\approx75m$ with ILP.

Heuristic-based approaches have also showed promise for CGRAs~\cite{himap,lee2021ultra,4dcgra,amp2020,chordmap,pathseeker} and are widely used for place \& route for ASIC design and FPGAs~\cite{chen2006fpga}.
% 
However, they may require an overprovisioning of resources --- routes in particular --- and may be suboptimal v. constraint-based formulations.
% Luckily there is much prior work on place \& route for FPGAs and ASICs that extremely relevant.
% % 
% Prior work on place and route for FPGAs and for ASICs is extremely relevant.
%
% that develop custom heuristic-based algorithms to quickly find valid mappings.
% Custom search heuristics or changes to the encoding could reduce this time even more.
% 
This might be a necessary tradeoff, though, as a boost in compilation performace can enable future compilers to use mapping in a feedback loop to choose which optimization (e.g., loop unrolling, loop vectorization, etc.) passes to apply to maximize/minimize different application objectives.
% 
Further, depending on the optimization, the suboptimality of heuristic-based mapping may be moot.

\subsubsection{Changing the programming interface}
The programming interface also affects compilation by dictating the amount and type of information supplied to the compiler.
% 
Program annotations or domain specific languages can better expose program structure, enabling the compiler to reduce memory dependences (by simplifying alias analysis), increase parallelization, or pipeline loops to name a few optimizations. 
% 
Different programming interfaces may also simplify the programmer's job by providing higher-level primitives.

However, rather than redesigning the entire system for a specific interface or DSL, future interfaces should build on top of general-purpose program support like what \riptide provides.
% 
The future DSL interfaces can use the general-purpose support as a crutch when programs only partially fit the the model of the DSL.
% 
This increases the applicability of DSLs and prevents one-off designs that require significant hardware changes when application requirements change.

