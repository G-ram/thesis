\chapter{Conclusion}
\label{chapter:conclusion}
This thesis has presented a new ULP sensor system stack that will enable future applications of ``beyond-the-edge'' intelligence.
%
The overarching goal has been to reduce energy at each level of the stack without sacrificing programmability.
%
From software to silicon, we have contributed the following systems to meeting this goal:
% discussed have reduced energy and in many cases improved performance, while maintaining programmability.
%

\begin{itemize}

\item \sonic is a machine inference software runtime system for intermittently-operating, energy-harvesting devices.
% 
It leverages the regular structure of inference to reduce the costs of guaranteeing correct execution under frequent power failures.
% 
\sonic was the first to demonstrate of inference on intermittent, energy-harvesting devices and showed the importance of accurate, local inference.
% 
It also exposed the inefficiencies of existing commercial devices and stressed the need for new architectures.

\item \manic was our response to the need for new architectures.
% 
\manic developed vector-dataflow execution that amortized the cost of instruction fetch (vector execution) and minimized data supply energy (VRF accesses) by forwarding intermediates directly from producers to consumers (dataflow execution).
% 
\msilicon showed the benefits of the vector-dataflow execution in a real testchip prototype, but also exposed the limitations of the implementation, wasting energy reconfiguring shared pipeline resources from cycle-to-cycle.

\item \snafu generates ULP CGRAs that eliminate this cost of reconfiguring shared pipeline resources.
% 
\snafu implements spatial-vector-dataflow execution where each PE is assigned a single operation for the duration of a kernel's execution.
% 
This along with \snafu's bufferless NoC and implementation of ordered dataflow minimize energy.
% 
At the same time, \snafu maximizes flexibility by taking a ``bring-your-own-functional'' unit approach that allows designers to easily integrate custom operations.
% 
\snafu is competitive with ASIC designs while maintaining a high-degree of programmability.
% 
Through iterative and selective specialization, \snafu can further close the gap to ASIC designs.

\item \riptide observes that the more computation offloaded to an ULP CGRA fabric the more efficient the overall system.
% 
It develops a dataflow compiler and ULP CGRA architecture that directly targets programs written in C to reduce programmer effort to offload computation.
% 
It introduces a general-purpose control-flow model that supports arbitrary control-flow and irregular memory accesses, leveraging existing resources in the on-chip network to implement CF operations.
% 
\riptide achieves the best of both worlds --- it narrows the gap to ASIC designs even more than \snafu, while improving on programmability.
\end{itemize}

These contributions form the basis of a new energy-minimal, ULP sensor system stack.
% 
They demonstrate that extreme energy-efficiency can be achieved without compromising on programmability.
%
This is the power of rethinking the entire stack, optimizing at multiple levels at once, reduces energy, increases performance, and ultimately enables new applications.