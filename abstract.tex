\begin{abstract}
\addchaptertocentry{\abstractname} % Add the abstract to the table of contents
Ultra-low-power (ULP) sensor devices are increasingly being deployed for a variety of use-cases that require sophisticated processing of sensed data.
% 
Regardless of the deployment, energy-efficiency is critical; for battery-powered devices, energy-efficiency determines device lifetime, while for energy-harvesting devices, energy-efficiency determines performance by dictating the frequency of recharging.
% 
Unfortunately, existing devices pay a severe energy tax for their programmability, wasting energy in instruction-fetch/decode, pipeline-control and data supply. 
% 
Further, offloading computation from an edge-device to the cloud is not practical as communication costs an order of magnitude more energy than local compute.
% 
The solution is to redesign the ULP sensor system stack to increase the energy-efficiency of on-board compute and enable sophisticated processing of sensed data.
% 
This thesis proposes such a stack --- from software to silicon --- that leverages new execution models to reduce the tax of programmability and achieve extreme energy-efficiency.
% 
Specifically it contributes 1) \sonic, a software framework that enables machine inference on intermittently-operating, energy-harvesting devices, 2) \manic, a vector-dataflow co-processor (and corresponding silicon prototype), 3) \snafu, an ULP coarse-grain-reconfigurable-array (CGRA) generation framework and architecture, and 4) \riptide, a co-designed dataflow compiler and energy-minimal CGRA. 
% 
\sonic was the first demonstration of machine inference on an intermittent device, but also exposed the flaws of existing ULP devices.
% 
\manic fixed these problems by combining vector execution to amortize instruction fetch with dataflow execution to minimize data supply energy by forwarding intermediate values directly from producers to consumers.
% 
\snafu extended \manic's vector-dataflow to further reduce energy by minimzing the toggling of shared pipeline resources.
% 
It implemented spatial-vector-dataflow execution that spatially lays out computation across a fabric of PEs and assigns at most one operation to each PE.
% 
Finally, \riptide improves overall system efficiency by compiling and offloading to its CGRA, programs written in C with complex control-flow and irregular memory accesses. 
% 
Together these contributions form the basis of a new ULP sensor system stack that is $>2$ orders of magnitude more efficient than existing devices, enabling new emerging applications that require intelligence ``beyond-the-edge.''
\end{abstract}