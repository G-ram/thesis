\chapter{Constraint-based scheduling} 

\section{\snafu's ILP for mapping}
The following describes the integer linear program for \snafu's mapper.
% 
It is based on prior work~\cite{nowatzki2013general}.

\begin{table}[htb]
	\centering
	\resizebox{0.75\linewidth}{!}{
		\begin{tabular}{ll}
			\toprule
			\bf Variables  					& \bf Explanation \\
			\midrule
			\multicolumn{2}{l}{\bf Inputs: Program} \\
			$V$ 							& Set of program vertices \\
			$E$ 							& \thead[l]{Set of program edges \\
			Each edge is a tuple of the form, $(v_{src},v_{dst})$, \\
			(source vertex, destination vertex)} \\
			\midrule
			\multicolumn{2}{l}{\bf Inputs: CGRA} \\
			$N$ 							& Set of CGRA nodes (PEs) \\
			$R$ 							& Set of CGRA routers \\
			$L$ 							& Set of CGRA NoC links \\
			$H_{nl}[N,L]$ 					& 
				\thead[l]{Node-to-link matrix \\ 
				$H_{nl}(n, l) = 1$ if link $l$ originates from node $n$} \\
			$H_{rl}[R,L]$ 					& 
				\thead[l]{Router-to-link matrix \\ 
				$H_{rl}(n, l) = 1$ if link $l$ originates from router $r$} \\
			$H_{ln}[L,N]$ 					& 
				\thead[l]{Router-to-node matrix \\ 
				$H_{ln}[l, n] = 1$ if link $l$ is an input to node $n$} \\
			$H_{lr}[L,R]$ 					& 
				\thead[l]{Link-to-router matrix \\ 
				$H_{lr}[l, r] = 1$ if link $l$ is an input to router $r$} \\
			\midrule
			\multicolumn{2}{l}{\bf Inputs: Compatibility} \\
			$C_{vn}[V,N]$ 					& \thead[l]{Vertex-node compatibility matrix \\
			$C_{vn}[v,n]=1$ if vertex $v$ can be mapped to node $n$} \\
			\toprule
			\multicolumn{2}{l}{\bf Outputs} \\
			$M_{vn}[V,N]$ & Mapping of program vertices to CGRA nodes\\
			$M_{el}[E,L]$ & Mapping of program edges to CGRA links\\
			\bottomrule
		\end{tabular}
	}
	\caption{Variables for \snafu's ILP.}
	\label{tab:snafu:appendix:ilp:vars}
\end{table}

\paragraph{Helper functions}
\begin{equation*}
src(e) = v_{src}
\end{equation*}
\begin{equation*}
dst(e) = v_{dst}
\end{equation*}

\paragraph{Integer linear program}

\begin{equation*}
\text{minimize }\sum_e \sum_l M_{el}[e,l] \text{ subject to}
\end{equation*}

\begin{enumerate}
\item $M_{vn} \le C_{vn}$

\textbf{Explanation:} Vertex is mapped to a compatible node.
\item $\forall v, \, \sum\limits_n M_{vn}[v,n] = 1$

\textbf{Explanation:} Every vertex is mapped to one node.
\item $\forall n, \, \sum\limits_v M_{vn}[v,n] \le 1$

\textbf{Explanation:} No node has multiple vertices mapped to it.
\item $M_{el} H_{lr} = M_{el} H_{rl}^T$

\textbf{Explanation:} If edge is mapped to incoming link of a router, then it also has to be mapped to an outgoing link of the router (preservation of flow).
\item $\forall e, \, M_{el}[e]H_{ln} = M_{vn}[dst(e)]$ 

\textbf{Explanation:} If a vertex is mapped to a node, then the inputs to the vertex must be mapped to incoming links of that node.
\item $\forall e, \, M_{el}[e]H_{nl}^T = M_{vn}[src(e)]$

\textbf{Explanation:} If a vertex is mapped to a node, then the outputs of the vertex must be mapped to outgoing links of that node.
\item $\forall e_1, e_2 | src(e_1) \ne src(e_2), \, M_{el}[e_1] + M_{el}[e_2] \le 1$

\textbf{Explanation:} If edges do not originate from the same node, they can not share links.
\end{enumerate}

Please note that we adopt numpy-like indexing; for example, $M_{el}[e]$ returns a $1\times |L|$ vector, representing the link assignments for edge $e$.
% 
Also, $M_{el}[:,l]$ extracts a $|E|\times 1$ column vector and represents the assignment of link, $l$, across all edges.

\section{\riptide's ILP for mapping}

\begin{table}[h]
	\centering
	\resizebox{0.75\linewidth}{!}{
		\begin{tabular}{ll}
			\toprule
			\bf Variables  					& \bf Explanation \\
			\midrule
			\multicolumn{2}{l}{\bf Inputs: Program} \\
			$V$ 							& Set of program vertices \\
			$E$ 							& \thead[l]{Set of program edges \\
			Each edge is a tuple of the form, $(v_{src},p_{src}, v_{dst}, p_{dst})$, \\
			(source vertex, source port, destination vertex, destination port)} \\
			\midrule
			\multicolumn{2}{l}{\bf Inputs: CGRA} \\
			$N$ 							& Set of CGRA nodes (PEs) \\
			$F$ 							& Set of control-flow nodes in NoC, $F \subset N$ \\
			$R$ 							& Set of CGRA routers \\
			$L$ 							& Set of CGRA NoC links \\
			$H_{nl}[N,L]$ 					& 
				\thead[l]{Node-to-link matrix \\ 
				$H_{nl}(n, l) = 1$ if link $l$ originates from node $n$} \\
			$H_{rl}[R,L]$ 					& 
				\thead[l]{Router-to-link matrix \\ 
				$H_{rl}(n, l) = 1$ if link $l$ originates from router $r$} \\
			$H_{ln}[L,N]$ 					& 
				\thead[l]{Router-to-node matrix \\ 
				$H_{ln}[l, n] = 1$ if link $l$ is an input to node $n$} \\
			$H_{lr}[L,R]$ 					& 
				\thead[l]{Link-to-router matrix \\ 
				$H_{lr}[l, r] = 1$ if link $l$ is an input to router $r$} \\
			\midrule
			\multicolumn{2}{l}{\bf Inputs: Compatibility} \\
			$C_{vn}[V,N]$ 					& \thead[l]{Vertex-node compatibility matrix \\
			$C_{vn}[v,n]=1$ if vertex $v$ can be mapped to node $n$} \\
			$C_{el}[E,L]$ 					& \thead[l]{Edge-link compatibility matrix \\
			$C_{el}[E,L]=1$ if edge $e$ can be mapped to link $l$} \\
			\toprule
			\multicolumn{2}{l}{\bf Outputs} \\
			$M_{vn}[V,N]$ & Mapping of program vertices to CGRA nodes\\
			$M_{el}[E,L]$ & Mapping of program edges to CGRA links\\
			\bottomrule
		\end{tabular}
	}
	\caption{Variables for \riptide's ILP.}
	\label{tab:riptide:appendix:ilp:vars}
\end{table}

\paragraph{Helper functions}
\begin{equation*}
src(e) = v_{src}
\end{equation*}
\begin{equation*}
dst(e) = v_{dst}
\end{equation*}
\begin{equation*}
oport(e) = p_{src}
\end{equation*}

\paragraph{Integer linear program}

\begin{equation*}
\text{minimize }\sum_e \sum_l M_{el}[e,l] \text{ subject to}
\end{equation*}

\begin{enumerate}
\item $M_{vn} \le C_{vn}$

\textbf{Explanation:} Vertex is mapped to a compatible node.
\item $M_{el} \le C_{el}$

\textbf{Explanation:} Edge is mapped to a compatible link (ports of producer \& consumer match).
\item $\forall n, \, \sum\limits_v M_{vn}[v,n] \le 1$

\textbf{Explanation:} No node has multiple vertices mapped to it.
\item $M_{el} H_{lr} = M_{el} H_{rl}^T$

\textbf{Explanation:} If edge is mapped to incoming link of a router, then it also has to be mapped to an outgoing link of the router (preservation of flow).
\item $\forall e, n |n \notin F, \, M_{el}[e]H_{ln}[:,n] = M_{vn}[dst(e)]$ 

\textbf{Explanation:} If a vertex is mapped to a node (not control-flow node in NoC), then the inputs to the vertex must be mapped to incoming links of that node.
\item $\forall e,n|n \notin F, \, M_{el}[e]H_{nl}^T[:,n] = M_{vn}[src(e)]$

\textbf{Explanation:} If a vertex is mapped to a node (not control-flow node in NoC), then the outputs of the vertex must be mapped to outgoing links of that node.
\item $\forall e, n |n \in F, \, M_{el}[e]H_{ln}[:,n] \ge M_{vn}[dst(e)]$ 

\textbf{Explanation:} If a vertex is mapped to a control-flow node in NoC, then the inputs to the vertex must be mapped to incoming links of that node.
\item $\forall e,n|n \in F, \, M_{el}[e]H_{nl}^T[:,n] \ge M_{vn}[src(e)]$

\textbf{Explanation:} If a vertex is mapped to a control-flow node in NoC, then the outputs of the vertex must be mapped to outgoing links of that node.
\item $\forall e_1, e_2 | src(e_1) \ne src(e_2)  \bigcup oport(e_1) \ne oport(e_2), \, M_{el}[e_1] + M_{el}[e_2] \le 1$

\textbf{Explanation:} If edges do not originate from the same node or do not have the same ouput port, they can not share links.
\item $\forall n |n \in F, \, M_{el}H_{ln}[:,n] + \sum\limits_v M_{vn}[v,n] \ge M_{el}H_{nl}^T[:,n]$ 

\textbf{Explanation:} If a vertex is not mapped to a control-flow node in NoC, then a value can be passed through, so if edge is mapped to incoming link then it is also mapped to outgoing link.
\item $\forall n |n \in F, \, M_{el}H_{ln}[:,n] \le \sum\limits_v M_{vn}[v,n] + M_{el}H_{nl}^T[:,n]$ 

\textbf{Explanation:} If a vertex is not mapped to a control-flow node in NoC, then a value can be passed through, so if edge is mapped to incoming link then it is also mapped to outgoing link.
\end{enumerate}
