The emergence of tiny, pervasively deployed, ultra-low-power sensor systems
enables important new applications in environmental sensing, in- and on-body
medical implants, civil infrastructure monitors, and even tiny chip-scale
satellites.
%
Existing systems for these applications suffer fundamental
inefficiencies that demand new, extremely energy-efficient computer
architectures. 

\paragraph{Sensing workloads are increasingly sophisticated}
Sensor devices collect data from a deployed environment and must process raw
data to support applications.  Processing varies and may entail digital signal
processing (DSP), computing statistics, sorting, or sophisticated computations
such as machine learning (ML) inference using a deep neural network (DNN) or a
support vector machine (SVM).
%
As processing sophistication has increased, sensor device capability also
matured to include high-definition image sensors~\cite{naderiparizi2018towards} and
multi-sensor arrays~\cite{laput2017synthetic}, increasing sensed data volume. 

This shift poses a challenge: how can we perform sophisticated computations
on simple, ultra-low-power systems?
One design is 
to offload work by wirelessly transmitting data to a more powerful
nearby computer (e.g., at the ``edge'' or cloud) for processing.  
%
In offloading, the more data a sensor produces, the more data the device must
communicate. 
%
Unfortunately, transmitting data takes much more energy per byte than
sensing, storing, or computing on those data~\cite{sonic,zebranet}.  
%
While a high-powered device like a smartphone, with a high-bandwidth,
long-range radio, can afford to offload data to the edge or cloud,
this is not practical for power-, energy-, and bandwidth-limited sensor devices~\cite{dongare2017openchirp,sonic}.

Since offloading is infeasible,
the alternative is to process data \emph{locally} on the sensor node itself.
%
For example, recent work~\cite{sonic} has shown how systems can use commodity off-the-shelf microcontrollers (COTS MCU)
to filter sensed data so that only meaningful data (as defined by the application) are transmitted.
%
Processing data locally at a sensor node eliminates most of the high energy
cost of communication,
but makes the device highly sensitive to the energy-efficiency of computation.
%

There are two key criteria that make a computation-heavy sensor system effective.
First, the device must process data locally at a low operating power and with \emph{extremely high energy-efficiency}.
Second, the device must be \emph{programmable} and general to support a wide variety of applications.
% 
These goals are in tension, since programmability often carries a significant energy penalty.
Our goal is to design a highly programmable architecture that \emph{hides microarchitectural complexity while eliminating the energy costs of programmability}.

\paragraph{Existing low-power architectures fall short}
%% Unfortunately, existing low-power architectures fall short of this ideal.
Ultra-low-power COTS MCUs used in many deeply embedded sensor nodes
(e.g., TI MSP430, ARM M0+ \& M4+) fail
to meet the criteria for an effective sensor node.
%
These MCUs are general-purpose, programmable devices that support a variety of
applications.  
%
However, COTS MCUs pay a high power, energy, and performance cost for their
generality and programmability (see the \textit{COTS MCU} dot in~\autoref{fig:intro}).  
%

Programmability is expensive in two main ways~\cite{horowitz:isscc14:energy-keynote,hameed2010understanding,balfour_elm_thesis}.
First, \emph{instruction supply} consumes significant energy: in the best case, the energy of
an instruction cache hit, and in the worst case, the energy of a main memory
read and instruction cache fill.  
%
Lacking sophisticated microarchitectural features such as superscalar and
out-of-order execution pipelines~\cite{msp430fr5994,traber2016pulpino}, the energy overhead of
instruction supply constitutes a significant fraction of total operating energy.
%
Second, data supply through \emph{register file (RF) access} also consumes significant energy.
%
Together, we find that instruction and data supply consume
$54.4\%$ of the average execution energy in our workloads.

\paragraph{Programming pitfalls of architectural specialization}
%Specialization
To combat the energy costs of generality, some recent work has turned to
microarchitectural specialization, making a system energy-efficient at the
expense of generality and
programmability~\cite{chen:isca16:eyeriss,chen:asplos14:diannao,du:isca15:shidiannao,liu:isca15:pudiannao,chen2014dadiannao,venkatesh2010conservation}.
Specialization customizes a system's control and datapath to accommodate a
particular workload (e.g., deep neural networks~\cite{chen:isca16:eyeriss,chen:asplos14:diannao}),
eliminating inessential inefficiencies like instruction supply and RF access.
The downside of specialization is its 
inability to support a wide range of applications (see the \textit{ASIC} dot in~\autoref{fig:intro}).

In contrast to specialization, another approach to programmable
energy-efficiency is to target a conventional vector architecture (such as
NVidia's Jetson TX2~\cite{jetsontx2}, ARM NEON~\cite{neon}, or TI LEA~\cite{lea}), amortizing the cost of instruction
supply across a large number of compute operations. Unfortunately,
vector architectures exacerbate the energy costs of RF access,
especially in high-throughput designs with multi-ported vector register files (VRFs)~\cite{kozyrakis2003overcoming,asanovic1996t0,iram},
and so remain far from the energy-efficiency of fully specialized designs~\cite{hameed2010understanding} (see the \textit{classic vector} dot in~\autoref{fig:intro}).

The ELM architecture stands out among prior efforts as an architecture that
targets ultra-low-power operation, operates with extremely high
energy-efficiency, and retains general-purpose
programmability~\cite{balfour_elm_thesis,balfour2008energy}.
%
The key to ELM's efficiency is an \emph{operand forwarding} network that avoids
latching intermediate results and a distributed RF that provides
sufficient register storage, while avoiding unfavorable RF energy scaling. 
%
Unfortunately, despite these successes, ELM faces fundamental limitations that prevent its
widespread adoption.
%
ELM makes significant changes to the architecture and microarchitecture of the
system, requiring a full re-write of software to target its exotic,
software-managed RF hierarchy and instruction-register design.  This programming task
requires expert-level assembly hand-coding,
as compilers for ELM are unlikely to be simple or efficient;
e.g., ELM itself cites a nearly 2$\times$ drop in performance when moving from hand-coded assembly to compiler-generated assembly~\cite{balfour2008energy}.
%
While ELM supports general-purpose programs, it does so with a high
programmability cost and substantial changes to software development tools (as shown in~\autoref{fig:intro}).

\paragraph{Our design and contributions}
In this work we present \manic: an efficient vector-dataflow
architecture for ultra-low-power embedded systems.  
%
As depicted in~\autoref{fig:intro}, \manic is closest to the \textit{Ideal} design, achieving high energy-efficiency while remaining general-purpose and simple to program.
%
%
\manic is simple to program because it exposes a standard vector ISA interface
based on the RISC-V vector extension~\cite{riscv_2019}.

\manic achieves high energy-efficiency by eliminating the two main costs of
programmability through its vector-dataflow design.
%
First, \textbf{vector} execution amortizes
instruction supply energy over a large number of operations.
%% like prior vector designs.
%
Second, \manic addresses the high cost of VRF accesses
through its \textbf{dataflow} component by forwarding operands 
directly between vector operations.
%
\manic transparently buffers vector outputs in a small {forwarding buffer}
and, at instruction issue, renames vector operands to directly access
the forwarding buffer, \emph{eliminating read accesses to the VRF}.
%
Additionally, \manic extends the vector ISA with \textbf{kill annotations} that denote
the last use of a vector register,
\emph{eliminating write accesses to the VRF}.
%
The vector-dataflow architecture is efficient because \manic amortizes the energy of tracking dataflow across many vector operations.
%
\manic thus eliminates a large fraction of VRF accesses (90.1\% on average in our experiments)
with simple microarchitectural changes that leave the basic vector architecture intact.

Finally, we have designed and implemented a code scheduling algorithm
that exploits \manic's operand forwarding to minimize VRF energy, while being \emph{microarchitecturally agnostic}.
In other words, it is \emph{not} necessary to expose the details
of the pipeline architecture or size of forwarding buffers
to minimize VRF energy---%
a single code schedule is near-optimal across a range of microarchitectural design points.

We implement \manic fully in RTL and use industry-grade CAD tools to
evaluate its energy efficiency across a collection of programs appropriate to the deeply embedded domain.
%
Using post-synthesis energy estimates, we show that \manic
is within $26.4\%$ of the energy of an idealized design while remaining fully general and making few, unobtrusive changes to the ISA and software development stack.
%
\figMANICDesignSpace
