\newcommand{\figMANICDesignSpace} {
	\begin{figure}[t]
		\centering
		\includegraphics[width=0.45\linewidth]{manic/figures/pdf/overview-crop.pdf}
		\hfill
		\caption{\manic seeks to improve energy efficiency without compromising programmability.}
		\label{fig:manic:intro}
	\end{figure}	
}

\newcommand{\figMANICOverview}{
	\begin{figure}[htb]
    \begin{subfigure}{0.14\linewidth}
    		\centering
        \includegraphics[height=1.7in]{manic/figures/pdf/scalar_overview-crop.pdf}
        \caption{Scalar}
        \label{fig:manic:overview:scalar}
    \end{subfigure}
    \hfill
    \begin{subfigure}{0.42\linewidth}
        \centering
        \includegraphics[height=1.7in]{manic/figures/pdf/vector_overview-crop.pdf}
        \caption{Vector}
        \label{fig:manic:overview:vector}
    \end{subfigure}
    \hfill
		\begin{subfigure}{0.42\linewidth}
			\centering
			\includegraphics[height=1.7in]{manic/figures/pdf/manic_overview-crop.pdf}
			\caption{Vector-dataflow}
			\label{fig:manic:overview:vde}
		\end{subfigure}
		\caption{Different execution models. Orange arrows represent control flow, blue arrows represent dataflow. \manic relies on vector-dataflow execution,
		avoiding register accesses by forwarding and renaming.}
		\label{fig:manic:overview}
	\end{figure}
}

\newcommand{\figMANICBlock}{
	\begin{figure}[t]
		\centering
		\includegraphics[width=0.9\linewidth]{manic/figures/pdf/block.pdf}
		\caption{A block diagram of \manic's microarchitectural components. \manic operates in two phases, Decode \& Rename and execute. Decode \& rename fills in the instruction buffer, identifying opportunities for dataflow forwarding. Execute cycles through the instruction buffer to compute across instructions (vertical) and then across vectors (horizontal).}
		\label{fig:manic:block}
	\end{figure}
}


\newcommand{\figMANICMANICa}{
\begin{figure}[htb]
  \centering
  \includegraphics[width=\columnwidth]{manic/figures/pdf/transform.pdf}
  \caption{\manic's Decode \& Rename logic constructs windows of instructions with dataflow. The rename table keeps track of registers and names, updating the instruction buffer when new opportunities for forwarding are identified.}
  \label{fig:manic:vde:issue}
\end{figure}
}

\newcommand{\figMANICMANICb}{
\begin{figure}[htb]
	\centering
	\includegraphics[width=.65\columnwidth]{manic/figures/pdf/vde.pdf}
	\caption{\manic's microarchitecture components execute a window of instructions using forwarding according to dataflow across an entire vector of input.}
	\label{fig:manic:vde:execute}
\end{figure}
}

\newcommand{\figMANICMANIC}{
	\begin{figure*}[htb]
		\centering
		\begin{subfigure}{0.20\linewidth}
			\includegraphics[width=\linewidth]{manic/figures/pdf/transform.pdf}
			\caption{}
			\label{fig:manic:vde:issue}
		\end{subfigure}
		\hfill
		\begin{subfigure}{0.69\linewidth}
			\includegraphics[width=\linewidth]{manic/figures/pdf/vde.pdf}
			\caption{}
			\label{fig:manic:vde:execute}
		\end{subfigure}
		\caption{An example illustrating \manic's vector-dataflow execution model.
		\protect\autoref{fig:manic:vde:issue} shows \manic's issue logic, which constructs
		windows of instructions that exhibit dataflow. \protect\autoref{fig:manic:vde:execute} show
		how \manic executes a window of instructions, forwarding according to dataflow, 
		across an entire vector of inputs}
		\label{fig:manic:vde}
	\end{figure*}
}

\newcommand{\figMANICMiss}{
	\begin{figure}[h]
			\begin{subfigure}{0.49\linewidth}
				\includegraphics[width=\linewidth]{manic/figures/pdf/dmm_miss-crop.pdf}
				\caption{Dense matrix multiply}
			\end{subfigure}
			\hfill
			\begin{subfigure}{0.49\linewidth}
				\includegraphics[width=\linewidth]{manic/figures/pdf/smm_miss-crop.pdf}
				\caption{Sparse matrix multiply}
			\end{subfigure}
			\begin{subfigure}{0.49\linewidth}
				\includegraphics[width=\linewidth]{manic/figures/pdf/sconv_miss-crop.pdf}
				\caption{Sparse convolution}
			\end{subfigure}
			\begin{subfigure}{0.49\linewidth}
				\includegraphics[width=\linewidth]{manic/figures/pdf/fft_miss-crop.pdf}
				\caption{Fast-fourier transform}
			\end{subfigure}
			\caption{Miss curves for four different applications.}
			\label{fig:manic:miss}
	\end{figure}
}

\newcommand{\figMANICKillDistro}{
	\begin{figure}[t]
	  \begin{subfigure}{0.32\linewidth}
	    \centering
	    \includegraphics[height=1.05in]{manic/figures/pdf/smm_kill_distro-crop.pdf}
	    \caption{SpMV}
	  \end{subfigure}
    \hfill
	  \begin{subfigure}{0.32\linewidth}
	    \centering
	    \includegraphics[height=1.05in]{manic/figures/pdf/sconv_kill_distro-crop.pdf}
	    \caption{Sparse conv}
	  \end{subfigure}
    \hfill
	  \begin{subfigure}{0.32\linewidth}
	  	\centering
	    \includegraphics[height=1.05in,trim=0 0 48pt 0,clip]{manic/figures/pdf/fft_kill_distro-crop.pdf}
	    \caption{FFT}
	  \end{subfigure}

	  \caption{Histograms of kill distances for three different applications.
	    Distances skew left, suggesting values are consumed for the last time
	    shortly after being produced.}
	  \label{fig:manic:kill_distro}
	\end{figure}
}

\newcommand{\tabMANICMethodology}{
%% \begin{table}
\footnotesize
\centering
\resizebox{\linewidth}{!}{
\begin{tabular}{lll}
\toprule
\multicolumn{1}{c}{\textbf{Parameter}} & \textbf{Possible Values} & \textbf{Optimal} \\
\midrule
Core Frequency                         & 100 MHz                  & 100 MHz               \\
Scalar Register \#                     & 16                       & 16                    \\
Vector Register \#                     & 16                       & 16                    \\
Vector Length                          & 16/32/64                 & 64                    \\
Window Size                            & 8/16/32                  & 16                     \\
Main Memory Size                       & 64 KB                    & 64KB                 \\
Cache Line Size                        & 4/8 B                    & 4B                    \\
DCache Size                            & 128/256/512 B       	  & 256B                  \\
DCache Associativity                   & 1/2/4/8                  & 8                     \\
ICache Size                            & 64/128/256 B             & 128B                   \\
ICache Associativity                   & 1/2/4/8                  & 2                     \\
\bottomrule
\end{tabular}
}
%% \vspace{-1em}
\captionof{table}{Microarchitectural parameters.}
\label{tab:manic:params}
%% \vspace{-2em}
%% \end{table}
}

\newcommand{\figMANICKill}{
	\begin{figure*}[t]
		\centering
		\includegraphics[width=0.7\textwidth]{manic/figures/pdf/kill_legend-crop.pdf}
		\vspace{0.5em}
            
		\begin{subfigure}{0.32\linewidth}
			\centering
			\includegraphics[height=0.85in]{manic/figures/pdf/smm_kill-crop.pdf}
			\caption{Sparse matrix multiply}
		\end{subfigure}
		\hfill
		\begin{subfigure}{0.32\linewidth}
			\centering
			\includegraphics[height=0.85in]{manic/figures/pdf/sconv_kill-crop.pdf}
			\caption{Sparse convolution}
		\end{subfigure}
    \hfill
		\begin{subfigure}{0.32\linewidth}
			\centering
			\includegraphics[height=0.85in]{manic/figures/pdf/fft_kill-crop.pdf}
      \caption{Fast-fourier transform}
		\end{subfigure}
		\caption{Code scheduling is microarchitecturally agnostic -- minimizing the sum of kill 
		  distances is good proxy for minimizing register writes for specific window size.}
		\label{fig:manic:kill}
  \end{figure*}
}

\newcommand{\figMANICCompare}{
	\begin{figure*}[htb]
		\centering
		\includegraphics[height=1.25em]{manic/figures/pdf/bar_legend-crop.pdf}

		\hfill
		\begin{subfigure}{0.18\linewidth}
			\includegraphics[width=\linewidth]{manic/figures/pdf/fft_bar-crop.pdf}
			\caption{Fast-fourier transform}
			\label{fig:manic:fft}
		\end{subfigure}
		\hfill
		\begin{subfigure}{0.18\linewidth}
			\includegraphics[width=\linewidth]{manic/figures/pdf/dmm_bar-crop.pdf}
			\caption{Dense matrix multiply}
			\label{fig:manic:dmm}
		\end{subfigure}
		\hfill
		\begin{subfigure}{0.18\linewidth}
			\includegraphics[width=\linewidth]{manic/figures/pdf/smm_bar-crop.pdf}
			\caption{Sparse matrix multiply}
			\label{fig:manic:smm}
		\end{subfigure}
		\hfill
		\begin{subfigure}{0.18\linewidth}
			\includegraphics[width=\linewidth]{manic/figures/pdf/dconv_bar-crop.pdf}
			\caption{Dense convolution}
			\label{fig:manic:dconv}
		\end{subfigure}
		\hfill
		\begin{subfigure}{0.18\linewidth}
			\includegraphics[width=\linewidth]{manic/figures/pdf/sconv_bar-crop.pdf}
			\caption{Sparse convolution}
			\label{fig:manic:sconv}
		\end{subfigure}
		\hfill
		\vspace{1em}
		\begin{minipage}{0.49\linewidth}
			\centering
			\begin{subfigure}{0.49\linewidth}
				\centering
				\includegraphics[width=0.72\linewidth]{manic/figures/pdf/dmv_bar-crop.pdf}
				\caption{Dense matrix-vector multiply}
				\label{fig:manic:dmv}
			\end{subfigure}
			\hfill
			\begin{subfigure}{0.49\linewidth}
                          \centering
				\includegraphics[width=0.72\linewidth]{manic/figures/pdf/smv_bar-crop.pdf}
				\caption{Sparse matrix-vector multiply}
				\label{fig:manic:smv}
			\end{subfigure}
            \hfill
		\end{minipage}
		\begin{minipage}{0.5\linewidth}
		\caption{Energy of \manic against various baselines across seven different
			applications. Bars (from left-to-right): plain-C running baseline scalar core, vector plain-C running on vector baseline, \manic with
			forwarding disabled, \manic without dataflow code scheduling (\autoref{sec:design:schedule}), \manic, and ideal \manic with no instruction or data supply energy.
			\manic is within $26.4\%$ of the ideal \manic design and is overall
			$2.8\times$ more energy efficient than the scalar baseline and $38.1\%$ more energy efficient than the vector baseline.
			\vspace{3em}
			}
		\label{fig:manic:compare}
		\end{minipage}
	\end{figure*}
}

\newcommand{\figMANICAltCompare}{
	\begin{figure*}[htb]
		\centering
		\includegraphics[height=1.25em]{manic/figures/pdf/bar_legend-crop.pdf}
		\vspace{0.5em}

		\begin{minipage}{0.49\linewidth}
			\begin{subfigure}{\linewidth}
				\centering
				\includegraphics[width=\linewidth]{manic/figures/pdf/overall_bar-crop.pdf}
			\end{subfigure}
			\caption{Full system energy of \manic against various baselines across seven different applications.
			Bars (from left-to-right): scalar baseline, vector baseline, \manic, and an idealized vector design with no instruction or data supply energy.
			\manic is within $26.4\%$ of the ideal design and is overall $2.8\times$ more energy efficient than the scalar baseline and $38.1\%$ more energy efficient than the vector baseline.
			}
			\label{fig:manic:overall}
		\end{minipage}
		\hfill
		\begin{minipage}{0.49\linewidth}
			\begin{subfigure}{\linewidth}
				\centering
				\includegraphics[width=\linewidth]{manic/figures/pdf/manic_bar-crop.pdf}
			\end{subfigure}
			\caption{
			  Impact of \manic's optimizations on full system energy,
                          comparing (from left-to-right): vector baseline, \manic with forwarding disabled, \manic without dataflow code scheduling, and full \manic.
                          Without forwarding, \manic's added components slightly increase energy by <5\%.
                          Forwarding saves $15.5\%$ system energy vs.\ the baseline,
                          and kill annotations and dataflow code scheduling saves a further $26.0\%$.}
			\label{fig:manic:compare}
		\end{minipage}
	\end{figure*}
}

\newcommand{\figMANICPower}{
	\begin{figure}[b]
		\centering
		\vspace{1em}
		\includegraphics[height=1.25em]{manic/figures/pdf/bar_legend-crop.pdf}
		\vspace{0.5em}

		\includegraphics[width=0.95\linewidth]{manic/figures/pdf/power_bar-crop.pdf}
		\caption{Power of the scalar baseline, vector baseline, and \manic across seven benchmarks. \manic uses $10.0\%$ less power than the scalar baseline and, despite using less energy than the scalar baseline, the vector baseline actually uses $29.5\%$ more power.}
		\label{fig:manic:power}
	\end{figure}
}

\newcommand{\figMANICSONICCompare}{
	\begin{figure}[h]
		\centering
		\includegraphics[height=1.25em]{manic/figures/pdf/bar_legend-crop.pdf}
		\vspace{0.5em}

		\includegraphics[width=0.95\linewidth]{manic/figures/pdf/sonic_bar-crop.pdf}
		\caption{In the intermittent computing domain, \manic with hardware JIT-checkpointing is $9.6\times$ more energy efficient than SONIC~\cite{sonic}, which maintains correctness in software alone.}
		\label{fig:manic:sonic_compare}
	\end{figure}
}

\newcommand{\figMANICPerformance}{
	\begin{figure}[h]
		\begin{subfigure}{\linewidth}
			\includegraphics[width=\linewidth]{manic/figures/pdf/cycles_bar-crop.pdf}
			\caption{Cycle count}
			\label{fig:manic:cycle_count}
		\end{subfigure}
		\begin{subfigure}{\linewidth}
			\includegraphics[width=\linewidth]{manic/figures/pdf/insn_count_bar-crop.pdf}
			\caption{Instruction count}
			\label{fig:manic:insn_count}
		\end{subfigure}
		\caption{Instruction and cycle counts for seven benchmarks running on the scalar baseline, vector baseline, and \manic. The vector baseline and \manic effectively do not differ. Vector execution means that both run $10.6\times$ less instructions and $2.5\times$ less cycles than the scalar baseline. }
		\label{fig:manic:performance}
		%% \vspace{-2.5em}
	\end{figure}	
}

\newcommand{\figMANICSens}{
	\begin{figure*}[ht]
		\centering
		\includegraphics[height=1.25em]{manic/figures/pdf/bar_legend-crop.pdf}
		\vspace{0.5em}

		\begin{subfigure}{0.49\linewidth}
			\includegraphics[width=\linewidth]{manic/figures/pdf/insn_buf_length_bar-crop.pdf}
			\caption{Window/Instruction buffer size}
			\label{fig:manic:sens_ws}
		\end{subfigure}
		\hfill
		\begin{subfigure}{0.49\linewidth}
			\includegraphics[width=\linewidth]{manic/figures/pdf/vl_bar-crop.pdf}
			\caption{Vector length}
			\label{fig:manic:sens_vl}
		\end{subfigure}
		
		\vspace{0.5em}
		\begin{subfigure}{0.49\linewidth}
			\includegraphics[width=\linewidth]{manic/figures/pdf/icache_bar-crop.pdf}
			\caption{Instruction cache size}
			\label{fig:manic:sens_l1i}
		\end{subfigure}
		\hfill
		\begin{subfigure}{0.49\linewidth}
			\includegraphics[width=\linewidth]{manic/figures/pdf/dcache_bar-crop.pdf}
			\caption{Data cache size}
			\label{fig:manic:sens_l1d}
		\end{subfigure}
		\caption{\manic's sensitivity to its microarchitectural parameters: 16 is the best window
		size, larger vector lengths are generally better, and moderately sized caches are generally more energy efficient.}
		\label{fig:manic:sens}
		\vspace{-1em}
	\end{figure*}
}

\newcommand{\figMANICChipBlock}{
\begin{figure}[htb]
	\centering
	\includegraphics[width=0.6\linewidth]{manic/figures/pdf/chip_block.pdf}
	\caption{Block diagrams of test-chip and a single design. 
	The chip includes scalar, vector, and \manic designs. Each design is isolated (dedicated power rails) and can run standalone.}	
	\label{fig:manic:chip:block}
\end{figure}
}

\newcommand{\figMANICChip}{
\begin{figure}[htb]
	\centering
	\begin{subfigure}{0.49\linewidth}
		\centering
		\includegraphics[height=1.8in]{manic/figures/pdf/chip_package.pdf}
		\caption{Package.}
		\label{fig:manic:chip:package}
	\end{subfigure}
	\begin{subfigure}{0.49\linewidth}
		\centering
		\includegraphics[height=1.8in]{manic/figures/pdf/chip_delid.pdf}
		\caption{Electron micrograph.}
		\label{fig:manic:chip:delid}
	\end{subfigure}
	\caption{\autoref{fig:manic:chip:package} shows the \manic test-chip. \autoref{fig:manic:chip:delid} shows an electron micrograph of delidded chip and highlights components of \manic design.}
	\label{fig:manic:chip}
\end{figure}
}

\newcommand{\tabMANICEval}{
\begin{table}[t]
\renewcommand{\arraystretch}{1.1}
\centering
\resizebox{\linewidth}{!}{
\begin{tabular}{c|ccccc}
&    \textbf{2017~\cite{keller2017risc}}    &    \textbf{2018~\cite{karnik2018cm}}    &    \textbf{2019~\cite{bol201919}}    &    \textbf{2020~\cite{prabhat202027}}    &    \textbf{This work}    \\
\hline
\textbf{Architecture}    &    \thead{Scalar \& \\ Vector}    &    Scalar    &    Scalar    &    \thead{Scalar w/ \\ SIMD ext.}    &    \thead{Scalar \& \\ Vector-dataflow}    \\
\hline
\textbf{ISA}    &    RISC-V    &    Thumb-2    &    Thumb-2    &    Thumb-2    &    RISC-V    \\
\hline
\textbf{Process (nm)}    &    28 FD-SOI    &    14 Tri-gate    &    28 FD-SOI    &    65 LP    &    22 bulk FF    \\
\hline
\textbf{Core Area ($mm^2$)}    &    1.07    &    6.25    &    0.675    &    6    &    0.57    \\
\hline
\textbf{Voltage (V)}    &    0.48-1.0    &    0.4-1.0    &    0.4    &    0.4-0.75    &    \thead{0.4-1.05 Core \\ 1.1 MRAM}    \\
\hline
\textbf{Frequency (MHz)}    &    20-797    &    0.2-950    &    40-80    &    0.8-38    &    4-48.9    \\
\hline
\textbf{Memory (KB)}    &    56 SRAM    &    64+64+384 SRAM    &    32+32 SRAM    &    \thead{128 ROM, \\ 16+4 SRAM}    &    \thead{64 SRAM, \\ 256 MRAM}    \\
\hline
\textbf{Power Budget (mW)}    &    1-200    &    1-20    &    1    &    1-4    &    \thead{0.019-2 w/o MRAM \\ 1-2 w/ MRAM}    \\
\hline
\textbf{Average Power (\textmu W)}    &    50000    &    80    &    144    &    47    &    \thead{19.1 w/o MRAM \\ @ 4MHz\footnotemark[1] \\ 1.7mW w/ MRAM @ 49MHz\footnotemark[1]} \\
\hline
\textbf{\thead{Peak Efficiency \\ (MOPS/mW)\footnotemark[2]}}   &    \thead{41.8 \\ MFlops/mW}    &    Not reported    &    97    &    Not reported    &    \thead{256 w/o MRAM \\ 33.2 w/ MRAM}    \\
\hline
\textbf{\thead{Best Active \\ Energy (pJ/Cycle)}}|    &    Not reported    &    6.2    &    3    &    10.9    &    \thead{3.7 w/o MRAM \\ 29 w/ MRAM} \\                  
\multicolumn{6}{l}{\footnotesize \footnotemark[1] Over all benchmarks
\footnotemark[2] 32b operations
}
\end{tabular}
}
\caption{\manic vs.\ prior work. \manic is $2.6\times$ more efficient than prior work, achieving 256 MOPS/mW (@19$\mu$W \& 4MHz).}
\label{tab:manic:eval}
%% \vspace{-1.5em}
\end{table}
}

\newcommand{\figMANICMRAMCharacter}{
\begin{table}[htb]
	\centering
	\resizebox{0.3\linewidth}{!}{
		\begin{tabular}{cc}
			\toprule
			Size (KB) & 256 \\
			Area (mm$^2$) & 0.31 \\
			Voltage (V) & 1.1 \\
			Leakage (\textmu W) & 663 \\
			\thead{32b Read Latency @ \\ 50 MHz (ns)} & 170 \\
			\thead{32b Write Latency @ \\  50 MHz (\textmu s)} & 8.4 \\
			32b Read Energy (pJ) & 437 \\
			32b Write Energy (nJ) & 29.7 \\
			Read Energy (pJ/bit) & 13.7 \\
			Write Energy (pJ/bit) & 929 \\
			\bottomrule
		\end{tabular}
	}
	\caption{MRAM characterization.}
	\label{tab:manic:mram}
\end{table}
}

\newcommand{\figMANICMRAMCase}{
\begin{figure}[htb]
	\begin{minipage}{0.55\columnwidth}
		\footnotesize
		\begin{subfigure}{\linewidth}
			\centering
			\resizebox{0.5\linewidth}{!}{
				\begin{tabular}{cc}
					\toprule
					Size (KB) & 256 \\
					Area (mm$^2$) & 0.31 \\
					Voltage (V) & 1.1 \\
					Leakage (\textmu W) & 663 \\
					\thead{32b Read Latency @ \\ 50 MHz (ns)} & 170 \\
					\thead{32b Write Latency @ \\  50 MHz (\textmu s)} & 8.4 \\
					32b Read Energy (pJ) & 437 \\
					32b Write Energy (nJ) & 29.7 \\
					Read Energy (pJ/bit) & 13.7 \\
					Write Energy (pJ/bit) & 929 \\
					\bottomrule
				\end{tabular}
			}
			\caption{MRAM characterization.}
			\label{fig:manic:case:tab:eval}
		\end{subfigure}
	\end{minipage}
	\hfill
	\begin{minipage}{0.375\columnwidth}
		\begin{subfigure}{\linewidth}
			\centering
			\includegraphics[width=0.9\linewidth]{manic/figures/pdf/mram_case-graph-crop.pdf}
			\caption{MOPS/mW for DMM.}
			\label{fig:manic:case:mram}
		\end{subfigure}
	\end{minipage}
	\hfill
	\vspace{0.5em}
	\begin{minipage}{\columnwidth}
		\centering
		\begin{subfigure}{\linewidth}
			\resizebox{0.85\linewidth}{!}{
				\begin{tabular}{ll}
					%% \toprule
					1: & Running from MRAM, DCache enabled, 48.9 MHz, 0.64V Core \\
					2: & Running from MRAM, DCache disabled, 231 MHz, 1.0V Core \\
					3: & Running from SRAM, MRAM enabled, DCache enabled, 48.9 MHz, 0.64V Core \\
					4: & Running from SRAM, MRAM enabled, DCache disabled, 166 MHz, 1.0V Core \\
					%% \bottomrule
				\end{tabular}
			}
			\label{fig:manic:case:tab:config}
		\end{subfigure}
	\end{minipage}

	\caption{MRAM characterization \& case study on DMM.}
	\label{fig:manic:case}

\end{figure}
}

\newcommand{\figMANICEnergyCache}{
\begin{figure}[htb]
	\centering
	\begin{subfigure}{0.7\linewidth}
		\centering
		\includegraphics[height=0.15in]{manic/figures/pdf/v0.4nocache-legend-crop.pdf}

		\includegraphics[width=\linewidth]{manic/figures/pdf/v0.4cache-graph-crop.pdf}
		\caption{Energy (normalized to scalar).}
		\label{fig:manic:cache:eff}
	\end{subfigure}
	\hfill
	\begin{subfigure}{0.7\linewidth}
		\centering
		\vspace{1em}
		\includegraphics[width=\linewidth]{manic/figures/pdf/eff_v0.4cache-graph-crop.pdf}
		\caption{Energy efficiency.}
		\label{fig:manic:cache:energy}
	\end{subfigure}
	\caption{Energy (normalized to scalar) and energy-efficiency of scalar, vector, and \manic designs across ten benchmarks.}
	\label{fig:manic:cache}
\end{figure}
}

\newcommand{\figMANICEnergyNoCache}{
\begin{figure}[htb]
	\centering
	\begin{subfigure}{0.7\linewidth}
		\centering
		\includegraphics[height=0.15in]{manic/figures/pdf/v0.4nocache-legend-crop.pdf}

		\includegraphics[width=\linewidth]{manic/figures/pdf/v0.4nocache-graph-crop.pdf}
		\caption{Energy (normalized to scalar).}
		\label{fig:manic:nocache:eff}
	\end{subfigure}
	\hfill
	\begin{subfigure}{0.7\linewidth}
		\centering
		\vspace{1em}
		\includegraphics[width=\linewidth]{manic/figures/pdf/eff_v0.4nocache-graph-crop.pdf}
		\caption{Energy efficiency.}
		\label{fig:manic:nocache:energy}
	\end{subfigure}
	\label{fig:manic:nocache}
\end{figure}
}
