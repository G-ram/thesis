Emerging sensing applications create an unprecedented need for
energy efficiency in programmable processors.
%
To achieve useful multi-year deployments on a small battery or energy
harvester, these applications must avoid off-device communication
and instead process most data locally.
%
Recent work has proven coarse-grained
reconfigurable arrays (CGRAs) as a promising architecture for this
domain.
%
Unfortunately, nearly all prior CGRAs support only computations with
simple control flow and no memory aliasing (e.g., affine inner loops),
causing an Amdahl efficiency bottleneck as non-trivial fractions of
programs must run on an inefficient von Neumann core.

\riptide is a co-designed compiler and CGRA architecture that
achieves both high programmability and extreme energy efficiency,
eliminating this bottleneck.
%
\riptide provides a rich set of control-flow operators
that support arbitrary control flow and memory access
on the CGRA fabric.
%
\riptide implements these primitives without tagged tokens
to save energy; this requires careful ordering analysis
in the compiler to guarantee correctness.
%
\riptide further saves energy and area by offloading most control
operations into its programmable on-chip network,
where they can re-use existing network switches.
%
\riptide's compiler is implemented in LLVM, and its hardware
is synthesized in Intel 22FFL. % an industrial sub-28nm process.
%
\riptide compiles applications written in C while saving
25\% energy v.\ the state-of-the-art ULP CGRA
and 6.6$\times$ energy v.\ a von Neumann core.
