Recent advances in machine learning,
sensor devices, and
embedded systems open the door
to a wide range of sensing applications,
such as
civil-infrastructure or wilderness monitoring,
public safety and security,
medical devices,
and chip-scale satellites~\cite{kicksat2}.
%
To achieve long (e.g., 5+ year) deployment lifetimes,
these applications rely on on-device processing
to limit off-device communication.
%
Computing at the extreme edge
calls for ultra-low-power ($<$1\,mW),
\emph{extremely energy-efficient},
and \emph{general-purpose} processing~\cite{sonic}.

\paragraph{Why general-purpose?} 
The need for extreme energy efficiency suggests a role for
application-specific integrated circuits (ASICs), but ASICs
come with several major disadvantages.
%% that make general-purpose hardware a better option.
%
Computations in smart sensing applications are diverse, spanning deep
learning, signal processing, compression, encoding, decryption,
planning, control, and symbolic reasoning~\cite{Gobieski2018IntermittentDN}.
%
Only a general-purpose solution can support all of these, as it is
infeasible to build an ASIC for every conceivable task~\cite{edge-offload,moonwalk}.
%
Moreover, the rapid pace of change in these applications (e.g., due to
new machine learning algorithms~\cite{jouppi2021ten})
puts specialized hardware at risk of premature obsolescence, especially in
a multi-year deployment~\cite{edge-offload}.
%
Finally, by targeting all computations, general-purpose designs
can achieve much greater scale than specialized designs --- perhaps trillions of devices~\cite{arm-trillions}.
Scale reduces device cost,
makes advanced manufacturing nodes economically viable,
and mitigates carbon footprint~\cite{gupta2022chasing}.

Unfortunately, traditional programmable cores are very inefficient,
typically using only 5\% to 10\% of their energy on useful work~\cite{manic,snafu,horowitz:isscc14:energy-keynote}.
%
Architects' challenge is thus to reconcile generality and efficiency.

\paragraph{CGRAs are both programmable and efficient!}
%
Recent work has shown that coarse-grained reconfigurable
arrays (CGRAs) can achieve energy efficiency competitive with ASICs
while remaining programmable by software~\cite{snafu,nowatzki2017domain,weng2020dsagen}.
%
As shown in \autoref{fig:riptide:intro},
a CGRA~\cite{remarc,adres,matrix,dyser,revamp,opencgra,cgrame,wave,nguyen2021fifer,morphosys,mozart,ppa,fpca,plasticine,dadu2019towards,parashar2013triggered,capstan,nowatzki:isca17:stream-dataflow,goldstein2000piperench,trips,weng2020dsagen,weng2020hybrid,voitsechov2014single,mishra2006tartan,tan2018stitch,karunaratne2017hycube,voitsechov2018inter,evx} is an array of processing elements
(PEs) connected by an on-chip network (NoC).
%
CGRAs are programmed by mapping a computation's control and dataflow
onto the array, i.e., by assigning operations to PEs and configuring
the NoC to route values between dependent operations.
%
A CGRA's efficiency derives from avoiding overheads intrinsic to von
Neumann architectures, specifically instruction fetch/control and data
buffering in a centralized register file.

In the context of ultra-low-power sensing applications,
\snafu~\cite{snafu} is a CGRA framework designed from
the ground up to minimize energy, in contrast to prior,
performance-focused CGRAs (\autoref{sec:background}).
%
\snafu CGRAs reduce energy by 5$\times$ vs.\ ultra-low-power von Neumann
cores, and they come within 3$\times$ of ASIC energy efficiency.

\figRipTideIntro

\paragraph{What's the problem?}
%
Amdahl's Law tells us that, to achieve significant end-to-end benefits,
CGRAs must benefit the vast majority of program execution.
%
This means CGRAs must provide a complete compiler and hardware stack
that goes from application code to an efficient CGRA configuration.
%
Unfortunately, prior CGRAs struggle to support common programming
idioms efficiently, leaving significant energy savings on the table.

On the hardware side,
%
many prior CGRAs only support simple, regular control flow,
such as inner loops with streaming memory accesses
and no data-dependent control~\cite{plasticine,nowatzki:isca17:stream-dataflow,snafu}.
%
To support complex control flow, other CGRAs employ expensive hardware mechanisms, e.g.,
associative tags to distinguish loop iterations,
large buffers to avoid deadlock,
and dynamic NoC routing~\cite{monsoon,ttda,swanson2003wavescalar,voitsechov2014single}.
%
In either case, the end result is wasted energy:
%
from the extra instructions needed to implement control flow unsupported
by the CGRA fabric,
%
or from inefficiency in the CGRA microarchitecture itself.

On the compiler side, mapping large computations onto a CGRA fabric is
perennial challenge.
%
Heuristic compilation methods often fail to find a valid
mapping~\cite{chlorophyll,nowatzki2018hybrid}, and optimization-based methods lead to
prohibitively long compilation times~\cite{nowatzki2018hybrid,cgrame-ilp}.
%
Indeed, one reason that many CGRAs support only regular control is to limit
the number of operations that must be mapped during compilation.
%
Control flow can significantly increase the size of a computation's
dataflow graph, increasing compilation time or preventing a
computation from mapping successfully.
%
To avoid these issues, some CGRAs (including \snafu) require
hand-coded assembly, raising a large barrier to adoption~\cite{snafu,yang2021spzip,nowatzki:isca17:stream-dataflow}.

\paragraph{\riptide's solution.}
%
\riptide is a co-designed CGRA compiler and microarchitecture that
supports arbitrary control flow and memory access patterns
without expensive hardware mechanisms.
%
\riptide targets emerging, highly energy-constrained applications
and is designed from the ground up to minimize energy.
%
\riptide is easy to program: its compiler supports arbitrary nested
control and loops as well as aliasing memory accesses.
%
To save energy, \riptide adopts a \emph{steering} control
paradigm~\cite{dennis1975preliminary,swanson2003wavescalar,budiu2005dataflow}, in which values are only routed to
where they are actually needed (unlike predication- and
selection-based control common in prior
CGRAs~\cite{trips,snafu}).
%
To support arbitrary nested control without tags, \riptide introduces new
control-flow primitives, such as the \emph{carry gate}, which selects
between tokens from inner and outer loops.
%
To minimize operation count and ease compilation,
\riptide introduces new operations for common programming idioms,
such as its \emph{stream generator} that generates an affine
sequence for, e.g., simple loops or streaming memory accesses.


\riptide implements the above features efficiently in the both
the compiler and hardware (\autoref{fig:riptide:intro}).
%
\riptide compiles programs from a high-level language (currently, C) and
employs novel analyses to safely parallelize operations.
%
We observe that, with steering control flow and no program counter,
conventional transitive reduction analysis fails to enforce all memory
orderings, and we introduce \emph{path-sensitive transitive reduction}
to infer orderings correctly.
%
\riptide implements arbitrary control flow without associative tags
by enforcing strict ordering among values,
%% so that operands are always matched and no tag is necessary.
leveraging its new primitives like the carry gate for nested loops.
%% leveraging the carry gate for nested loops.
%
\riptide supports common idioms like affine generators directly in hardware,
%
and its compiler recognizes these idioms in program code
and maps them onto a single PE.
%
Finally, \riptide implements its new control flow primitives without wasting energy
or PEs by \emph{offloading control flow to the on-chip network}.
%
The insight is that a NoC switch already contains essentially all of
the logic needed for steering control flow, and with a few
trivial additions it can implement a wide range of control primitives.
%
Mapping control-flow into the NoC frees PEs for arithmetic and memory
operations, so that \riptide can support deeply nested loops with complex
control flow on a small CGRA.
%

\paragraph{Contributions.}
This paper contributes the following:
\begin{compactitem}

\item \bigemph{Instruction set architecture:} We co-design \riptide's
  compiler and CGRA microarchitecture to provide a rich operation set
  that supports arbitrary control flow and irregular memory accesses
  with minimum execution energy.
  %
  We identify common programming idioms and introduce new primitives
  to support them in fewer operations.
  
\item \bigemph{Compiler:} \riptide compiles programs from high-level C code
  to an efficient CGRA configuration. \riptide identifies and enforces all
  control-flow and memory orderings, introducing \emph{path-sensitive
  transitive reduction} to safely prune unnecessary memory orderings.
  
\item \bigemph{Hardware:} \riptide implements its operation set
  efficiently in hardware. It incorporates numerous techniques to
  minimize energy, including steering control flow and tagless
  dataflow firing. \riptide \emph{offloads control flow to the
  on-chip network}, freeing PEs for other useful work.
  
\item \bigemph{Broader implications on architecture:} We perform
  an in-depth case study of dense matrix-matrix multiplication,
  comparing \riptide to an ASIC implemented in the same design flow. \riptide
  is competitive on energy and performance, but consumes significantly
  more area than the ASIC.
  %
  ASICs thus offer a cost advantage over CGRAs, but this
  advantage disappears in SoC designs with a large number of ASIC
  blocks. Given the large advantages gained by software
  programmability, we argue that energy-minimal CGRAs like \riptide have a
  compelling edge over ASICs for the majority of computations.
\end{compactitem}

\figRipTideIntroResults
\paragraph{Summary of results.}
%
We implement a complete \riptide system in RTL and synthesize it in an
industrial sub-28nm FinFET process with compiled memories.
%
Across ten benchmarks, \riptide reduces energy by
25\% vs.\ \snafu, the state-of-the-art programmable design,
and improves performance by 17\% (\autoref{fig:riptide:intro:results}).
%
At nominal voltage, \riptide achieves 141MOPs/mW (including main memory) at 50MHz
and takes $\approx$0.5mm$^2$.
%
Compared to equivalent ASICs for {\tt dmm}, {\tt sort}, and {\tt fft},
\riptide consumes just 2.1$\times$ more energy.
%
\riptide achieves these benefits on software written in C, cf.\ hand-coded
vector assembly in \snafu.

\paragraph{Road map.}
%
\autoref{sec:background} covers background, and
\autoref{sec:overview} gives an overview of \riptide.
%
Secs.~\ref{sec:cf}, \ref{sec:compiler}, and \ref{sec:arch} present
\riptide's architecture, compiler, and microarchitecture, respectively.
%
Secs.~\ref{sec:method} and \ref{sec:eval} evaluate \riptide.
\autoref{sec:implications} concludes by discussing \riptide's broader implications.
