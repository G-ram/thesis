\section{RipTide Overview}
\label{riptide:overview}

\figRipTideSystem

\riptide is a compiler and microarchitecture for ultra-low-power, energy-minimal CGRAs
(\autoref{fig:riptide:system}).
%
Its architecture is carefully co-designed across compiler and hardware
to maximize flexibility while minimizing energy.
% 
\riptide can compile and run programs written in high-level C with arbitrary control flow and memory access.
% 
\riptide improves energy efficiency / performance
by 25\% / 17\% vs.\ the
state-of-the-art~\cite{snafu} and by 6.6$\times$ / 6.2$\times$
vs.\ a ULP core.

\paragraph{\riptide's control-flow paradigm minimizes energy}
\riptide introduces a new control-flow paradigm, discussed in \autoref{riptide:cf}, that supports arbitrary programs without expensive tag-token matching hardware.
% 
Specifically, \riptide adopts a steering ($\phi^{-1}$) control-flow model, which minimizes energy by gating disabled paths so that values are only sent where they will be actually used.
% 
\riptide introduces the \emph{carry gate}, a new control-flow operator that supports nested loops without tags,
as well as other operations to maintain memory and cross-iteration orderings efficiently.
% 

\paragraph{\riptidecomp compiles arbitrary C-code to energy-minimal and performant CGRA configurations}
%
\riptide's compiler (\autoref{riptide:compiler}) leverages LLVM~\cite{llvm} and several
custom compiler passes to convert arbitrary programs to dataflow
graphs schedulable onto \riptide's CGRA hardware.
%
\riptidecomp supports steering control and introduces a novel
memory-ordering analysis to enforce load-store ordering with low
overhead.
% 
Additionally, \riptidecomp applies several optimizations to dataflow
graphs, including operation fusion (e.g., ``streamifying'' affine loops by fusing loop headers)
to reduce operation count, improve performance, and ease mapping operations onto the fabric.

\paragraph{\riptide's CGRA microarchitecture minimizes switching activity}
%
\riptide's CGRA microarchitecture prioritizes energy efficiency above all other metrics.
%
It adopts and improves techniques from prior work~\cite{snafu} to
reduce switching activity to minimize fabric energy:
%
e.g., mapping exactly one operation per PE,
firing operations without tags,
and statically routing values in a bufferless NoC.

\paragraph{Offloading control flow to the NoC enables complex programs on small CGRA fabrics with negligible hardware overhead}
\riptidearch supports control-flow operations directly in the NoC by reusing existing NoC hardware.
% 
Control-flow operations are simple, but numerous.
% 
This means that allocating them to entire PEs is unnecessary and wasteful
--- programs are often unmappable if control-flow operations require PEs.
% 
Instead, \riptide implements these operations directly in NoC routers by reusing existing switch crossbars,
which already perform most of the required logic.
