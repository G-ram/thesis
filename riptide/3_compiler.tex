\newcommand{\pes}{PEs\xspace}
\newcommand{\fend}{frontend\xspace}
\newcommand{\mend}{middle-end\xspace}
\newcommand{\bend}{backend\xspace}

\section{RipTide Compiler}
\label{riptide:compiler}

\figRipTideCompiler
\figRipTideLSO

\riptidecomp compiles, optimizes, and maps high-level C code to \riptide's CGRA fabric. 
%
Its compiler has a \fend, \mend, and \bend.
%
The \fend compiles C to target-independent intermediate representation
(IR) based on LLVM's IR.
%
The \mend optimizes the code in LLVM's IR, then translates to
target-specific IR, represented as a dataflow graph (DFG) of the operator
types from \autoref{riptide:cf}, including control flow and memory ordering.  
%
The \bend takes target-specific IR as input and maps operators onto specific
CGRA hardware units, producing a configuration bitstream directly usable for
configuring the CGRA.
%
\autoref{fig:riptide:compiler} demonstrates the \fend and \mend's compiler passes. 
%

\subsection{Memory-ordering analysis}

\riptidecomp maps sequential code onto a CGRA fabric in which many operations,
including memory operations, may execute in parallel.  
%
For correctness, some dependent memory operations must execute in a particular
order.
%
\riptidecomp's \mend computes required orderings between memory operations present
in the IR and adds control-flow operators to enforce those orderings. 
%
The compiler uses alias analysis to identify dependent memory operations that
may access the same memory locations and may execute on the \riptide fabric
simultaneously, requiring ordering.  

\riptidecomp computes ordering relations by analyzing memory dependences.  
%
A \textit{memory dependence} is an ordered relation from one memory operation
(source) to another (destination) that access same memory location,
where the destination is reachable from the source and one (or both)
are writes.
%
An \textit{ordering graph} (OG) is a digraph of the ordering relations
required to ensure that parallel memory operations produce a
result consistent with a sequential execution of the program.
%

\paragraph{Constructing an ordering graph} 
To build the OG, \riptidecomp uses alias analysis to identify memory operations
that do not, may, or must access the same memory location (i.e., alias).
%
\riptidecomp makes no assumptions on the alias analysis.
%
\riptide queries alias analysis for all memory operations and adds
an edge to the OG for each pair that may or must alias.
%
\riptidecomp need not consider self-dependences because repeated instances of the
same memory operation are always ordered on its CGRA fabric. %mapped into \riptide's CGRA fabric are always ordered.
%
\autoref{fig:riptide:lso} shows a basic, unoptimized OG in the top left for an 
example function (\texttt{example2}).

\paragraph{Pruning the ordering graph} The OG as computed includes redundant
ordering edges because data dependences and control dependences suffice to
order some operations.
%
A data dependence orders a pair of operations in \riptide 
because the destination will not execute before receiving an input token 
produced by the source.
% 
\riptidecomp finds data dependences by walking LLVM's definition-use (def-use)
chain from source to destination and removes arcs for dependent operations~\cite{nachos}.
%
For instance, in \texttt{example2}'s CFG from \autoref{fig:riptide:compiler}, 
\textbf{S2} is data-dependent \textbf{L1}, so there need not
be an ordering arc in the OG. This is reflected in the blue-outlined edge
from \textbf{L1} to \textbf{S2} that is pruned in the OG in \autoref{fig:riptide:lso}.
%
Similarly, control dependences order some memory operations if the execution of
the destination is control-dependent on the source.
%
\riptidecomp analyzes the CFG to identify control dependences
between memory operations and removes those orderings from the OG.
%
In \texttt{example2}'s CFG from \autoref{fig:riptide:compiler}, the
edge from \textbf{L0} to \textbf{S1} in \autoref{fig:riptide:lso} is pruned using this analysis.

\paragraph{Transitive memory-ordering analysis}
% 
Two dependent memory operations are transitively ordered if there is a path (of
ordering arcs) in the OG from source to destination.
%
\riptidecomp finds and eliminates edges in the OG between operations already
transitively ordered by some other path in the OG corresponding to control- and
data-dependence orderings.
%

To simplify its OG, \riptidecomp uses transitive reduction (TR)~\cite{aho-tr}, which prior work
deployed to simplify ordering relation graphs for parallel
execution of loops~\cite{midkiff-padua,compiler-sync}.
%
We apply TR to the OG, which converts a (potentially cyclic) ordering 
graph into an acyclic graph of strongly connected components (the SCCDAG). 
%
Traditional TR then eliminates transitive edges between SCCs, removes 
all edges in an SCC, and adds edges forming a simple cycle through all 
vertices in the SCC.
%
We modify the algorithm in two ways to make it work for \riptide's OG.
First, we require the inserted cycle to add edges compatible with program order
instead of being arbitrary.  Second, the inserted cycle must add edges that
respect proper loop nesting, avoiding edges directly from the inner to outer
loop.  To handle these edges, we add synthetic loop entry and exit nodes to
each loop (shown as \texttt{src} and \texttt{sink} nodes at the bottom 
of \autoref{fig:riptide:lso}). Any edge inserted % in the condensed SCC 
that links an inner loop node to an outer loop node instead uses the inner 
loop's exit as its destination.
Symmetrically, an edge inserted that links an outer loop node to an inner loop
node has the inner loop's entry as its destination.   With these two changes,
the SCCDAG is usable for TR.
%

However, we observe that applying existing TR analysis to the OG in \riptide fails to
preserve required ordering operations.
%
The problem is that a source and destination may be ordered along one
(transitive) path, and ordering along another (direct) path may be removed as
redundant.  The transitive path's execution enforces ordering, but the direct
path's execution does not, which is incorrect.
%
This problem arises due to \riptide's steering control and lack of a program
counter to order memory operations.
%
In \autoref{fig:riptide:lso}, we show a scenario where path-sensitivity is critical.
%
The transitive edge in the SCCDAG of the OG between \textbf{SCC3} 
(\textbf{S2}) and \textbf{SCC1} (\textbf{L0}) cannot be eliminated 
in TR because the alternative path through \textbf{SCC2} 
in the OG does not inherently capture the control-flow path from \textbf{S2} to
\textbf{L0} via the backedge of the loop.

To correctly apply TR to remove redundant ordering edges, \riptidecomp uses
\emph{path-sensitive} TR. 
%
This analysis confirms that a transitive ordering path between a source and
destination subsumes all possible control-flow paths between source and
destination before removing any ordering arc from the OG.
%% because of that transitive path. 
%
With this constraint applied,
% removed ordering arcs do not correspond to CFG
%paths that must enforce ordering, and
\riptide can safely use this modified TR on its OG.
%
Prior efforts also incorporated control-flow to reduce ordering constraints~\cite{midkiff-padua,doacross} 
or improve alias analysis accuracy to begin with~\cite{hind-aa}.


\paragraph{Enforcing ordering constraints} Memory operations in \riptide 
produce a control token on completion and can optionally consume a
control token to enforce memory ordering.
%
The \mend encodes ordering arcs as defs and uses of data values in the IR
(as seen in the IR transform of loads and stores in \autoref{fig:riptide:compiler})
before lowering them as dependences in the DFG.
%
For a memory operation that must receive multiple control signals,
the \mend inserts merge operators (\autoref{riptide:cf}) to consolidate those signals.

\subsection{Control-flow operator insertion}

The compiler lowers its IR to use \riptide's control paradigm by inserting \riptide
control-flow operators into the DFG.
%

\paragraph{Steer}
%
The compiler uses the control dependence graph (CDG)~\cite{cytron} to 
insert steers.
%
For each consumer of a value, the compiler walks the CDG from the producer
to consumer
%% the node that
and inserts a steer operator at
each node along the CDG traversal. 
%
The steer's control input is the decider of the basic block that the steer depends on,
and its data input is the value or the output of an earlier inserted steer.
%

\paragraph{Carry and invariant}
%
For loops, the compiler inserts a carry operator for loop-carried dependences
and an invariant operator for loop-invariant values
into the loop header.
%
A carry's data
input comes from the loop backedge that produces the value.  An invariant's
data input comes from the loop pre-header.
%
These operators should produce a token only if the next iteration of the loop is
certain to execute; to ensure this behavior, the compiler sets their control
signal to the decider of the block at the loop exit. 
%

\paragraph{Order}
If two iterations of a loop may take different control-flow
paths that converge at a single {\em merge node}\footnote{Not to be confused with merge \emph{operators} in \riptide.} in the loop body, either may
produce a token to the merge node first.
But for correctness, the one from the
earlier iteration must produce the first token.
%
The compiler inserts an order operator at a merge node in the CFG to ensure that 
tokens flow to the merge node in iteration order.
%
The control signal {\tt D} for the order operator is the decider of
nearest common dominator of the merge node's predecessor basic blocks.
%
%% This decider determines control flow to the merge node. 
%
Since the earlier iteration sends its control signal first, it blocks the later
iteration % from sending a control signal and token to the order operator
until the earlier iteration completes.
%

\subsection{Stream fusion}

\riptidecomp performs target-specific operation fusion on the DFG to
reduce required operations and routes by combining value {\em stream generators} with loop
control logic and address computation logic.
%
\riptide supports stream operations and applies them for the common case 
of a loop with an affine loop governing induction variable (LGIV).
%
A stream makes loop logic efficient by fusing the LGIV update and the loop
exit condition into a single operation.
%
In the DFG, loop iteration logic is represented by the exit condition, an
update operation, the carry for the LGIV's value, and the steer that
gates the LGIV in a loop iteration.
%
The \mend fuses these operations into a single stream operation and sets
the stream's initial, step, and bound values.
%
\autoref{fig:riptide:compiler} shows stream compilation, where
the operations for loop iteration logic (outlined in blue in the 
DFG) are fused into a stream operator.
%
\riptidecomp uses applies induction variable analysis~\cite{dragon,zima-scev} to 
find affine LGIVs.
%
\riptidecomp also identifies address computations, maps these to an affine
stream if possible, and fuses the stream into the memory operation.
%

\subsection{Mapping DFGs to hardware}
\riptidecomp's backend takes a DFG and a CGRA topology description and generates
scalar code to invoke \riptide and a bitstream to configure the \riptide fabric.
% 
The compiler's mapper uses a mixed integer linear program constraint problem formulation (see~\autoref{appendix:riptide:ilp}), the
solution to which maps DFG nodes and edges to hardware PEs, CF-modules, and
links.
% 
The mapper finds isomorphisms between the DFG and the CGRA fabric, while
minimizing distance between operations, and maps control-flow operations onto
the NoC (\autoref{riptide:fin}).
%