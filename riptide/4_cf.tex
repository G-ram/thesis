\section{RipTide Instruction Set Architecture}
\label{riptide:cf}


% \figRipTideCFExecute

\figRipTideISA

\riptide provides a rich set of control-flow operators to support complex programs.
% 
Its ISA, shown in \autoref{tab:riptide:isa}, has six categories of operators: arithmetic, multiplier, memory, control flow, synchronization, and streams.
%
(Multiplication is split from other arithmetic because, to save area, only some PEs can perform multiplication.)
%
We now highlight the control-flow, synchronization, and stream operators.

\subsection{Control-flow operators}
%% There are three fundamental control-flow operators: steer, order and carry.
%% % 
%% These operators are universal, i.e., sufficient to implement \riptide's control-flow paradigm.
%
\riptide's operators are illustrated in \autoref{fig:riptide:cf:gates}.
%
Whenever a value is read, it is implied that the
operator waits until a valid token arrives for that value over the
NoC. Tokens are buffered at the inputs if they are not
consumed or discarded.

\figRipTideCFGates

\figRipTideCompiler

\paragraph{Steer} 
Steers ($\phi^{-1}$) come in two flavors --- True and False --- and take two inputs: a decider, {\tt D} and a data input, {\tt A}.
% 
If {\tt D} matches the flavor, then the gate passes {\tt A} through; otherwise, {\tt A} is discarded.
% 
Steers are necessary to implement conditional execution, as they gate the inputs to disabled branches.

\paragraph{Carry}
\label{cf:carry}
% Carry is the other fundamental control-flow operator.
% 
Carry represents a loop-carried dependency and takes a decider, {\tt D}, and two data values, {\tt A} and {\tt B}.
% 
Carry has the internal state machine shown in \autoref{fig:riptide:cf:gates}.
%
In the {\em Initial} state, it waits for {\tt A}, and then passes it through and transitions to the {\em Block} state.
%
While in {\em Block}, if {\tt D} is True, the operator passes through {\tt B}.
%
It transitions back to {\em Initial} when {\tt D} is False, and begins waiting for the next {\tt A} value (if not already buffered at the input).

Carry operators keep tokens ordered in loops, eliminating the need to tag tokens.
%
All backedges are routed through a carry operator in \riptide.
%
%% All loop-carried dependencies must go through a carry operator.
%% Since loop-carried dependencies must go through a carry operator, these values never get reordered.
% 
By not consuming {\tt A} while in {\em Block}, carry operators prevent
outer loops from spawning a new inner-loop instance before the
previous one has finished.
%
(Iterations from one inner-loop may be pipelined if they are independent, but entire instances of the inner loop will be serialized.)

\paragraph{Invariant}
The invariant operator is a slight variation of carry.
%
It represents a loop invariant and can be implemented as a carry with a self-edge back to {\tt B}.
%
Invariants are used to generate a new loop-invariant token for each loop iteration.

%% \paragraph{Select}
%% The select operator is a ternary operator that chooses between two inputs given a condition.
%
% The select operator is a direct implementation of LLVM's {\tt select} ternary operator, choosing between two inputs given a condition.

\subsection{Synchronization operators}
%% There are two synchronizations operators: order and merge.
%% % 
%% These enforce cross-iteration and memory ordering.

\paragraph{Merge}
The merge operator enforces cross-iteration ordering by making sure that tokens from different loop iterations
appear in the same order, regardless of the control path taken within by each loop iteration.
%% the outputs of divergent control flow in a loop iteration do not reorder with another outputs from another iteration.
% 
The operator takes three inputs: a decider, {\tt D}, and two data inputs, {\tt A} and {\tt B}.
%
Merge is essentially a mux that passes through either {\tt A} or {\tt B}, depending on {\tt D}.
%
But note that only the value passed through is consumed.

\paragraph{Order}
The order operator is used to enforce memory ordering by guaranteeing that multiple preceding operations have executed.
% 
It takes two inputs, {\tt A} and {\tt B}, and fires as soon as both arrive, passing {\tt B} through.

\subsection{Stream operators}
\label{cf:streams}
Streams generate a sequence of data values, which are 
produced by evaluating an affine function
across a range of inputs. 
%
These operators are used in loops
governed by affine induction variables.
%
A stream takes three inputs: {\tt start}, {\tt step}, and {\tt bound}.
It initially sets its internal {\tt idx} to {\tt start},
%
and then begins iterating a specified arithmetic operator {\tt f}
as \mbox{\tt idx' = f(idx, step)}.

A stream operator produces two output tokens per iteration: {\tt idx} itself,
and a control signal {\tt last}.
%
{\tt last} is False until {\tt idx} reaches {\tt bound},
whereupon it is True and the stream stops iterating.
%
{\tt last} is used by downstream control logic to, e.g., control
a carry operator for outer loops.
