\subsection{Control flow in the NoC}
\label{riptide:fin}

Control-flow operations are simple to implement (often a single
multiplexer), but there are often many of them.
%
Mapping each to a PE wastes energy and area, and can render infeasible mapping
to the CGRA.
%
We observe that much of the logic required to implement control flow is already
plentiful in the NoC.
%
Each NoC switch is a crossbar that can be {\em re-purposed} to mux values for
control.
%
Implementing each control-flow operator simply requires routing values and manipulating
upstream and downstream ready/valid signals.

\riptide's router microarchitecture is shown in \autoref{fig:arch:router}.
% 
The router shares routing configuration and its data and valid crossbars with
the baseline NoC.
%
\riptide adds a control-flow module (CFM) at output ports.
% 
The CFM determines when to send data to the router's output port and
manipulates inputs to the data switch to select which data to send.

\paragraph{Control-flow module.}
%
The CFM takes eight inputs and produces five outputs that control router
configuration and the dataflow through the network.
% 
The inputs are:
\begin{compactitem}

\item {\tt cfg}: configuration of the CFM;

\item {\tt A\_valid}, {\tt B\_valid}, {\tt D\_valid}: whether inputs are valid;

\item {\tt D}: value of the decider;

\item {\tt A\_cxn} and {\tt B\_cxn}: input ports for {\tt A} and {\tt B}; and

\item {\tt noc\_ready}: backpressure signal from the output port.
  
\end{compactitem}
%
From this, the CFM produces outputs:
\begin{compactitem}
  
\item {\tt A\_ready}, {\tt B\_ready}, and {\tt D\_ready}: upstream
  backpressure signals that allow the CFM to block upstream
  producers until all signals required are valid;

\item {\tt noc\_valid}: the valid signal for the CF's output; and

\item {\tt cxn}: which port ({\tt A\_cxn} or {\tt B\_cxn}) to route
  to the output port on the data switch.
  
\end{compactitem}

\begin{figure}[t]
\begin{subfigure}{\linewidth}
\begin{lstlisting}[style=custompython]
cxn = A_cxn
forever:
    A_ready = D_ready = 0
    if A_valid && D_valid: # wait for A and D
        # if D is true, pass through A;
        # else discard A 
        noc_valid = D
        A_ready = D_ready = noc_ready || !D
        if D: wait for noc_ready
\end{lstlisting}
\caption{Steer (True flavor).}
\label{lst:fin:steer}
\end{subfigure}

\begin{subfigure}{\linewidth}
\begin{lstlisting}[style=custompython]
forever:
    # begin in Initial state
    if A_valid:
        cxn = A_cxn         # pass through A
        noc_valid = A_valid
        D_ready = A_ready = noc_ready
        B_ready = xxx       # don't care
    wait for noc_ready
    # transition to Block state
    do until D_valid && !D:
        cxn = B_cxn         # pass through B
        noc_valid = B_valid
        D_ready = B_ready = noc_ready
        A_ready = false     # hold A at input
        wait for noc_ready
\end{lstlisting}
\caption{Carry.}
\label{lst:fin:carry}
\end{subfigure}
  \caption{Implementing control-flow operators using NoC control
    signals.}
  \label{lst:fin}
\end{figure}

\paragraph{Supported operations.}
The CFM can be configured for routing or for the control operators in
\autoref{riptide:cf}.
%
Implementing routing, e.g., {\tt out = A}, is simple: {\tt cxn = A\_cxn}, {\tt
noc\_valid = A\_valid}, and {\tt A\_ready = noc\_valid}.

Implementing other operators is slightly more involved, but requires only a
small state machine.
%
\autoref{lst:fin} is pseudocode for steer and carry operators
(\autoref{riptide:cf}).
%
A steer forwards {\tt A} if {\tt D} is true; otherwise,
it discards {\tt A}.
%
To implement steer, the CFM waits for {\tt A} and {\tt D}
to be valid.
%
If {\tt D} is true, then {\tt noc\_valid} is raised, and the {\tt
  noc\_ready} signal propagates upstream to {\tt A} and {\tt D}
and the CFM waits for {\tt noc\_ready}, i.e., for the value to be consumed.
%
If {\tt D} is false, then {\tt noc\_valid} is kept low, and
{\tt A\_ready} and {\tt D\_ready} are raised to discard these
tokens.

Carry is a more complex control-flow operator.
%
Carry begins in {\em Initial} state, waiting for a valid {\tt A}
token, which
%
It forwards the token and transitions to {\em Blocked} state, where
it forwards {\tt B} until seeing a false {\tt D} token.
%

\paragraph{Control-flow in the NoC adds small hardware overheads.}
%
Implementing control flow in the NoC is far more energy- and
area-efficient than in a PE.
%
The CFM deals only with narrow control signals and the 1b decider
value {\tt D}.
%
It does not need to touch full data signals at all; these are
left to the pre-existing data switch.
%
Importantly, this means that the CFM adds no data buffers.
%
Instead, the CFM simply raises the {\tt *\_ready} signals to park
values in the upstream output channels until they are no longer
needed.

By contrast, implementing control flow in a PE requires full
data-width muxes and, if an entire PE is dedicated to control, an
output channel to hold the results.
%
Nevertheless, \riptide is sometimes forced to allocate a PE for control
flow.
%
Specifically, if a control-flow operation takes a constant or
software-supplied value, it currently requires \textmu core
support.

\paragraph{Buffering of decider values.}
The CF module provides a small amount of buffering for decider values.
% 
This is because loop deciders often have high fanout, which means that the next iteration of a loop is likely blocked by one or more downstream consumers.
% 
To remove this limitation, \riptide provides a small amount of downstream buffering for 1b decider signals, improving performance with minimal impact on area.
%
The CFM uses run-length encoding to buffer up to eight decider values with just 3b of additional state.
