\section{RipTide Microarchitecture}
\label{sec:arch}

\riptide is an energy-minimal coarse-grained reconfigurable array (\autoref{fig:riptide:arch:block}).
% 
The 6$\times$6 fabric contains
heterogeneous PEs connected via a bufferless, 2D-torus NoC.
% 
A complete \riptide system contains a CGRA fabric, a RISC-V scalar core, and a 256KB
(8$\times$32KB banks) SRAM main memory.

\subsection{Tagless dataflow scheduling}

\riptide implements asynchronous dataflow firing via \emph{ordered dataflow} (\autoref{sec:background}).
%
By adding ordering operators where control may diverge,
%
\riptide ensures that tokens always match on arrival at a PE, obviating the need for tags.
%
Tagless, asynchronous firing has a low hardware cost (one bit per input plus
control logic),
%
and it lets \riptide tolerate variable operation latency
(e.g., due to bank conflicts) while eliminating the need for the compiler to reason about
operation timing.

\subsection{Processing elements}
\riptide's PEs perform all arithmetic and memory operations in the fabric.
%
\autoref{fig:riptide:arch:pe} shows the microarchitecture of a PE.
% 
The PE includes a functional unit (FU) and the \textmu core.
% 
The \textmu core interfaces with the NoC, buffers
output values, and interfaces with top-level fabric control for PE configuration.

\paragraph{Functional units.}
The \textmu core exposes a generic interface using a
latency-insensitive ready/valid protocol to make it easy to add new
operators.
%% that makes integrating
%% new FUs easy.
%% % 
%% Any FU implementing the interface can be configured to receive and send data over the NoC.
%
Inputs arrive on {\tt in\_data} when {\tt in\_valid} is high,
and are consumed when {\tt fu\_ready} is high.
%
The FU reserves space in the output channel by raising {\tt fu\_alloc}
(e.g., for pipelined, multi-cycle operations),
and output arrives on {\tt fu\_data} when {\tt fu\_valid} is high.
%
{\tt out\_ready} supplies back pressure from downstream PEs.
%
The remaining signals deal with top-level configuration and control.

\paragraph{Communication.}
The \textmu core decouples NoC communication from FU computation.
% 
The \textmu core tracks which inputs are valid, raises backpressure on
input ports when its FU is not ready, buffers intermediate results in
output channels, and sends results over the NoC.
%
Decoupling simplifies the FU.

\paragraph{Configuration.}
The \textmu core handles PE and FU configuration, storing configuration state in a
two-entry {\em configuration cache} that enables single-cycle reconfiguration. 
% 
Additionally, the \textmu core enables the fabric to overlap
reconfiguration of some PEs while others
finish computation on an old configuration.

\paragraph{PE types.}
\riptide includes a heterogeneous set of PEs:
\begin{compactitem}
\item \emph{Memory PEs} issue loads and stores to memory and have a ``row buffer'' that coalesces non-aliasing subword loads.

\item \emph{Arithmetic PEs} implement basic ALU operations, e.g., compare, bitwise logic, add, subtract, shift, etc.

\item \emph{Multiplier PEs} implement multiply, multiply + shift, multiply + fixed-point clip, and multiply + accumulate.

\item \emph{Control-flow PEs} implement steer, invariant, carry, %select,
  merge, and order (\autoref{sec:cf}) ---
%
but most of these are actually implemented in \riptide's NoC (see below).

\item \emph{Stream PEs} implement common affine iterators (\autoref{sec:cf}).
\end{compactitem}

\subsection{Bufferless NoC}
\label{arch:noc}

\riptide connects PEs via a statically configured, multi-hop, bufferless
on-chip network with routers.
%
Instead of buffering values in the NoC,
PEs buffer values in their output channel.
%
NoC buffers are a primary energy sink in prior CGRAs~\cite{karunaratne2017hycube,snafu},
and \riptide completely eliminates them.
%
Similarly, \riptide's NoC is statically routed to eliminate routing
look-up tables and flow-control mechanisms.
% 

% \subsection{Fabric control}
% \riptide has a top-level fabric controller that works with the \textmu cores of PEs 
% to control execution.
% % 
% When a kernel is invoked, the fabric controller raises {\tt ctrl\_en} high, enabling configured \textmu cores and corresponding FUs.
% %
% During execution, if the FU is not executing and/or does not have valid inputs, it must raise {\tt fu\_done}.
% % 
% The \textmu core passes this signal as {\tt ctrl\_done} to the fabric controller.
% % 
% Once fabric controller sees {\tt ctrl\_done} raised for every PE, execution is finished.
% % 
% The fabric controllers raises {\tt ctrl_clear} which resets the \textmu cores and functional units.
