\section{Experimental Methodology}
\label{riptide:method}

We evaluate a complete \riptide system: the compiler built using LLVM and the
microarchitecture fully implemented in RTL in an industrial, sub-28nm FinFET process.

\paragraph{Compiler.}
\riptide's compiler passes extend LLVM 12.0~\cite{llvm} and we compile workloads with
(\texttt{-Oz}) to optimize code size.  \riptide's compiler middle-end uses LLVM's
flow-insensitive alias analyses for memory ordering.
%
The backend uses Gurobi 9.5 to solve its ILP to map
to hardware~\cite{gurobi}.

\paragraph{Hardware.}
\riptide is a complete RTL hardware implementation, including the \riptide fabric, the RISC-V
(RV32IMACE) scalar core, and a 256KB (8$\times$32KB banks) SRAM main memory. 
% 
We use Cadence Xcelium to simulate RTL to verify correctness and measure
performance.
% 
We synthesize \riptide using Cadence Genus and an industrial-grade, sub-28nm,
high-threshold-voltage, FinFET PDK with compiled memories.
% 
To estimate power, we simulate full benchmarks post-synthesis and use Cadence
Joules to estimate power from annotated switching activities.

\paragraph{Baselines.}
The evaluation compares to several baselines---scalar, vector, \snafu, and
three ASICs---also entirely in RTL, using the same design
flow.
% 
All baselines and \riptide use the same scalar core and main memory.
% 
The scalar baseline is a simple, six-stage microcontroller.
% 
The vector baseline adds a single-lane co-processor~\cite{manic}.
% 
\snafu is the state-of-the-art ULP CGRA, but requires 
hand-coded vector assembly programs.
% 
We compare to three custom ASICs on {\tt dmm}, {\tt sort}, and {\tt fft} to evaluate the costs of
programmability in \riptide.

\paragraph{Benchmarks.}
We evaluate ten workloads important to the ULP domain running on random inputs.
% 
For the vector baseline, we vectorized all code by hand (except {\tt dfs}, which does not 
vectorize well).
% 
\snafu uses the vectorized code to generate its bitstreams.
% 
For \riptide, we compile and run the plain scalar C implementation of each benchmark. 
%
The only exception is {\tt sort}, for which we use merge sort on the scalar core and
a C version of radix sort for \riptide because it maps to the CGRA fabric.
% 
