\section{Discussion}
\label{riptide:discuss}
This chapter presented \riptide, the final piece in the new ULP system stack.
% 
\riptide's dataflow compiler makes it easy for a programmer to access the energy-efficiency of it's co-designed CGRA.
% 
\riptide provides a rich set of control-flow operators, letting it support arbitrary control flow and memory access on the CGRA fabric. 
% 
It implements these primitives without tagged tokens and offloads most control operations into its programmable on-chip network, saving energy and hardware resources.
% 
\riptide compiles applications written in C while using $25\%$ less energy vs. \snafu and $6.6\times$ less vs. a von Neumann core.

% So what trade-off is \riptide making to achieve these results? 
% 
% \subsection{Implications for General-Purpose Architecture and Dark Silicon}
% \label{riptide:implications}
% 
\paragraph{So where is the trade-off?}
While \riptide does not compromise much on energy or performance --- coming within 32\% and 3\%, respectively, on {\tt dmm} v. an equivalent ASIC --- it is not a free lunch.
%
There is a high area cost for \riptide's programmability: \riptide is 57$\times$ (only $9\times$ for {\tt fft}) larger than the ASIC.
%
The question is, is \riptide's programmability worth the extra area?

\riptide area is inflated partly because of low utilization on PEs that perform outer loops.
%
\riptide only supports one operation per PE, so entire PEs are consumed even if an operation fires rarely.
%
A future design could revisit this constraint to allow limited time-multiplexing,
either through fine-grain~\cite{weng2020hybrid} or coarse-grain~\cite{nguyen2021fifer} time multiplexing.

Regardless, the area difference shows potentially large cost savings from ASICs,
so long as a computation is performed frequently enough to overcome
ASICs' upfront design and verification costs.
%
Standardized, pervasive tasks like JPEG compression and wireless communication protocols are good candidates for ASICs.
%
But if the computation is prone to change
or used infrequently, then this cost advantage rapidly disappears.

\paragraph{``Garden of ASICs''}
Some have proposed that, with increasing transistor budgets and
stagnating power budgets, processors should embrace extreme
heterogeneity and assemble a large number of distinct
ASICs~\cite{venkatesh2010conservation,taylor2012dark}.
%
The ``garden of ASICs'' approach lets architects do something
with extra transistors, but it dramatically increases
system design and verification cost.
%
Moreover, the ``garden of ASICs'' approach creates herculean
challenges in system integration,
%
as there is no standard programming interface for ASICs,
obsolescence is monotonic and inevitable,
and programs must be somehow partitioned between ASICs
and cores with accompanying data coordination issues.

\paragraph{\riptide suggests an alternative approach}
Rather than spend area on ASICs that will idle most of the time, instead \riptide suggests building an energy-minimal, programmable dataflow fabric.
%
The two designs take similar area with a few dozen ASICs.
%
And the dataflow fabric is cheaper to design, more broadly applicable,
and easier to use --- programs can be simply compiled for a
different target.
%
Finally, as a general-purpose design, programmable dataflow fabrics
can create a self-sustaining ecosystem that aggregates optimizations
and achieves sufficient scale to justify cutting-edge silicon.
%
All told, while dataflow fabrics like \riptide are not a replacement
for ASICs by any means, they could play an important role in
improving the efficiency of general-purpose processing.