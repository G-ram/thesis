\section{Experimental Methodology}
\label{riptide:method}

We evaluate a complete \riptide system: the compiler built using LLVM and the
microarchitecture fully implemented in RTL in an industrial, sub-28nm FinFET process.

\paragraph{Compiler}
\riptide's compiler passes extend LLVM 12.0~\cite{llvm} and we compile workloads with
\texttt{-Oz} to optimize code size.  
% 
\riptide's compiler middle-end uses LLVM's
flow-insensitive alias analyses for memory ordering.
%
We evaluate both \riptide's SAT and ILP mappers (see \autoref{riptide:compiler:map}), but unless otherwise specified we use the ILP mapper.
% 
The ILP mapper uses CVXPY~\cite{cvxpy} and Gurobi 9.5~\cite{gurobi}.
%
The SAT mapper uses CaDiCal~\cite{cadical} to rewrite and simplify the problem's clauses and then uses a new parallel SAT solver, developed concurrently and based on YalSAT~\cite{yalsat}, to find a valid mapping.

\paragraph{Hardware}
\label{method:hardware}
\riptide is implemented completely in RTL, including the 6$\times$6 CGRA, RISC-V
(RV32EMC) scalar core, and 256KB
%(8$\times$32KB banks w/ work-conserving round-robin arbitration)
SRAM main memory. 
% 
We use Cadence Xcelium to verify correctness and measure
performance.
% 
We synthesize \riptide using Cadence Genus and a
high-threshold-voltage, FinFET PDK with compiled memories.
% 
To estimate power, we simulate full benchmarks post-synthesis and use Cadence
Joules to estimate power from annotated switching activities.

\paragraph{Baselines}
The evaluation compares to several baselines---scalar, vector, \snafu, and
three ASICs---also implemented entirely in RTL, using the same design
flow.
% 
All baselines and \riptide use the same scalar core and main memory.
% 
The scalar baseline is a simple, six-stage microcontroller\footnote{This is a more energy-efficient and higher performance design than the scalar baseline in \autoref{chapter:snafu}}.
% 
The vector baseline adds a single-lane co-processor~\cite{manic}.
% 
\snafu is the state-of-the-art energy-minimal CGRA.
%% , but requires 
%% hand-coded vector assembly programs.
%% % 
%% We compare to three custom ASICs on {\tt dmm}, {\tt sort}, and {\tt fft} to evaluate the costs of
%% programmability in \riptide.

\paragraph{Benchmarks}
We evaluate ten workloads important to the ULP domain on random inputs.
% 
For the vector baseline, we vectorized all code by hand (except {\tt dfs}, which does not 
vectorize well).
% 
\snafu uses the vectorized code to generate its bitstreams.
% 
For \riptide, we compile and run the plain C implementation of each benchmark.
%
The exceptions are {\tt sort}, for which we use merge sort on the scalar core and
radix sort for \riptide (because it maps entirely onto the fabric),
%because it more easily maps to the CGRA fabric.
and {\tt dmm}, for which, where explicitly noted, we tune its C implementation to maximize efficiency.
% 

%\paragraph{Metrics}
%We evaluate three main metrics: energy, performance, and the number of
%operations.
%% 
%We measure performance and energy across the full execution of each benchmark.
%% 
%Energy results are normalized to the scalar design and performance results are reported in speedup v.\ scalar.
%% 
%We evaluate the optimizations made by our compiler by the number of operations required to be mapped to hardware to run a particular benchmark.
