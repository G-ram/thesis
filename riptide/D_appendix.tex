\onecolumn
\appendix
\section{Appendix}
\label{sec:appendix}
% This section contains the SAT and ILP formulations for mapping. 

\tabRipTideMapVars
\autoref{tab:riptide:map:vars} lists the inputs and variables of the SAT and ILP formulations for mapping.
% 
The goal of either mapper is to solve for $M_{vn}$ and $M_{el}$, which map a DFG's vertices to hardware PEs and CF modules (hardware nodes) and a DFG's edges to hardware links, respectively.
% 
Matrices $C_{vn}$ and $C_{el}$ capture the compatibility of a DFG's vertex-to-hardware node (i.e., a memory operation must be mapped to a memory PE) and a DFG's edge-to-hardware link (to make sure ports match), respectively.
% 
The remaining matrices $H_{nl}$, $H_{ln}$, $H_{rl}$, $H_{lr}$, describe the topology of the CGRA fabric by specifying the connectedness of links to hardware nodes and routers.

\tabRipTideILP
\autoref{tab:riptide:ilp} describes the (binary) ILP formulation.
% 
The formulation minimizes average routing distance given the constraints in the table.

\tabRipTideSAT
\autoref{tab:riptide:sat} describes the SAT formulation.
% 
Since there is no objective, the formulation may yield longer routes, duplicate routes or routes with cycles.
% 
We post-process the routes to find the shortest between two nodes.
