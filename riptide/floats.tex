\newcommand{\figRipTideIntroResults}{
  \begin{figure}[t]
    \centering
    \begin{minipage}{0.32\linewidth}
    	\centering
	\includegraphics[width=\linewidth]{riptide/figures/pdf/loc_avg_legend-graph-crop.pdf}
	\includegraphics[width=\linewidth]{riptide/figures/pdf/loc_avg-graph-crop.pdf}
    \end{minipage}
    \hfill
    \begin{minipage}{0.32\linewidth}
    	\centering
	\includegraphics[width=\linewidth]{riptide/figures/pdf/energy_avg-graph-crop.pdf}
    \end{minipage}
    \hfill
    \begin{minipage}{0.32\linewidth}
    	\centering
	\includegraphics[width=\linewidth]{riptide/figures/pdf/perf_avg-graph-crop.pdf}
    \end{minipage}

    \caption{\riptide improves energy-efficiency and performance over the
      state of the art, while compiling programs from high-level C
      (v.\ vector assembly in SNAFU).}
    \label{fig:intro:results}
  \end{figure}
}

\newcommand{\figRipTideIntro}{
  \begin{figure}[t]
    \centering
    \includegraphics[width=0.9\linewidth]{riptide/figures/pdf/new_overview.pdf}
    \caption{\riptide is a co-designed compiler and CGRA architecture
      that executes programs written in a \emph{high-level language}
      with \emph{minimal energy} and high performance.
      %
      \riptide introduces new control-flow primitives to
      support common programming idioms, like deeply nested loops
      and irregular memory accesses, while minimizing overhead.
      %
      \riptide implements control flow \emph{in the NoC} to
      increase utilization and ease compilation.}
    \label{fig:intro}
  \end{figure}
}

\newcommand{\figRipTideCTime}{
  \begin{figure}[t]
    \centering
    \includegraphics[width=0.9\linewidth]{riptide/figures/pdf/ctime-graph-crop.pdf}
    \caption{Compilation times (16 threads, Intel i9-9900K) using SAT and ILP mapping. SAT is 15.1$\times$ faster than ILP on average.}
    \label{fig:ctime}
  \end{figure}
}

\newcommand{\figRipTideSATEnergy}{
  \begin{figure}[htb]
    \centering
    \includegraphics[width=0.9\linewidth]{riptide/figures/pdf/sat-graph-crop.pdf}
    \caption{Normalized energy of \riptide w/ SAT mapping v.\ ILP mapping. Solutions from the ILP mapper use 4.3\% less energy on average.}
    \label{fig:sat:energy}
  \end{figure}
}

\newcommand{\figRipTideSATResults}{
	\begin{figure}[htb]
		\centering
		\begin{subfigure}{0.49\linewidth}
			\begin{Overpic}{
		      \includegraphics[width=\linewidth]{riptide/figures/pdf/ctime-graph-crop.pdf}
		    }
		    \put (18,57){
		    	\includegraphics[height=0.14in]{riptide/figures/pdf/ctime_tlegend-graph-crop.pdf}
		    }
		  \end{Overpic}
			\caption{Compilation time.}
			\label{fig:sat:ctime}
		\end{subfigure}
		\hfill
		\begin{subfigure}{0.49\linewidth}
			\vspace{-0.6em}
			\includegraphics[width=\linewidth]{riptide/figures/pdf/sat-graph-crop.pdf}
			\caption{Normalized energy.}
			\label{fig:sat:energy}
		\end{subfigure}
		\caption{Compilation time (16 threads, Intel i9-9900K) and normalized energy (v.\ ILP) of SAT and ILP mappers. SAT is 15.1$\times$ faster than ILP, but uses 4.7\% more energy.}
		\label{fig:sat}
	\end{figure}
}

\newcommand{\figRipTideSystem}{
\begin{figure}[t]
	\centering
    \includegraphics[width=\linewidth]{riptide/figures/pdf/system.pdf}
    \caption{\riptide is a compiler and CGRA microarchitecture that maps
      high-level code with arbitrary control flow and memory access to
      an energy-minimal CGRA fabric. To maximize efficiency while
      minimizing area, \riptide implements control flow in the NoC.}
    \label{fig:system}
\end{figure}
}

\newcommand{\figRipTideCFGates}{
\begin{figure}[t]
	\centering
	\includegraphics[width=\linewidth]{riptide/figures/pdf/cf.pdf}
	\caption{Semantics of new control-flow operators in \riptide.}
	\label{fig:cf:gates}
\end{figure}
}

\newcommand{\figRipTideCFExecute}{
\begin{figure*}[t]
	\centering
	\includegraphics[width=\linewidth]{riptide/figures/pdf/execute.pdf}
	\caption{}
	\label{fig:cf:execute}
\end{figure*}
}

\newcommand{\figRipTideCompiler}{
\begin{figure*}[t]
	\centering
	\includegraphics[width=\linewidth]{riptide/figures/pdf/compiler.pdf}
	\caption{\riptidecomp's frontend and middle-end components. The frontend
    compiles C code to LLVM-IR using \texttt{clang}. The middle-end
    produces an optimized dataflow graph (DFG) that enforces memory ordering
    and \riptide's control paradigm.}
	\label{fig:compiler}
\end{figure*}
}

\newcommand{\figRipTideLSO}{
\begin{figure}[htb]
	\centering
	\includegraphics[width=0.8\linewidth]{riptide/figures/pdf/lso.pdf}
	\caption{\riptidecomp's middle-end enforces memory ordering. For 
    \texttt{example}, an ordering graph (OG) that is 
    iteratively pruned and reduced.}
	\label{fig:lso}
\end{figure}
}

\newcommand{\tabRipTideSnippets}{
\begin{table*}[t]
	\caption{Qualitative comparison of \riptide to prior work.
          \riptide's goal is to \emph{minimize energy}
          by executing \emph{entire functions} on the CGRA fabric.
          To achieve this, \riptide supports arbitrary control flow and irregular memory access,
          whereas prior CGRAs are limited to affine loops or a subset of common program idioms.
          \riptide compiles lightly annotated C to an efficient CGRA configuration.
        }
	\label{tab:background}
	\centering
	\resizebox{\linewidth}{!}{
		\scriptsize
		\begin{tabular}{p{0.35in}P{1.25in}P{1.25in}P{1.25in}P{1.25in}P{1.25in}P{1.35in}}
			\toprule & 
			 	% \bf HyCube~\cite{karunaratne2017hycube} & 
			 	\bf Revel~\cite{weng2020hybrid} & 
			 	\bf UE-CGRA~\cite{torng2021ultra} & 
			 	\bf Wavescalar~\cite{swanson2003wavescalar} &
			 	\bf \snafu~\cite{snafu} & 
			 	\bf \riptide \\[.5ex]
			\midrule
			% \bf Related &
			% 	\cite{hamzeh2012epimap,dave2018ramp,pager2015software,ureca,amp2020,lee2021ultra,chordmap,pathseeker,karunaratne2018dnestmap,hamzeh2014branch,balasubramanian2018laser} &
			% 	- &
			% 	\cite{voitsechov2014single,parashar2013triggered,trips,ttda} & 
			% 	- & 
			%	- \\[.5ex]
			\bf Goal & 
				% \raisebox{-.3\height}{\includegraphics[height=0.125in]{riptide/figures/pdf/watch.pdf}} Performance & 
				\raisebox{-.3\height}{\includegraphics[height=0.125in]{riptide/figures/pdf/watch.pdf}} Performance / Area & 
				\raisebox{-.3\height}{\includegraphics[height=0.125in]{riptide/figures/pdf/watch.pdf}} Performance or \linebreak \raisebox{-.3\height}{\includegraphics[height=0.125in]{riptide/figures/pdf/lightning.pdf}}Energy & 
				\raisebox{-.3\height}{\includegraphics[height=0.125in]{riptide/figures/pdf/watch.pdf}} Performance & 
				\raisebox{-.3\height}{\includegraphics[height=0.125in]{riptide/figures/pdf/lightning.pdf}} Energy & 
				\raisebox{-.3\height}{\includegraphics[height=0.125in]{riptide/figures/pdf/lightning.pdf}} Energy \\[.5ex]
			\bf Power & 
			% 10s mW & 
			100s mW & 1s mW & 1000s mW & <\,1 mW & <\,1 mW \\[.5ex]
			%% \bf Mapping
                        %% & Spatial and temporal
                        %% & Spatial and/or temporal
                        %% & Spatial and temporal
                        %% & Spatial only
                        %% & Spatial only \\[.5ex]
                        %% \bf Who maps?
                        %% & Compiler
                        %% & Compiler
                        %% & Hardware
                        %% & Compiler
                        %% & Compiler \\[.5ex]
      %% \bf \thead{$\dagger$ Constructs\\\& idioms} &
      %% \texttt{HL, AL, CE,} [\texttt{DLN}] & 
      %% \thead{\texttt{HL, DLN, AL,} \\ \texttt{CE, AP,} [\texttt{IMA}]} &
      %% \thead{\texttt{HL, DLN, IMA,} \\ \texttt{IAL, MO, CE}} &
      %% \thead{\texttt{IMA, AL, CE, AP}} & 
      %% \thead{\texttt{HL, DLN, IMA, IAL,} \\ \texttt{OMO, CE,} [\texttt{AP}]}
      %% \\[.5ex]
                        \bf Target
                        % & Affine loop nests
                        & Imperfectly nested loops
                        & Irregular inner loops
                        & Arbitrary programs
                        & Affine inner loops
                        & Arbitrary functions \\[.5ex]
                        \bf Code changes
                        % & Loop pragmas
                        & Loop pragmas
                        & None
                        & None
                        & Vector assembly
                        & Function annotation \\[.5ex]
			% \bf \shortstack{Program\\consructs}&  &  & & & \\[.5ex]
			%% \centering
			%% \vspace{-5em}
			%% \rotatebox[origin=c]{0}{\bf \shortstack{Program \\ support}}
                        \bf \vspace{-6.25em}Exemplar\newline program
                        &
			% \includegraphics[height=0.95in]{riptide/figures/pdf/program_2.pdf} &
			\includegraphics[height=0.95in]{riptide/figures/pdf/program_3.pdf} &
			\includegraphics[height=0.95in]{riptide/figures/pdf/program_7.pdf} &
			\includegraphics[height=0.95in]{riptide/figures/pdf/program_6.pdf} &
			\includegraphics[height=0.95in]{riptide/figures/pdf/program_5.pdf} &
			\includegraphics[height=0.95in]{riptide/figures/pdf/program_4.pdf} \\
			\bottomrule
		\end{tabular}
	}
  %% \begin{tablenotes}
  %% \scriptsize
  %% \item \textbf{Legend}: 
  %% \texttt{HL} = uses high-level language; 
  %% \texttt{DLN} = deep loop nests; 
  %% \texttt{IMA} = indirect memory accesses; 
  %% \texttt{AL} = affine loops; 
  %% \texttt{IAL} = irregular or affine loops;  
  %%       \texttt{MO} = memory ordering; 
  %% \item \texttt{OMO} = optimized \texttt{MO}; 
  %% \texttt{CE} = conditional execution; 
  %% \texttt{AP} = annotations for programming; 
  %% \texttt{AT} = annotations for targeting/offloading 
  %% \item \textbf{*} Scheduling listed as "spatial" or "temporal" are \textit{static} schedules.
  %% \item \textbf{$\dagger$}: Limited support for constructs/idioms indicated using brackets (e.g., [\texttt{CE}])
  %% \item \textbf{$\ddagger$}: We uses a compiler command-line option to specify which function(s) to target for \riptide. 
  %% \item
  %% \end{tablenotes}
\end{table*}
}

\newcommand{\figRipTideLSOResults}{
\begin{figure}[htb]
	\centering
	\includegraphics[width=\linewidth]{riptide/figures/pdf/lso-graph-crop.pdf}
	\caption{}
	\label{fig:eval:fin}
\end{figure}
}

\newcommand{\tabRipTideMotivate}{
	\begin{table}[htb]
	\label{tab:motivate}
	\caption{Qualitative comparison of \riptide to prior work.}
	\centering
	\resizebox{\linewidth}{!}{
		\renewcommand{\arraystretch}{1.5}
		 \begin{tabular}{p{3cm}|ccccc|c} 
	 		\toprule 
		 		&	
				\bf HiMap~\cite{himap} & 
				\bf 4D-CGRA~\cite{4dcgra} &
				\bf Plasticine~\cite{plasticine} &
				\bf Wavescalar~\cite{swanson2003wavescalar} &
				\bf Revel~\cite{weng2020hybrid} & 
				\bf \thead{\riptide \\ (this work)} 
			\\[.5ex] 
			\midrule
				\bf Most related &
				\cite{hamzeh2012epimap,dave2018ramp,pager2015software,ureca,amp2020,lee2021ultra,chordmap,pathseeker} &
				\cite{karunaratne2018dnestmap,hamzeh2014branch,balasubramanian2018laser} &
				\cite{dyser,q100} &
				\cite{voitsechov2014single,parashar2013triggered,trips,ttda} &
				- &
				\cite{snafu}
			\\[.5ex]
				\bf Goal &
				Performance & 
				Performance & 
				Performance & 
				Performance & 
				Performance & 
				Energy
			\\[.5ex]
				\bf Power &
				10s mW &
				10s mW &
				1-3 W &
				NA &
				100s mW &
				<1 mW
			\\[.5ex]
				\bf Dataflow &
				Systolic &
				Systolic &
				Ordered &
				Tagged &
				Systolic \& Tagged &
				Ordered	
			\\[.5ex]
				\bf Conditional \newline execution &
				NA &
				Steering &
				Selection &
				Steering &
				Selection &
				Steering
			\\[.5ex]
				\bf Scheduling \newline dimensions	&
				Time \& space &
				Time \& space &
				Space &
				Space &
				Time \&/ space &
				Space
			\\[.5ex]
				\bf Source language &
				C & C & Delite & C & C & C
			\\[.5ex]
				\bf Resource \newline heterogeneity &
				\xmark & \xmark & Limited & \xmark & \cmark & \cmark
			\\[.5ex]
				\bf No tag-matching &
				\cmark & \xmark & \cmark & \xmark & \cmark/\xmark & \cmark
			\\[.5ex]
				\bf Operations w/ \newline unknown latencies &
				\xmark & \xmark & \cmark & \cmark & \cmark/\xmark & \cmark	
			\\[.5ex]
				\bf Deep loop nests &
				\cmark & Limited & Limited & \cmark & Fit pattern & \cmark	
			\\[.5ex]
				\bf Irregular loops &
				\xmark & \xmark & \xmark & \cmark & \xmark & \cmark	
			\\[.5ex]
				\bf Irregular memory \newline accesses &
				\xmark & \xmark & Fit pattern & \cmark & Fit pattern & \cmark	
			\\[.5ex]
				\bf Enforces memory \newline ordering &
				\xmark & \xmark & DSL-assisted & \cmark & None & \cmark
			\\[.5ex]
			\bottomrule
		 \end{tabular}
	}
	\end{table}
}

\newcommand{\tabRipTideTax}{
	\begin{table}[htb]
	\label{tab:motivate}
	\caption{Qualitative comparison of \riptide to prior work.}
  \begin{threeparttable}
	\centering
	\resizebox{\linewidth}{!}{
		\renewcommand{\arraystretch}{1.5}
		 \begin{tabular}{p{3cm}|cccc|c} 
	 		\toprule 
		 		\bf CGRA Paradigm &	
        \bf Static * & 
        \bf \thead{High-perf. \\ (ordered dataflow)} &
        \bf \thead{High-perf. \\ (tag-token dataflow)} &
				\bf Hybrid & 
        \bf \thead{ULP \\ (ordered dataflow)}
			\\[.5ex] 
			\midrule
				\bf Most related &
        \thead{HiMap~\cite{himap}, 4D-CGRA~\cite{4dcgra}, etc. \\ (~\cite{hamzeh2012epimap,dave2018ramp,pager2015software,ureca,amp2020,lee2021ultra,chordmap,pathseeker, karunaratne2018dnestmap,hamzeh2014branch,balasubramanian2018laser})} &
        \thead{Plasticine$\ddagger$~\cite{plasticine}, \\ etc. (~\cite{dyser,q100})} &
        \thead{Wavescalar~\cite{swanson2003wavescalar}, \\ etc. (~\cite{voitsechov2014single,parashar2013triggered,trips,ttda})} &
				Revel~\cite{weng2020hybrid} &
        \thead{\textbf{\riptide (this work)} \\ and SNAFU~\cite{snafu}}
			\\[.5ex]
				\bf Power consumption &
				10s mW &
				10s mW &
				1-3 W &
				100s mW &
				<1 mW
			\\[.5ex]
        \bf Program support $\dagger$ &
        \texttt{RA, AL,} [\texttt{CE, DLN}] & 
        \texttt{AL, CE,} [\texttt{DLN, IMA}] & 
        \thead{\texttt{IAL, CE, DLN,} \\ \texttt{IMA, MO,} [\texttt{OMO}]} &
        \texttt{RA, AL, CE,} [\texttt{IMA}] &
        \thead{\texttt{IAL, CE, DLN,} \\ \texttt{IMA, OMO,} [\texttt{RA}]}
			\\[.5ex]
				\bf Offload granularity &
        \thead{Single loops or loop \\ nests (see~\autoref{fig:snippets}a,b)} &
        \thead{Loop nests or  whole \\ programs (see~\autoref{fig:snippets}b,c)} &
        \thead{Whole programs \\ (see~\autoref{fig:snippets}d)} &
        \thead{Loop nests \\ (see~\autoref{fig:snippets}c)} &
        \thead{Whole programs \\ (see~\autoref{fig:snippets}d)}	
      \\[.5ex]
			\bottomrule
		 \end{tabular}
	}
  \begin{tablenotes}
    \tiny
  \item \textbf{Legend}: \texttt{RA} = requires annotations; \texttt{DLN} = deep loop nests; \texttt{IMA} = indirect memory accesses; \texttt{AL} = affine loops;
  \item \texttt{IAL} = [irregular | affine] loops; \texttt{MO} = memory ordering; \texttt{OMO} = optimized \texttt{MO}; \texttt{CE} = conditional execution
  \item \textbf{$\dagger$}: Limited program support is indicated using brackets (e.g., [\texttt{CE}])
  \item \textbf{$\ddagger$}: We note that Plasticine uses a parallel patterns DSL, different than most CGRAs. 
  \item \textbf{*} "Static" refers to kernels scheduled to the CGRA across space \textit{and} time.
  \end{tablenotes}
  \end{threeparttable}
	\end{table}
}

\newcommand{\figRipTideISA}{
\begin{table}[t]
    \begin{center}
      \caption{\riptide's instruction set architecture (ISA).}
      \label{tab:isa}
      \scriptsize
      \resizebox{\linewidth}{!}{%
      \begin{tabular*}{4.0in}{lccl}
        \toprule 
        \textbf{Operator(s)}    & \textbf{Category} & \textbf{Symbol(s)}                                & \textbf{Semantics}                         \\
        \midrule 
        Basic binary ops        & Arithmetic        & $+$, $-$, $<<$, $!=$, etc.                        & $a$ \texttt{op} $b$                        \\ 
        Multiply, clip          & Multiplier        & $*$, \texttt{clip}                                & $a$ \texttt{op} $b$                        \\ 
        Load                    & Memory            & \texttt{ld}                                       & \texttt{ld} $base$, $idx$(, $dep$)                 \\
        Store                   & Memory            & \texttt{st}                                       & \texttt{st} $base$, $idx$, $val$(, $dep$)           \\
        Select                  & Control Flow      & \texttt{sel}                                      & $cond$ \texttt{?} $val0$ \texttt{:} $val1$ \\ 
        Steer, carry, invariant & Control Flow      & (\texttt{T} | \texttt{F}), \texttt{C}, \texttt{I} & See \autoref{fig:cf:gates}                 \\ 
        Merge, order            & Synchronization   & \texttt{M}, \texttt{O}                            & See \autoref{fig:cf:gates}                 \\ 
        Stream                  & Stream            & \texttt{STR}                                      & See \autoref{fig:cf:gates}                 \\ 
        \bottomrule 
      \end{tabular*}}
    \end{center}
\end{table}
}

\newcommand{\figRipTideCF}{
\begin{figure*}[t]
	\begin{minipage}{0.24\linewidth}
		\centering
		\includegraphics[height=1.52in]{riptide/figures/pdf/cf.pdf}
		\caption{Control-flow gates.}
		\label{fig:cf:gates}
	\end{minipage}
	\hfill
	\begin{minipage}{0.74\linewidth}
		\centering
		\includegraphics[width=\linewidth]{riptide/figures/pdf/execute.pdf}
		\caption{Shows an example execution.}
		\label{fig:cf:execute}
	\end{minipage}
\end{figure*}
}

\newcommand{\figRipTideArch}{
\begin{figure*}[t]
\begin{minipage}{0.3\linewidth}
		\centering
		\includegraphics[height=2in]{riptide/figures/pdf/block.pdf}
		\caption{\riptide's ULP CGRA fabric.}
		\label{fig:arch:block}
	\end{minipage}
	\hfill
	\hspace{4em}
	\begin{minipage}{0.3\linewidth}
		\centering
		\includegraphics[height=2in]{riptide/figures/pdf/pe.pdf}
		\caption{PE microarchitecture}
		\label{fig:arch:pe}
	\end{minipage}
	\hfill
	\begin{minipage}{0.3\linewidth}
		\centering
		\includegraphics[height=2in]{riptide/figures/pdf/router.pdf}
		\caption{Router microarchitecture}
		\label{fig:arch:router}
	\end{minipage}
	% \hfill
	% \begin{minipage}{0.36\linewidth}
	% 	\centering
	% 	\includegraphics[height=0.5in]{riptide/figures/pdf/filter.pdf}
	% 	\hfill
	% 	\vspace{0.5em}
	% 	% \begin{table}
	% 	\small
	% 	\begin{tabular}{ll}
	% 	\toprule
	% 	\textbf{Operation} & \textbf{Explanation} \\
	% 	\midrule
	% 	No connection	 & - 		\\
	% 	Direct         & O = I 	\\
	% 	Steer      	 & (\autoref{fig:cf:gates}) \\
	% 	Carry	       & A = carry, I = init (\autoref{fig:cf:gates}) \\
	% 	Invariant      & (\autoref{fig:cf:gates}) \\
	% 	Select         & O = (D) ? I : A \\
	% 	Merge         & O = I || A \\
	% 	Order         & O = (D) ? I : A (D \& I or !D \& A) \\
	% 	\bottomrule
	% 	\end{tabular}
	% 	\caption{Operations supported in NoC.}
	% 	\label{fig:micro:ops}
	% 	% \end{table}
	% \end{minipage}
\end{figure*}
}

\newcommand{\figRipTideFilter}{
\begin{figure}[htb]
	\centering
	\includegraphics[width=0.5\linewidth]{riptide/figures/pdf/filter.pdf}
	\caption{Basic control-flow module in \riptide's routers.}
	\label{fig:filter}
\end{figure}
}

\newcommand{\figRipTideEnergyResults}{
\begin{figure*}[t]
	\begin{minipage}{\linewidth}
		\centering
		\includegraphics[height=0.15in]{riptide/figures/pdf/energy_legend-graph-crop.pdf}
		
		\vspace{0.2em}

		\includegraphics[width=\linewidth]{riptide/figures/pdf/energy-graph-crop.pdf}
		\caption{Energy (v. \riptide) of scalar, vector, \snafu, \riptide across ten benchmarks. \riptide uses 25\% less energy than \snafu. }
		\label{fig:eval:energy}
	\end{minipage}
\end{figure*}
}
\newcommand{\figRipTidePerformanceResults}{
\begin{figure*}[t]
	\begin{minipage}{\linewidth}
		\centering
		\vspace{1em}
		\includegraphics[width=\linewidth]{riptide/figures/pdf/perf-graph-crop.pdf}
		\caption{Speedup (v.\ scalar) of scalar, vector, \snafu, and \riptide across ten benchmarks. \riptide is 17\% faster than \snafu.}
		\label{fig:eval:perf}
	\end{minipage}
\end{figure*}
}

\newcommand{\figRipTidePrimaryCompilerResults}{
\begin{figure*}[htb]
	\begin{minipage}{\linewidth}
		\centering
		\includegraphics[height=0.15in]{riptide/figures/pdf/counts_legend-graph-crop.pdf}
		
		\vspace{0.2em}

		\includegraphics[width=\linewidth]{riptide/figures/pdf/counts-graph-crop.pdf}
        \caption{Operator counts for ten different benchmarks. Starting with an unoptimized, unordered baseline (\texttt{Raw}), compiler optimizations reduce operator counts while enforcing memory ordering, making it feasible to map benchmarks to hardware.}
		\label{fig:eval:ops}
	\end{minipage}
	% \begin{minipage}{0.49\linewidth}
	% 	\centering
	% 	\vspace{1em}
	% 	\includegraphics[width=\linewidth]{riptide/figures/pdf/insn-graph-crop.pdf}
		
	% 	\caption{Dynamic scalar instruction counts}
	% 	\label{fig:eval:insn}
	% \end{minipage}
\end{figure*}
}

\newcommand{\figRipTideFINResults}{
\begin{figure}[htb]
	\centering
	\includegraphics[width=\linewidth]{riptide/figures/pdf/fin-graph-crop.pdf}
	\caption{Control flow in the NoC saves energy. \riptide uses 45\% , 40\%, and 27\% less energy than \riptide w/ No CFiN, a fabric where all CF ops are PEs (``All PEs''), and a fabric that fuses CF ops into PEs (``Fused'').  }
	\label{fig:eval:fin}
\end{figure}
}

\newcommand{\figRipTideLoCResults}{
\begin{figure}[htb]
	\centering
	\includegraphics[width=0.6\linewidth]{riptide/figures/pdf/loc_legend-graph-crop.pdf}
	
	\vspace{0.1em}

	\includegraphics[width=\linewidth]{riptide/figures/pdf/loc-graph-crop.pdf}
	\caption{The number of code additions for ten benchmarks running on scalar, vector, \snafu, and \riptide. \riptide requires no hand-coded assembly unlike vector and \snafu.}
	\label{fig:eval:loc}
\end{figure}
}

\newcommand{\figRipTideBlock}{
\begin{figure}[htb]
	\centering
	\includegraphics[width=0.9\linewidth]{riptide/figures/pdf/block.pdf}
	\caption{}
	\label{fig:block}
\end{figure}
}

\newcommand{\tabRipTideMicroParams}{
\begin{table}
\caption{Microarchitectural parameters.}
\label{tab:params}
\end{table}
}

\newcommand{\tabRipTideAppParams}{
\begin{table}
\caption{Application parameters.}
\label{tab:apps}
\end{table}
}

\newcommand{\figRipTideProgress}{
\begin{figure}[t]
	\centering
	\begin{subfigure}{0.49\linewidth}
		\centering
		\includegraphics[width=0.9\linewidth]{riptide/figures/pdf/progress_energy-graph-crop.pdf}	
		\caption{Energy savings.}
		\label{fig:progress:energy}
	\end{subfigure}
	\hfill
	\begin{subfigure}{0.49\linewidth}
		\centering
		\includegraphics[width=0.9\linewidth]{riptide/figures/pdf/progress_perf-graph-crop.pdf}	
		\caption{Speedup.}
		\label{fig:progress:perf}
	\end{subfigure}
\caption{Energy savings and speedup of DNN inference on \riptide v.\ MSP430, ARM Cortex-M3, and our scalar core.}
\label{fig:progress}
\end{figure}
}

\newcommand{\figRipTideArea}{
  \begin{figure}[h]
    \begin{minipage}{0.45\linewidth}
      \caption{Area breakdown for a complete \riptide system.
        Area is dominated by the CGRA fabric and main memory
        (SRAM + arbitration logic),
        eaching taking about half of the system.
        The scalar core is just 2.5\% of system area.
        The NoC takes 54\% of CGRA area,
        and PEs (\textmu core + FUs) take 42\%.
        }
    \label{fig:eval:area}
    \end{minipage}
    \hfill
    \begin{minipage}{0.49\linewidth}
      \includegraphics[width=\linewidth]{riptide/figures/pdf/area_pie-crop.pdf}
    \end{minipage}
  \end{figure}
}

\newcommand{\tabRipTideSATHelpers}{
\begin{table}[htb]
  \caption{SAT formulation helper functions.}
  \label{tab:sat:helpers}
  \centering
  \resizebox{\linewidth}{!}{
    \begin{tabular}{p{2.2cm}|l}
      \toprule
      \bf Function & \bf Explanation \\[.5ex] 
      \midrule
      \lstinline[style=custompython]|add(clauses)| & 
        add clauses to formulation \\[.5ex] 
      \lstinline[style=custompython]|where(condition)| & 
        returns indices where condition is true \\[.5ex] 
      \lstinline[style=custompython]|or(a, axis=None)| & 
        conjuction of array elements along an axis \\[.5ex] 
      \lstinline[style=custompython]|atmost1(a)| & 
        constrains at most one array element to true \\[.5ex] 
      \lstinline[style=custompython]|exactly1(a)| & 
        constrains exactly one array element to true \\[.5ex] 
      \lstinline[style=custompython]|src(e)| & 
        returns source vertex of edge, e \\[.5ex] 
      \lstinline[style=custompython]|dst(e)| & 
        returns destination vertex of edge, e \\[.5ex] 
      \lstinline[style=custompython]|can\_share(e1, e2)| & 
       checks if edges, e1 \& e2, can share the same link \\[.5ex] 
      \midrule
    \end{tabular} 
  }
\end{table}	
}

\newcommand{\tabRipTideMapVars}{
\begin{table}[h]
  \centering
  % Yeah I know this hacky but to get all these tables on the same page is real
  % challenge
  %% \captionsetup{singlelinecheck=false,justification=raggedright,margin=38pt}
  \caption{Inputs \& variables of ILP \& SAT formulations.}
  \label{tab:map:vars}

  \resizebox{0.5\linewidth}{!}{
    \begin{tabular}{ll}
			\toprule
			\bf Input & \bf Explanation \\
			\midrule
			$V$ & Set of DFG vertices \\
			$E$ & Set of DFG edges \\
			$N$ & Set of hardware nodes (PEs \& CF-modules) \\
			$F$ & Set of CF-modules $F \subset N$ \\
			$R$ & Set of hardware routers \\
			$L$ & Set of hardware links \\
			$C_{el}(E, L) = 1$ if $e$ can map to $l$ & Edge-link compatibility matrix \\
			$C_{vn}(V, N) = 1$ if $v$ can map to $n$ & Vertex-node compatibility matrix \\
			$H_{ln}(L, N) = 1$ if $l$ originates from $n$ & Link-to-node matrix \\
			$H_{nl}(N, L) = 1$ if $l$ comes from $n$ & Node-to-link matrix \\
			$H_{lr}(L, R) = 1$ if $l$ originates from $r$ & Link-to-router matrix \\
			$H_{rl}(R, L) = 1$ if $l$ comes from $r$ & Router-to-link matrix \\
			\toprule
			\bf Variable & \bf Explanation \\
			\midrule
			$M_{vn}(V,N) = 1$ if $v$ is mapped to $n$& Vertex-to-node matrix \\
			$M_{el}(E,L) = 1$ if $e$ is mapped to $l$& Edge-to-link matrix \\
			\bottomrule
		\end{tabular}
  }
\end{table}	
}

\newcommand{\tabRipTideSAT}{
\begin{table*}[h]
\caption{SAT formulation.}
\label{tab:sat}
\centering
\resizebox{\linewidth}{!}{
	\begin{tabular}{ll}
	\toprule
		\bf Clause & \bf Explanation \\
	\midrule
	$\forall e \in E, l \in L | C_{el}(e,l) = 0, \neg M_{el}(e,l)$ & 
			Edges are mapped to compatible links \\ 
	$\forall v \in V, n \in N | C_{vn}(v,n) = 0, \neg M_{vn}(v,n)$ &	
			Vertices are mapped to compatible nodes \\	
	$\forall v \in V, ExactlyOne(\{M_{vn}(v,n) | n \in N\})$ &
			Every vertex must be mapped to a node \\
	$\forall n \in N, AtMostOne(\{M_{vn}(v,n) | v \in V\})$ &
			No node can be used by more than one vertex	\\
	$\forall r \in R, e \in E, \lor_{l|H_{rl}(r,l)} M_{el}(e,l) \iff \lor_{H_{lr}(l,r)} M_{el}(e,l)$ &
			An edge mapped to incoming link to a router must also be mapped to an outgoing link \\
	$\forall r \in R, e \in E, AtMostOne(\{M_{el}(e,l)| l \in L \; and \; H_{rl}(r,l)\})$ &
			An edge can only be mapped to a single outgoing link of a router \\
	$\forall e \in E, n \in N | n \notin F, \lor_{l|H_{nl}(n,l)} M_{el}(e,l) \iff M_{vn}(src(e), n)$ &
			If a vertex is mapped to a non-CF node, then the input edges are mapped to incoming links\\
	$\forall e \in E, n \in N | n \notin F, \lor_{l|H_{ln}(l,n)} M_{el}(e,l) \iff M_{vn}(dst(e), n)$ &
			If a vertex is mapped to a non-CF node, then the output edges are mapped to outgoing links\\
	$\forall l \in L, e_1 \in E, e_2 \in E | src(e_1) \neq src(e_2), \neg M_{el}(e_1,l) \lor \neg M_{el}(e_2, l)$ &
			Edges that do not share the same source are not mapped to the same links\\
	$\forall e \in E, n \in F, K_{nl}(e,n) \lor \neg M_{vn}(src(e), n)$ &
			If a vertex is mapped to a CF node, then the output edges are mapped to outgoing links\\
	$\forall e \in E, n \in F, K_{ln}(e,n) \lor \neg M_{vn}(dst(e), n)$ &
			If a vertex is mapped to a CF node, then the input edges are mapped to incoming links\\
	$(\forall e \in E, n \in F, (K_{ln}(e,n) \lor K_{n} \lor \neg K_{nl}(e,n) )\land$
 	&
			Unused CF-modules can pass through edges\\
	\hspace{4em}$(\neg K_{ln}(e,n) \lor K_{n}(n) \lor K_{nl}(e,n)) \land$
	$(\neg K_{ln}(e,n) \lor \neg M_{vn}(src(e), n))$ & \\
	$\forall e \in E, n \in F, l \in L | H_{ln}(l,n), \neg M_{el}(e,l) \lor K_{ln}(e)$ &
			An edge mapped to an output link of CF-module cannot be mapped to an input of the CF-module\\
	\midrule
	\end{tabular}
}

\resizebox{\linewidth}{!}{
  \begin{tabular}{l|l|l|l|l}
		$src(e) := v \in V $ and v is the source of e &
		$dst(e) := v \in V $ and v is the destination of e&
		$K_{nl}(e,n) := \lor_{l|H_{nl}(n,l)}M_{el}(e,l)$ &
		$K_{ln}(e,n) := \lor_{l|H_{ln}(l, n)}M_{el}(e,l)$&
		$K_{n}(n) := \lor_{v\in V}M_{vn}(v,n)$\\
	\end{tabular}
}
\end{table*}
}

\newcommand{\tabRipTideILP}{
\begin{table*}[h]
\caption{ILP formulation.}
\label{tab:ilp}
\centering
\resizebox{0.4\linewidth}{!}{
	\begin{tabular}{ll}
		\bf Objective: &
		\textbf{\textit{minimize}} $\sum_{e \in E, l \in L} M_{el}(e, l)$ \textit{subject to}
	\end{tabular}
}

\resizebox{\linewidth}{!}{
	\begin{tabular}{ll}
	\toprule
		\bf Constraint & \bf Explanation \\
	\midrule
		$\forall e \in E, l \in L, M_{el}(e,l) \le C_{el}(e,l)$ &
		Edges are mapped to compatible links \\	
		$\forall v \in V, n \in N, M_{vn}(v,n) \le C_{vn}(v,n)$ &
		Vertices are mapped to compatible nodes \\
		$\forall v \in V, \sum_{n \in N} M_{vn}(v,n) = 1$ &
		Every vertex must be mapped to a node \\
		$\forall n \in N, \sum_{v \in V} M_{vn}(v,n) \le 1$ &
		No node can be used by more than one vertex \\
		$\forall e \in E, r \in R, \sum_{l\in L}M_{el}(e,l)H_{lr}(l,r) = \sum_{l \in L}M_{el}(e,l)H_{rl}(r,l)$ &
		Flow into a router must equal the flow out \\
		$\forall e \in E, n \in N | n \notin F, \sum_{l \in L}M_{e,l}(e)H_{nl}(n, l) = M_{vn}(src(e),n)$ &
		If a vertex is mapped to a non-CF node, then the output edges are mapped to outgoing links\\
		$\forall e \in E, n \in N | n \notin F, \sum_{l \in L}M_{e,l}(e)H_{ln}(l, n) = M_{vn}(dst(e),n)$ &
		If a vertex is mapped to a non-CF node, then the input edges are mapped to incoming links\\
		$\forall l \in L, e_1 \in E, M_{el}(e_1,l) + max_{e_2 \in E | src(e_1) \ne src(e_2)} M_{el}(e_2,l)\le 1$ &
		Edges that do not share the same source are not mapped to the same links\\
		$\forall e \in E, n \in F, \sum_{l \in L} M_{el}(e,l)H_{ln}(l,n) + \sum_{v \in V}M_{vn}(v,n) \ge \sum_{l \in L} M_{el}(e,l)H_{nl}(n,l)$ &
		Unused CF-modules can pass through edges \\
		$\forall e \in E, n \in F, \sum_{l \in L} M_{el}(e,l)H_{ln}(l,n) \le \sum_{v \in V}M_{vn}(v,n) + \sum_{l \in L} M_{el}(e,l)H_{nl}(n,l)$ &
		Unused CF-modules can pass through edges \\
		$\forall e \in E, n \in F, \sum_{l \in L} M_{el}(e,l)H_{nl}(n, l) \ge M_{vn}(src(e), n)$ &
		If a vertex is mapped to a CF node, then the output edges are mapped to outgoing links\\
		$\forall e \in E, n \in F, \sum_{l \in L} M_{el}(e,l)H_{ln}(l, n) \ge M_{vn}(dst(e), n)$ &
		If a vertex is mapped to a CF node, then the input edges are mapped to incoming links\\
	\midrule
	\end{tabular}
}
\resizebox{0.5\linewidth}{!}{
	\begin{tabular}{l|l}
		$src(e) := v \in V $ and v is the source of e &
		$dst(e) := v \in V $ and v is the destination of e
	\end{tabular}
}
\end{table*}
}

\newcommand{\lowtext}[1]{\raisebox{-0.5em}[0pt][0pt]{#1\ \ }}

\newcommand{\tabRipTideEvalCompare}{
  \begin{table*}[h]
    \centering
    \caption{Comparison of \riptide to other low-power CGRAs.
      \riptide supports a broader set of programs
      while improving energy efficiency.
      \riptide's performance is low, but this is primarily because
      we have not yet pushed frequency (there is ample
      timing slack at 50\,MHz).}
    \label{tab:eval:compare}
    \footnotesize
%%     \resizebox{0.8\linewidth}{!}{
      \begin{tabular}{lcccc@{\hskip 2em}cc}
	\toprule
	& 
	% \bf CMA-1~\cite{cma} &
	% \bf ULP-SRP~\cite{srp} & 
	\bf \makecell[c]{HyCube\\testchip$^\star$~\cite{wang2019hycube}} &
        \bf \makecell[c]{HM-HyCube\\REVAMP~\cite{revamp}} &
	\bf \makecell[c]{UE-CGRA\\\cite{torng2021ultra,torng_pan_2021}} &
%% 	\bf Revel$^\dag$~\cite{nowatzki2018hybrid} &
	\bf \makecell[c]{\snafu\\\cite{snafu}} &
	\multicolumn{2}{c}{\makecell{\bf \riptide\\\em(this work)}}\\ 
	\midrule
	Irregular loops &
	% \xmark & 
	% \xmark & 
	\xmark & 
	\xmark & 
	\cmark & 
%% 	\xmark &
	\xmark & 
	\multicolumn{2}{c}{\cmark} \\
	Loop nesting &
	% \xmark & 
	% \xmark & 
	\xmark & 
	\xmark & 
	\xmark & 
%% 	\cmark &
	\xmark & 
	\multicolumn{2}{c}{\cmark} \\
	Memory ordering &
	% \xmark & 
	% \xmark & 
	\xmark & 
	\xmark & 
	\xmark & 
%% 	\xmark & 
	\xmark & 
	\multicolumn{2}{c}{\cmark} \\
	Variable-latency ops &
	% \xmark & 
	% \xmark & 
	\xmark & 
	\xmark & 
	\xmark & 
%% 	\cmark &
	\xmark & 
	\multicolumn{2}{c}{\cmark} \\
        \midrule
	Node &
	% Fujitsu\,65 &
	% TSMC\,40LP &
	40LP &
        22 &
	TSMC 28 &
%% 	UMC 28 &
	Intel 22FFL &
	\multicolumn{2}{c}{Intel 22FFL} \\
        Fabric dimensions &
        % 8$\times$8 &
        % 3$\times$3 &
        4$\times$4 &
        6$\times$6 &
        8$\times$8 &
%%         5$\times$5 &
        6$\times$6 &
        \multicolumn{2}{c}{6$\times$6} \\
	Fabric area (mm$^{2}$) &
	% 5.4 &
	% --- &
	--- & % 2.86
	0.2 &
	0.25 &
%% 	0.13$\times$8 &
	0.27 &
	\multicolumn{2}{c}{0.25} \\
        Frequency (MHz) &
        % 210 &
        % 7 &
        488 &
        100 &
        750 &
%%         1250 &
        50 &
        \multicolumn{2}{c}{50} \\
        Memory size (KBs) &
        % 6.25 &
        % 384 &
        4 &
        64 &
        192 &
%%         256 &
        256 &
        \multicolumn{2}{c}{256} \\
        \midrule
        Benchmark &
        % {\tt sum-abs-diff} &
        % {\tt fft} &
        {\tt fft} &
        Linear algebra &
        {\tt fft} &
%%    	{\tt fft} &
        {\tt fft} &
        {\tt fft} & {\tt dmm$^\ddag$} \\
	Fabric power (mW) &
	% 11.2 &
	% --- &
	--- &
	8.4 &
	14.0 &
%% 	1282 &
	0.54 &
	0.24 & 0.50 \\
	System power (mW) &
	% --- &
	% 0.55 &
	140 &
	--- &
	16.7 &
%% 	1663 &
	0.74 &
	0.52 & 0.91\\
	Performance (MOPS) &
	% 2720 &
	% --- &
	5380 &
	--- &
	625 &
%% 	20,078 &
	71 &
	62 & 164 \\
	Fabric efficiency (MOPS/mW) &
	% 243 &
	% --- &
	--- &
	103 &
	45 &
%% 	16 &
	134 &
	254 & 328 \\
	System efficiency (MOPS/mW) &
	% --- &
	% --- &
	26 &
	--- &
	38 &
%% 	12 &
	97 &
	117 & 180\\
	\bottomrule
      \end{tabular}
%%     }
    
    \vspace{1pt}
    
    %% 	\resizebox{0.6\linewidth}{!}{
    \begin{tabular}{c}
      $^\star$\,Silicon implementation. \quad
%%       $^\dag$\,High-level simulation. \quad
      $^\ddag$\,Hand-tuned C software. % that unrolls twice along output column. 
      % $^2$dmm operations: $ops\,=\,2n^3$ where n is matrix dimension. \\
      % $^3$fft operations: $ops\,=\,2n(4nlog_2(n) - 6n + 8)$. where n is matrix dimension.\\
    \end{tabular}
    %% 	}
  \end{table*}
}

% Revamp
% X
% X
% X
% X
% Commercial 22
% 6x6
% 0.2
% 100
% Linear algebra kernels
% 8.4
% ---
% Performance?
% 172
% ---
