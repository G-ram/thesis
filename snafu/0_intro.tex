The opportunity for tiny, ULP devices is enormous~\cite{lucia2017intermittent}, but enabling sophisticated processing on these devices remains challenging.
% 
Prior chapters have presented progress on this challenge, contributing to a new ULP sensor system stack.
% 
\autoref{chapter:sonic} described softwate to enable machine inference on commodity MCUs and \autoref{chapter:manic} proposed \manic and \msilicon, a new computer architecture and silicon prototype, respectively.
% 
\manic is an energy-efficient vector-dataflow co-processor that is a big improvement over COTS devices.
% 
However, it stills fall short due to high switching activity
in the shared execution pipeline, a significant inefficiency at ULP-scale.
% 
Eliminating these overheads can reduce energy by nearly half, proving that, despite their low operating power, \manic and other existing programmable ULP designs~\cite{dally:ieee08:elm,hempstead2005ultra,warneke200417,nazhandali2005energy} are not energy-minimal.
% 
Alternatively, custom ASICs would achieve energy-minimality, but come at high upfront cost~\cite{hotmobile2021} and with severely limited application scope, risking quick obsolece.
% 
Thus, there is still a need for new architectures that achieve ULP ($<$1\,mW), \emph{energy-minimal} operation while maintaining a high degree of design flexibility and ease of programmability.

\paragraph{\Ulp CGRAs are the answer}
%
This chapter presents \snafuframe,%
\footnote{\underline{S}imple \underline{N}etwork of \underline{A}rbitrary \underline{F}unctional \underline{U}nits.}
a framework to generate ULP, energy-minimal coarse-grain reconfigurable arrays (CGRAs).
%
\snafuframe CGRAs execute in a \emph{spatial vector-dataflow} fashion,
mapping a dataflow graph (DFG) spatially across a fabric of processing elements (PEs),
applying the same DFG to many input data values,
and routing intermediate values directly from producers to consumers.
%
The insight is that spatial vector-dataflow minimizes instruction and data-movement energy, just like \manic,
but also eliminates unnecessary switching activity because operations do not share execution hardware.

The major difference from most prior CGRAs~\cite{plasticine,dyser,nowatzki:isca17:stream-dataflow,goldstein2000piperench,trips,weng2020dsagen,weng2020hybrid,voitsechov2014single,mishra2006tartan,tan2018stitch,karunaratne2017hycube,voitsechov2018inter,evx} is the extreme design point
--- \snafuframe operates at \emph{orders-of-magnitude lower energy and power budget},
demanding an exclusive focus on energy-minimal design.
%
\snafuframe is designed from the ground up to minimize energy, even at
the cost of area or performance.
%
For example, \snafuframe schedules only one operation per PE, which
minimizes switching activity (energy) but increases the number of PEs needed (area).
%
As a result of such design choices, \snafuframe comes within 2.6$\times$
of ASIC energy efficiency while remaining fully programmable.
%

\snafuframe generates ULP CGRAs from a high-level description of available PEs and the fabric topology.
%
\snafuframe defines a standard PE interface that lets designers \emph{``bring your own function unit''}
and easily integrate it into a ULP CGRA,
along with a library of common PEs.
%
The \snafuframe framework schedules operation execution and routes intermediate values to dependent operations
while consuming minimal energy.
%
\snafuframe is easy to use:
it includes a compiler that maps vectorized C-code to efficient CGRA bitstreams,
and it reduces design effort of tape-out via top-down synthesis of CGRAs.

\figSNAFUIntro

\paragraph{Contributions} This chapter contributes the following:
\begin{compactitem}
\item We present \snafuframe, the first flexible CGRA-generator for ULP, energy-minimal systems.
  \snafuframe makes it easy to integrate new functional units,
  compile programs to energy-efficient bitstreams,
  and produce tape-out-ready hardware.
  
\item We discuss the key design choices in \snafuframe that minimize energy:
  scheduling at most one operation per PE;
  asynchronous dataflow without tag-token matching;
  statically routed, bufferless, multi-hop NoC;
  and producer-side buffering of intermediate values.
  
\item We describe \snafuarch, a complete ULP system-on-chip with a CGRA fabric,
  RISC-V scalar core, and memory.
  We implement \snafuarch in an industrial sub-28\,nm FinFET process with compiled memories.
  \snafuarch operates at $<$1\,mW at 50\,MHz.
  \snafuarch reduces energy by $81\%$ vs.\ a scalar core
  and $41\%$ vs.\ \manic;
  and improves performance by $9.9\times$ vs.\ a scalar core
  and $4.4\times$ vs. \manic.

\item Finally, we quantify the cost of programmability through three
  comprehensive case studies that compare \snafuarch against
  fixed-function ASIC designs. We find that programmability comes at
  relatively low cost: on average, \snafuarch takes $2.6\times$ more
  energy and $2.1\times$ more time than an ASIC for the same
  workload. We break down \snafuarch's energy in detail, showing that it
  is possible to close the gap further while retaining significant
  general-purpose programmability. These results call into
  question the need for extreme specialization in most ULP
  deployments.
  
\end{compactitem}