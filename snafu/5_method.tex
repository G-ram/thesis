\section{Experimental Methodology}
\label{snafu:method}

\figSNAFUDie
\tabSNAFUTabs

We implemented \snafuarch as well as three baselines entirely in RTL and synthesized each system using
an industrial sub-28\,nm FinFET process with compiled memories.
% 
We evaluated the systems using post-synthesis timing, power, and energy models.
% 
Additionally, we placed and routed \snafuarch (see \autoref{fig:snafu:die}) to validate top-down synthesis.

\paragraph{Software-hardware stack}
We developed a complete software and hardware stack for \snafu.
% 
We implemented \snafuarch, its 256\,KB banked memory, and its five-stage pipelined \mbox{RISC-V} scalar core in RTL and verified correctness by running full applications in simulation using both Verilator~\cite{snyder2004verilator} and Cadence Xcelium RTL simulator~\cite{xcelium}.
% 
We synthesized the design using Cadence Genus~\cite{genus} and an industrial sub-28\,nm, high-threshold voltage FinFET PDK with compiled memories.
% (high-level SRAM models introduce significant error; \autoref{snafu:motivation}).
% 
Next, we placed and routed the design using Cadence Innovus~\cite{innovus};
% 
\autoref{fig:snafu:die} shows the layout.
% 
We also developed \snafuframe's compiler that converts vectorized C-code to an optimal fabric configuration and injects the bitstream into the application binary.
% 
Finally, we simulated full applications post-synthesis, annotated switching, and
used Cadence Joules~\cite{joules} to estimate power.

\paragraph{Baselines}
We compare \snafuarch against three baseline systems:
\emph{(i)}~a RISC-V scalar core with a standard five-stage pipeline,
\emph{(ii)}~a vector baseline that implements the \mbox{RISC-V V} vector extension,
and \emph{(iii)}~\manic~\cite{manic}, the prior state-of-the-art in general-purpose ULP design.
%
The scalar core is representative of typical ULP microcontrollers like the TI MSP430~\cite{msp430fr5994}.
% 
Each baseline is implemented entirely in RTL using the same design flow. 
%
\autoref{tab:snafu:params} shows their microarchitectural parameters.
%
The vector baseline and \manic both have a single vector lane, which minimizes energy at the cost of performance.

\paragraph{Benchmarks}
We evaluate \snafuarch, \manic, and the vector baseline across ten benchmarks on three different input sizes, shown in \autoref{tab:snafu:app}.
%
We use random inputs, generated offline. 
% 
For \manic and the vector baseline, each benchmark has been vectorized from a corresponding plain-C implementation.
% 
For \snafuarch, these vectorized benchmarks are fed into our compiler to produce CGRA-configuration bitstreams.
% 
In cases where the benchmarks contain a permutation, the kernel is manually split and pieces are individually fed into our compiler.

\paragraph{Metrics}
%
We evaluate \snafuarch and the baselines primarily on their energy efficiency
and secondarily on their performance.
%
We measure the full execution of each benchmark after initializing the system, and we measure efficiency by the energy to execute the complete benchmark normalized to either the scalar baseline or \snafuarch.
%
We measure performance by execution time (cycles) or speedup normalized to either the scalar baseline or \snafuarch.
