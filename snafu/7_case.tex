\figSNAFUAccel
\section{The Cost of Programmability}
\label{snafu:case}
%
With the mainstream acceptance of architectural specialization,
the architecture community faces an ongoing debate over the ideal form and degree of specialization.
%
Some results suggest a need for drastic specialization to compete
with ASICs~\cite{hameed2010understanding,taylor2012dark,shao2014aladdin};
%
whereas others argue that programmable designs can compete by
adopting a non-von Neumann execution
model~\cite{nowatzki2016dsa,nowatzki2017domain}.

We contribute to this debate by performing an apples-to-apples
comparison of a programmable design (\snafuarch) against three hand-coded ASICs,
%
demonstrating that the cost of programmability is low in the ULP domain.
%
We compare end-to-end systems in the same technology and design flow
using an industrial PDK with compiled memories.
%
Our results thus avoid pitfalls of prior studies based on simulations,
analytical models, or extrapolations from different
technologies.

We find that, on average, \snafuarch uses $2.6\times$ more energy
and $2.1\times$ more time than an ASIC implementation.
%
Breaking down the sources of inefficiency in \snafuarch,
%
we find that \snafu lets designers trade off programmability and efficiency
via simple, incremental design changes.
%
\snafu makes {\em selective} specialization easy, letting the architect
focus their design effort where it yields the most efficiency gains.

\paragraph{\snafu is within $2.6\times$ of ASIC energy}
\autoref{fig:snafu:accel} shows the energy of DMM, Sort, and FFT on large inputs.
%
The leftmost bars in \autoref{fig:snafu:accel} represent \snafuarch and the rightmost bars %, {\scshape DMM-Accel}, {\scshape Sort-Accel}, and {\scshape FFT2D-Accel},
represent a fixed-function, statically scheduled ASIC implementation of the same algorithm.
% 
\snafuarch uses as little as $1.8\times$ and on average $2.6\times$ more energy than the ASICs.
%
To explain the gap,
we now consider intermediate designs that build up from the ASIC back to \snafuarch.

\paragraph{\snafuarch, inner loops, \& Amdahl's Law}
\snafu maps only inner loops to its fabric and runs outer loops on the
scalar core, limiting its benefits to the fraction of dynamic
execution spent in inner loops.
%
To make a fair comparison, we built ASICs for DMM and FFT that
accelerate only the inner-loop of the kernel ({\scshape Dot-Accel} and
{\scshape FFT1D-Accel}), just like \snafuarch.
% 
These designs add $33\%$ energy to run outer loops on the scalar core,
reducing the energy gap to $2.2\times$ (vs.\ $2.5\times$ for these benchmarks previously).
%
A future version of \snafu could eliminate this extra scalar energy by mapping outer loops to its fabric~\cite{weng2020hybrid}.

\paragraph{Asynchronous dataflow firing adds minimal overhead}
Next, we add asynchronous dataflow firing to the ASIC designs
({\scshape $*$-Async} bars).
% 
Comparing {\scshape Async} designs to the ASICs shows that asynchronous dataflow
firing adds little energy overhead, just 3\% in DMM and Sort.
%
The 30\% overhead in {\scshape FFT-Async} is inessential and could be optimized away:
\snafu's current implementation of asynchronous dataflow firing adds an unnecessary pipeline stage when reading scratchpad memories in the ASIC designs.

\paragraph{Closing the gap with negligible design effort}
Next, we consider variants of \snafuarch to break down the cost of
software programmability.
%
{\scshape \snafu-Bespoke} hardwires the fabric configuration, eliminating unused
logic during synthesis (like prior work~\cite{cherupalli:isca17:bespoke}),
removing \snafuarch's software programmability.
%
{\scshape \snafu-Bespoke} uses $54\%$ more energy than the {\scshape Async} designs.
% 
The gap in energy with {\scshape Async} can be attributed to \snafu's logic for
predicated execution and the operation set that \snafu implements, which is not
well suited to every application.  For instance, {\scshape Sort-Accel} can
select bits directly, whereas \snafu must do a {\tt vshift} and {\tt vand}.

\paragraph{BYOFU makes it easy to specialize where it matters}
\snafu's flexible, ``bring your own functional unit'' design
(\autoref{snafu:flexible:byofu}) makes it easy to add missing operations
to improve efficiency.
%
To illustrate this,
Sort-{\sc Byofu} and FFT-{\sc Byofu}
improve {\sc \snafu-Bespoke}'s efficiency by adding specialized PEs to the fabric.
% 
For Sort, we add a PE that fuses {\tt vshift} and {\tt vand} operations.
% 
For FFT, we size scratchpads properly for their data.
% 
In both cases, the energy savings are significant.
% 
{\scshape \snafu-Bespoke} uses $20\%$ more energy than the {\scshape Byofu} designs, and the {\scshape Byofu} designs come within $44\%$ of the {\sc Async} ASIC designs.
%
These savings come with much lower design effort than a full ASIC, and we
expect the gap would narrow further if more {\sc Byofu} PEs were added to the fabric.
%

\paragraph{The cost of software programmability is low}
Next, {\scshape \snafu-Tailored} specializes the CGRA to eliminate
extraneous PEs, routers, and NoC links, but is not hardwired like {\scshape \snafu-Bespoke} or {\scshape \snafu-Byofu} ---
%
i.e., \snafu-\fauxsc{Tailored} is where the design becomes programmable in software.
% 
{\scshape \snafu-Tailored} uses only $15\%$ more energy than {\scshape
\snafu-Bespoke}, illustrating that the cost of software programmability is low.


The original \snafuarch design uses just $10\%$ more energy
than {\scshape \snafu-Tailored}.
%
This gap includes the cost of PEs, routers, and links that are
not needed by every application, but may be used by some.
%
This gap is also small, suggesting that \emph{general-purpose}
programmability also has a low cost.
%

\newcommand{\hbreak}{\vspace{-0.6em}\begin{center}\textcolor{gray}{\line(1,0){150}}\end{center}\vspace{-0.1em}}

\hbreak

The above comparisons yield three major takeaways.
%
The big picture is that \bigemph{the total cost of \snafu's programmability is {\mbox{2--3$\times$}}} in energy and time vs.\ a fully specialized ASIC.
%
While significant, this gap is much smaller than the 25$\times$ (or larger) gap found in some prior studies~\cite{hameed2010understanding}.%
%
\footnote{The smaller gap comes from comparing to a programmable design that
exploits \emph{both} vector and dataflow techniques to improve
efficiency~\cite{nowatzki2017domain,nowatzki2016dsa}.}
%
Whether further specialization is worthwhile will depend on the application;
for many applications, a 2--3$\times$ gap is good enough~\cite{hotmobile2021}.

Moreover, for applications that benefit from more specialization, \bigemph{\snafu allows for selective specialization with incremental design effort}
to trade off efficiency and programmability.
%
\autoref{fig:snafu:accel} shows that by tailoring the fabric topology,
hardwiring configuration state,
or partially specializing PEs,
designers can get within 2$\times$ of ASIC efficiency
at a small fraction of ASIC design effort.
%
Designers can incrementally scale their design effort
to the degree of specialization appropriate for their application.

Finally, digging deeper, we can better understand the cost of programmability as separate
%
costs incurred at design time (i.e., hardware implementation quality)
and run time (i.e., overhead for running software).
%
With current synthesis tools,
\bigemph{software programmability itself is surprisingly cheap,
  but carries a hidden design-time cost:}
%
the gap between \snafu and {\sc \snafu-Bespoke} is just $27\%$,
%
whereas the gap between {\sc \snafu-Bespoke} and ASICs is $2.1\times$.
% 
Even when runtime reconfigurability is removed from a design,
synthesis tools cannot produce circuits similar to a hand-coded ASIC
because they do not understand the intent of RTL.
%
Barring a breakthrough in synthesis, this challenge will remain. 
% 
\snafu provides a path forward:
% 
{\sc BYOFU} lets a designer specialize for critical operations,
enabling programmable designs to compete with ASICs at a fraction of the design effort
and without forfeiting programmability.
