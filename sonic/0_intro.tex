The maturation of energy-harvesting technology and the recent
emergence of viable intermittent computing models creates the
opportunity to build sophisticated battery-less systems with most of
the computing, sensing, and communicating capabilities of existing
battery-powered systems.
%
Many future IoT applications require frequent decision making, e.g.,
when to trigger a battery-draining camera,
%
and these decisions must be taken locally,
as it is often impractically expensive % (in energy or latency)
to communicate with other devices.
%
Future IoT applications will require {\em local} inference on raw sensor data,
and their performance will be determined by inference accuracy. Using energy
numbers from recent state-of-the-art systems, we show that such local inference
can improve end-to-end application performance by 480$\times$ or more.

Recently, deep neural networks (DNNs)~\cite{alexnet, vgg, googlenet} have made
large strides in inference accuracy.
%
DNNs enable sophisticated inference using limited, noisy inputs, relying on
rich models learned from many examples.
%
Unfortunately, while DNNs are much more accurate than traditional
alternatives~\cite{gupta2017protonn, Mitchell:1997:ML:541177}, they are also
more computationally demanding.

Typical neural networks use tens of millions of weights and require
billions of compute operations~\cite{vgg,googlenet,alexnet}.
%
These networks target high-powered,
throughput-optimized processors like GPUs or Google's TPU, which
executes up to 9 trillion operations per second while drawing around
40 watts of power~\cite{jouppi:isca17:tpu}.
%
Even a small DNN (e.g., LeNet~\cite{lecun:ieee89:lenet}) has over a million
weights and millions of operations.
%
The most efficient DNN accelerators optimize for performance as
well as energy efficiency and consume hundreds of
mW~\cite{han:isca16:eie,chen:asplos14:diannao,du:isca15:shidiannao,chen:isca16:eyeriss}.

\paragraph{Challenges}
%
In stark contrast to these high-performance systems, energy-harvesting devices
use simple microcontrollers (MCUs) built for extreme low-power operation.
These MCUs systems run at low frequency (1--16 MHz) and have very small
memories (tens or hundreds of kilobytes).
%
Their simple architectures limit them to executing a few 
million operations per second, while consuming only 3--5mW%
---a power envelope two orders of magnitude lower than recent DNN accelerators.

DNN inference on these devices is unexplored, and several challenges
must be overcome to enable emerging IoT applications on
energy-harvesting systems built from commodity components.
%
Most importantly, energy-harvesting systems operate
\emph{intermittently} as power becomes available, complicating the
development of efficient, correct software.
%
The operating period depends on the properties of the
power system, but is short---typically around 100,000 instructions.
%
As a result, \emph{existing DNN inference implementations do not tolerate 
  intermittent operation}.

Recent work proposed software systems that guarantee correct
execution on intermittent power for arbitrary
programs~\cite{dino,ratchet,clank,chain,alpaca,mayfly}.
%
These systems add significant runtime overheads to ensure correctness,
slowing down DNN inference by on average $10\times$ in our experiments.
%
What these systems have missed is the opportunity to \emph{exploit the
  structure of the computation to lower the cost of guaranteeing correctness.}
%
This missed opportunity is especially costly for highly structured and
loop-heavy computations like DNN inference.

\paragraph{Our approach and contributions}
%
We present the \emph{first demonstration of intermittent DNN
inference} on real-world neural networks running on a widely
available energy-harvesting system.
%
We make the following contributions:

\begin{compactitem}
\item We first analyze where energy is spent in an energy-harvesting system
and show that inference accuracy largely determines IoT
application performance (\autoref{sonic:motivation}).
%
This motivates using DNNs despite their added cost over simpler but
less accurate inference techniques.
%
\item Building on this analysis, we present \genesis, a tool that
automatically compresses networks to maximize IoT application
performance (\autoref{sonic:genesis}).
%
\genesis uses known compression techniques%
~\cite{nabhan1994toward, han:iclr16:deep-compression,
  chollet2016xception, bhattacharya2016sparsification};
%
our contribution is that \genesis optimally balances inference energy
v.\ accuracy. % to maximize IoT application performance.
%
\item We design and implement \sonic, 
  a software system for DNN inference with specialized support for intermittent execution
  (\autoref{sonic:sonic}).
  %
  To ensure correctness at low overhead,
  \sonic introduces \emph{loop continuation},
  which exploits the regular structure of DNN inference
  to selectively violate task-based abstractions from prior work~\cite{alpaca},
  allowing direct modification of non-volatile memory.
%
  Loop continuation is safe because
  \sonic ensures loop iterations are idempotent
  through \emph{loop-ordered buffering} (for convolutional layers) and \emph{sparse undo-logging} (for sparse fully-connected layers).
%
These techniques let \sonic resume from where it left off after a power failure,
{eliminating task transitions and wasted work} that plague prior task-based systems.
%
\item Finally, we build \tails to show how to incorporate hardware
  acceleration into \sonic (\autoref{sonic:tails}). \tails uses
  hardware available in some microcontrollers to accelerate matrix
  multiplication and convolution.
  %
  \tails automatically calibrates its parallelism
  to ensure correctness with intermittent power.

\end{compactitem}

\noindent
We evaluate \sonictails on a TI MSP430
microcontroller~\cite{msp430fr5994} using an RF-energy
harvester~\cite{powercastboard,powercasttransmitter} (Secs.~\autoref{sonic:methodology}~\&~\autoref{sonic:evaluation}).
%
On three real-world DNNs~\cite{lecun:ieee89:lenet,okgoogle,har},
\sonic improves inference efficiency by $6.9\times$ on average
over Alpaca~\cite{alpaca}, a state-of-the-art intermittent system.
%
\tails exploits DMA and SIMD to further improve efficiency by $12.2\times$ on average.

We conclude with a discussion of the limitations of current energy-harvesting MCUs and the need for new ULP architectures (\autoref{sonic:discuss}).


