\section{System overview}
\label{sonic:overview}

This paper describes the first system for performing DNN inference
efficiently on intermittently-operating, energy-harvest\-ing
devices. \autoref{fig:overview} shows the new system components in this work
and how they produce an efficient, intermittence-safe executable starting from
a high-level DNN model description.  There are three main components to the
system: \genesis, \sonic, and \tails.

\genesis ({\underline g}enerating {\underline e}nergy-aware {\underline n}etworks for {\underline
  e}fficien{\underline s}y on {\underline i}ntermittent {\underline s}ystems) is a tool that automatically
optimizes a DNN, starting from
a programmer's high-level description of the network.  \genesis attempts to compress each layer of the network using well-known separation and pruning techniques.
\genesis's goal is
to \emph{find a network that optimizes \metric}
while meeting resource
constraints.  As \autoref{fig:overview} shows, \genesis's input is
a network description and its output is an optimally compressed network. \autoref{sonic:genesis} describes \genesis.

\sonic ({\underline s}oftware-{\underline o}nly {\underline n}eural {\underline i}ntermittent {\underline
c}omputing) is an intermittence-safe,
task-based API and runtime system that
includes specialized support for DNN inference
that \emph{safely ``breaks the rules'' of existing task-based systems
to improve performance}.
\sonic is compatible with existing task-based
frameworks~\cite{chain,alpaca}, allowing seamless integration into larger applications.
% 
\autoref{sonic:sonic} describes \sonic in detail.

\tails ({\underline t}ile-{\underline a}ccelerated {\underline i}ntermittent {\underline L}EA {\underline
s}upport) is an alternative to the \sonic runtime library that
leverages hardware vector acceleration, specifically targeting the TI Low
Energy Accelerator (LEA)~\cite{lea}.  To use \tails, the
programmer need only link their compiled binary to the \tails-enabled
runtime system.  This runtime includes all of \sonic's optimizations and a
suite of hardware-accelerated vector operations, such as convolutions. 
% 
\autoref{sonic:tails} describes \tails in detail.

Starting with a high-level network description, a
programmer can use \genesis, \sonic, and \tails to build an efficient,
intermittent DNN-enabled application that meets resource constraints, is robust
to intermittent operation, and leverages widely available hardware acceleration.
Our code and datasets can be found at: \textsf{\href{https://github.com/CMUAbstract/SONIC}{https://github.com/CMUAbstract/SONIC}}. 
