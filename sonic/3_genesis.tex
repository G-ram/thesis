\figSONICGenesisNNs
\section{Optimal DNN compression with \genesis}
\label{sonic:genesis}

The first challenge to overcome in \sonictails is fitting neural networks into the
resource constraints of energy-harvesting systems.
%
In particular, the limited memory capacity of current microcontrollers
imposes a hard constraint on networks.
%
We have developed a tool called \genesis
that automatically
explores different configurations of a baseline neural network,
applying separation and pruning techniques (\autoref{sonic:background}) to reduce the network's resource requirements.
%
\genesis improves upon these known techniques by
optimally balancing inference energy and true positive/negative rates to maximize \metric,
building on the the model in \autoref{sonic:motivation}.

\subsection{Neural networks considered in this paper}

%
This paper considers three networks, summarized in \autoref{tab:nns}.
%
To represent image-based applications (e.g., wildlife monitoring and
disaster recovery), we consider MNIST~\cite{lecun1998mnist}. We consider
MNIST instead of ImageNet because ImageNet's large images 
do not fit in a resource-constrained device's memory.
%
To represent wearable applications, we consider human activity recognition
(HAR). HAR classifies activities using accelerometer data~\cite{har}.
%
To represent audio applications, we consider Google keyword
spotting (OkG)~\cite{okgoogle}, which classifies words in audio snippets.

We also evaluated binary neural networks and several SVM models
and found that they perform poorly on current energy-harvesting MCUs.
%
A 99\%-accurate binary network for MNIST required 4.4MB of weights~\cite{courbariaux2016binarized},
exceeding the device's scant memory, and compressing this to 360KB lost nearly 10\% accuracy~\cite{binarynetgithub}.
%
Likewise, no SVM model that fit on the device was competitive with 
the DNN models~\cite{lecun1998gradient}: measured by \metric, SVM under-performed 
by 2$\times$ on MNIST and by 8$\times$ on HAR, and we could not find an SVM model 
for OkG that performed anywhere close to the DNN.

\subsection{Fitting networks on energy-harvesting systems}

\genesis evaluates many compressed configurations of a 
network and builds a Pareto frontier.
%
Compression has trade-offs in four dimensions, difficult to capture with a pareto curve;
these include true negative rate, true positive rate, memory size (i.e.,
parameters), and compute/energy (i.e., operations).
%
Fully-connected layers typically dominate memory, whereas convolutional layers
dominate compute. \genesis compresses both.

\genesis compresses each layer using two known techniques: separation and pruning.
%
Separation (or rank decomposition) splits an $m \times n$
fully-connected layer into two $m \times k$ and $k \times n$ matrix
multiplications, or an $m \times n \times k$ convolutional filter into three $m
\times 1 \times 1$, $1 \times n \times 1$, and $1 \times 1 \times k$, 
filters~\cite{chollet2016xception, bhattacharya2016sparsification}.
%
\genesis separates layers using the Tucker tensor decomposition,
using the high-order orthogonal iteration algorithm~\cite{tucker1966some, de2000best, de2000multilinear}.
%
Pruning involves removing parameters below a given threshold, since
they have small impact on results~\cite{nabhan1994toward, han:iclr16:deep-compression}.

\genesis sweeps parameters for both separation and pruning across each layer of
the network, re-training the network after compression to improve
accuracy.
%
\genesis relies on the Ray Tune black box optimizer with the Median
Stopping Rule to explore the configuration space~\cite{golovin2017google, moritz2017ray}.
%
\autoref{fig:genesis:train} shows the results for the networks in
\autoref{tab:genesis:nns}.
%
Each marker on the figure represents one compressed configuration,
shown by inference accuracy on the $y$-axis and inference energy on
the $x$-axis.
%
Feasible configurations (i.e., ones that fit in our device's 
small memory; see \autoref{sonic:methodology}) are shown as green
circles and infeasible configurations are grey $\times$s.
%
Note that the original configuration (large $\times$) is infeasible for all three networks,
meaning that they cannot be na\"ively ported to the device because their parameters would not fit in memory.

\autoref{fig:genesis:train} also shows the Pareto frontier for each
compression technique.
%
Generally, pruning is more effective than separation, but the
techniques are complementary.

\tabSONICleGenesisNNS

\subsection{Choosing a neural network configuration}

\genesis estimates a configuration's \metric using the model
from \autoref{sonic:motivation}, specifically \autoref{eq:impj}.
%
The user specifies $E_\text{sense}$ and $E_\text{comm}$ for their
application as well as per-compute-operation energy cost.
%
From these parameters, \genesis estimates $E_\text{infer}$ for each
configuration, and uses the inference accuracy from the prior training step to
estimate application performance.
%
The user can specify which class in the training set is
``interesting,'' letting \genesis compute true positive $t_p$ and negative $t_n$ rates for the
specific application.

\autoref{fig:genesis:perf} shows the results by mapping each point in
\autoref{fig:genesis:train} through the model.
%
For these results, we use $E_\text{sense}$ from
\autoref{sonic:motivation}, per-operation energy from our \sonictails
prototype in \autoref{sonic:methodology}, and estimate $E_\text{comm}$
from input size assuming OpenChirp networking~\cite{dongare2017openchirp}.

\genesis chooses the feasible configuration that maximizes
estimated end-to-end performance (i.e., \metric).
%
\autoref{fig:genesis:perf} shows that this choice is non-trivial.
%
True positive, true negative, and inference energy affect end-to-end application
performance in ways that are difficult to predict. Simply choosing the most accurate
configuration, as the twisty blue curve suggests in \autoref{fig:genesis:perf}, 
is insufficient since it may waste too much energy or underperform other 
configurations on true positive or true negative rates.
