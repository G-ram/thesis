\section{Efficient intermittent inference with \sonic}
\label{sonic:sonic}

\sonic is the first software system optimized for inference on
resource-constrained, intermittently operating devices.
%
\sonic supports operations common to most DNN computations,
exposing them to the programmer through a simple API.
%
\sonic's functionality is implemented as a group of {\em tasks}
supported by the \sonic runtime system, which is a modified version of the
Alpaca runtime system~\cite{alpaca}.
%
These tasks implement DNN functionality, and the \sonic runtime system 
guarantees correct intermittent operation.

Specializing intermittence support for DNN inference yields large benefits.
Prior task-based intermittent execution models~\cite{alpaca,chain} can degrade performance by up to 19$\times$
and by 10$\times$ on average
(\autoref{sonic:evaluation}).
\sonic dramatically reduces these overheads to just 25\%--75\% over
a standard baseline of DNN inference that does not tolerate intermittent operation.

\sonic achieves these gains by eliminating the three major sources of overhead in prior task-based systems:
redo-logging, task transitions, and wasted work (\autoref{sonic:background}).
%
Our key technique is \emph{loop continuation},
which selectively violates the task abstraction
for loop index variables.
Loop continuation lets \sonic directly modify loop indices without frequent and expensive saving and restoring.
By writing loop indices directly to non-volatile memory,
\sonic checkpoints its progress after each loop iteration,
eliminating expensive task transitions
and wasting work upon power failure.

Loop continuation is safe because \sonic
ensures that each loop iteration is idempotent.
\sonic ensures idempotence 
in convolutional and fully-connected layers
through \emph{loop-ordered buffering} and \emph{sparse undo-logging}.
These two techniques ensure idempotence without statically privatizing or dynamically checkpointing data,
avoiding the overheads imposed by prior task-based systems.

\subsection{The \sonic API}

The \sonic API lets the programmer describe a DNN's structure through 
common linear algebra primitives.
%
Just as a programmer chains tasks together in a task-based intermittent
programming model~\cite{chain,alpaca,mayfly}, the programmer chains \sonic's
tasks together to represent the control and data flow of a DNN inference
pipeline. 
%
\sonic's API exposes functionality that the programmer invokes like
any other task in their program (specifically, a {\em modular task
  group}~\cite{alpaca,chain}).
%
Though \sonic ``breaks the rules'' of a typical task-based
intermittent system, the programmer does not need to reason about these
differences when they are writing a program using the \sonic API.
%
The program-level behavioral guarantee that \sonic provides is the
same as the one underlying other task-based intermittent execution
models: a \sonic task will execute atomically despite power
interruptions by ensuring that repeated, interrupted attempts to
execute are idempotent.

\subsection{The \sonic runtime implementation}
\label{sonic:sonic:runtime}

DNN inference is dominated by loops within each layer of the neural network.
\sonic optimizes DNN inference by ensuring that these loops execute correctly
on intermittent power while adding much less overhead than prior task-based systems.

\paragraph{Loops in task-based systems}
A typical task-based intermittent system sees two kinds of loops: {\em short
loops} and {\em long loops}.  All iterations of a {\em short loop} fit in a
single task and will complete without consuming more energy than the device can
buffer.  A short loop maintains control state in volatile memory and these
variables clear on power failure. When power resumes, the task restarts and
completes.  Data manipulated by a short loop are usually non-volatile (i.e.,
``task-shared''~\cite{alpaca}) and if read and updated, they must be backed up
(either statically or dynamically) to ensure they remain consistent.  The
problem with short loops is that they always restart from the beginning,
wastefully repeating loop work that was already done. 
%
In contrast, a {\em long loop} with many iterations does not fit in a single
task; a long loop demands more energy than the device can buffer and may never
terminate. A programmer must split loop iterations across tasks, requiring a
task transition on each iteration and requiring control state and data to be
non-volatile and backed up.  
%
The problem with long loops is that may not terminate and, when
split across tasks, impose hefty privatization and task transition overheads.

\figCompDesigns

{\em Task-tiling} is a simple way to split a loop's iterations into tasks.  A
task-tiled loop executes a fixed number of iterations per task.  Task-tiling
amortizes task transitioning overhead, but risks executing more iterations in a
single task than the device's energy buffer can support, causing
non-termination.  Figure~\ref{fig:looptrick} shows the intermittent execution
(energy trace on left) of a loop computing a dot product using two
fixed tile sizes of five (Tile-5) and twelve (Tile-12). Tile-5 wastes work when four
iterations complete before a failure.  Tile-12 prevents forward
progress because the device buffers insufficient energy to complete twelve
iterations.

\subsubsection{Loop continuation}

\sonic's \emph{loop continuation} is an intermittence-safe optimization that avoids wasted work,
unnecessary data privatization, and task transition overheads in tasks
containing long-running loop nests. Loop continuation works by directly
modifying loop control variables and memory manipulated in a loop nest, rather
than splitting a long-running loop across tasks. 
%
Loop continuation
permits loops of arbitrary iteration count within a single task, with neither
non-termination nor excessive state management overhead.
%
Loop continuation stores a loop's control variables and data manipulated directly in non-volatile
memory \emph{without backing either up}.  When a loop continuation task restarts, its
(volatile) local variables are reinitialized at the task's start.  The loop
control variables, however, retain their state and the loop continues from the
last attempted iteration.
(Pseudocode for \sonic's loop continuation can be found in our technical report~\cite{sonic-arxiv}.)

\figLoopContinuation

\autoref{fig:tails_sonic} shows how loop continuation
works by storing the loop control state for \textsf{Task\_Convolve} in non-volatile memory.  
%
\sonic ensures that the loop's control variable {\tt\bfseries i} is correct by updating it at the end of the iteration and \emph{not resetting it upon re-execution.}
A power failure during or after the update to
the control variable may require the body of the loop nest to repeat a single
iteration, but %% never leaves activation data inconsistent and 
it never skips an iteration.

Figure~\ref{fig:looptrick} shows \sonic executing using loop
continuation.  Despite the power interruption, execution resumes on the ninth
loop iteration, rather than restarting the entire loop nest or every fifth iteration
like Tile-5 does.

\subsubsection{Idempotence tricks}
\label{sonic:sonic:idempotence}

Normally, restarting from the middle of a loop nest could leave manipulated
data partially updated and possibly inconsistent.  However, loop continuation
is safe because \sonic's runtime system ensures each loop iteration is idempotent
using either {\em loop-ordered buffering} or {\em sparse undo-logging}.
%
\sonic never requires an operation in an iteration to read a value
produced by another operation in the same iteration.
Thus, an iteration that repeatedly re-executes due to power interruption will always
see correct values.

\paragraph{Loop-ordered buffering}
Loop-ordered buffering is a double-buffering mechanism used in convolutional
layers (and dense fully-connected layers)
that ensures each loop iteration is idempotent without expensive redo-logging (cf., \cite{alpaca}). 
%
Since the MSP430 devices do not possess sophisticated caching
mechanisms, \emph{rather than optimizing for reuse and data locality, \sonic
  optimizes the number of items needed to commit.}
%
By re-ordering the loops in DNN inference and double-buffering partial activations as needed,
\sonic is able to \emph{completely eliminate} commits within a loop iteration.

Evaluating a sparse or dense convolution requires \sonic to apply a filter to a
layer's entire input activation matrix.  
%
\sonic orders loop iterations to apply each element of the filter to each
element of the input activation (i.e., multiplying them) before moving on
to the next element of the filter.
%
For idempotence,
\sonic writes the partially accumulated value to an intermediate output buffer,
rather than applying updates to the input matrix in-place.
%
After applying a single filter element to each entry in the input and storing
the partial result in the intermediate buffer, \sonic swaps the input
buffer with the intermediate buffer and moves on to the next filter value.

Since \sonic never reads and then writes to the same memory locations within an iteration,
it avoids the WAR problem described in \autoref{sonic:background}
and loop iterations are thus idempotent.
%
\autoref{fig:tails_sonic} shows how under loop-ordered buffering, \sonic never 
reads and writes to the same matrix buffer while computing a partial result in \textsf{Task\_Convolve}.
After finishing this task, \sonic transitions to \textsf{Task\_Next\_Filter}, which swaps the buffer pointers
and gets the next value to apply from the filter.

\paragraph{Sparse undo-logging}
While loop-ordered buffering is sufficient to ensure each loop iteration is idempotent,
it is sometimes unnecessarily wasteful.
%
The problem arises because loop-ordered buffering swaps between buffers after every task,
so it must copy data between buffers in case it is read in the future---even if the data has not been modified.
%
This copying is wasteful on sparse fully-connected layers,
where most filter weights are pruned and thus few activations are modified in a single iteration.
%
With loop-ordered buffering, \sonic ends up spending most of its time 
and energy copying unmodified activations between buffers.

To eliminate this inefficiency, \sonic introduces \emph{sparse undo-logging}
which ensures idempotence through undo-loggi\-ng instead of double buffering.
To ensure atomicity, sparse undo-logging tracks its progress through the loop via two index variables,
the \emph{read} and \emph{write} indices.
When applying a filter,
\sonic first copies the original, unmodified activation into a canonical memory location,
and then increments the read index.
\sonic then computes the modified activation and writes it back to the original activation buffer
(there is no separate output buffer).
Then it increments the write index and proceeds to the next iteration.
This two-phase approach guarantees correct execution,
since sparse undo-logging resumes computing the output value from the buffered original value if power fails in the middle of an update.

Sparse undo-logging ensures that the work per task grows with the number of modifications made,
not the size of the output buffer (unlike loop-ordered buffering).
However, sparse undo-logging doubles the number of memory writes per modified element,
so it is inefficient on dense layers where most data are modified.
In those cases, loop-ordered buffering is significantly more efficient.
We therefore only use sparse undo-logging in sparse fully-connected layers.
Finally, unlike prior task-based systems such as Alpaca,
sparse undo-logging ensures idempotence with \emph{constant} space overhead
and \emph{no} task transition between iterations.

\paragraph{Related work}
Prior work in persistent memory~\cite{elnawawy2017efficient} uses techniques similar to
our sparse undo-logging.
This work is in the high-performance domain,
and therefore focuses on cache locality and scheduling cache flushes and barriers.
In contrast, our prototype has no caches,
and we exploit this fact in loop-ordered buffering to re-arrange loops in a way that would destroy cache performance on conventional systems.
Moreover, \sonic is more selective than~\cite{elnawawy2017efficient},
only using undo-logging in sparse fully-connected layers where it outperforms double buffering.
