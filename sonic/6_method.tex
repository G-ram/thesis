\section{Methodology}
\label{sec:methodology}

We implement \sonic and \tails on the TI-MSP430FR5994~\cite{msp430fr5994} at
16MHz in the setup in \autoref{fig:hardware}.
The board is connected to a
Powercast P2210B~\cite{powercastboard} harvester 1m away from a 3W Powercaster
transmitter~\cite{powercasttransmitter}. We ran all configurations on
continuous power and on intermittent power with three different capacitor
sizes: 1mF, 50mF, and 100\textmu F.

\figSONICMethod

\figSONICEvalTime

\paragraph{Running code on the device}
We compile with MSPGCC 6.4 and use TI's MSPDriverlib for DMA and TI's DSPLib
for LEA. We use GCC instead of Alpaca's LLVM backend because LLVM lacks support
for 20-bit addressing and produces slower code for MSP430 than GCC.

\paragraph{Measurement}
We use a second MSP430FR5994 to measure intermittent executions.
%
GPIO pins on the measurement MCU connect through a level-shifter to the
intermittent device, allowing it to count reboots and signal when to start and
stop timing.
%
We automate measurement with a combination of software and hardware that
compiles a configuration binary, flashes the binary to the device, and
communicates with the measurement MCU to collect results.
%
The system actuates a relay to switch between continuous power for reprogramming
and intermittent power for testing.

\paragraph{Measuring energy}
By counting the number of charge cycles between GPIO pulses, we can determine
the amount of energy consumed in different code regions.
%
For a more fine-grained approach, we built a suite of 
microbenchmarks to count how many times a particular operation (e.g., a load from FRAM) can run in single charge cycle.
%
We then profile how many times each operation is invoked during inference
and scale by per-operation energy to get a detailed energy breakdown.

\paragraph{Baselines for comparison}
We compare \sonictails to four DNN inference implementations. The first
implementation is a standard, baseline implementation that does not tolerate
intermittent operation (it does not terminate).
The other three implementations are based on
Alpaca~\cite{alpaca} and split up loops by tiling iterations, as in \autoref{fig:looptrick}.
