\section{Discussion}
\label{sonic:discuss}
This chapter argued that intelligence ``beyond the edge'' will enable
new classes of IoT applications and presented \sonic, the software component to the new ULP sensor system stack.
% 
\sonic specializes intermittence support for DNN inference to guarantee correct
execution, regardless of power system, while reducing overheads by
up to $6.9\times$ and $12.2\times$, respectively, over the prior state-of-the-art.

However, our experience in building \sonictails also showed that there is a large
opportunity to accelerate inference on ULP sensor devices.
% 
But, current microcontrollers for energy-harvesting systems are poorly
suited to efficient inference.
% 
They are sequential, single-cycle processors,
and so spend very little of their energy on ``useful work''~\cite{horowitz:isscc14:energy-keynote}.
%
For example, by deducting the energy of \texttt{nop} instructions from \autoref{fig:sonic:evaluation:energy:micro},
we estimate that \sonic spends at least 40\% of its energy on instruction fetch and decode.
%
This cost is a waste in highly structured computations like DNN inference,
where overheads easily amortize over many operations. %~\cite{hameed2010understanding}.

LEA should bridge this efficiency gap, but unfortunately LEA has many limitations.
%
Invoking LEA is expensive. Each LEA invocation should therefore do as much work as
possible, but LEA's parallelism is limited by its small (4KB) SRAM buffer.
%
This small buffer also forces frequent DMA between SRAM and FRAM,
which cannot be overlapped with LEA execution
and does not support strided accesses or scatter-gather.
%
LEA also has surprising gaps in its support:
it does not support vector left-shift or scalar multiply,
forcing \tails to fall back to software.
%
In software, integer multiplication is a memory-mapped peripheral that takes
four instructions and nine cycles.
%
All told, these limitations cause \sonictails to spend much more energy than necessary.
% 
There is ample room to improve inference efficiency via a better architecture -- the subject of the next three chapters.
% 
\autoref{chapter:manic} discusses \manic a ULP vector-dataflow co-processor, \autoref{chapter:snafu} describes \snafu, a ULP CGRA generation framework and architecture, and \autoref{chapter:riptide} presents \riptide, a dataflow compiler and ULP CGRA.
% 
These architectures achieve much higher energy-efficiency v. existing scalar MCUs (like the MSP430) by leveraging vector and dataflow execution that minimize instruction and data supply energies.

